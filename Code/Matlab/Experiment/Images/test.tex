\documentclass{article}
\usepackage{xparse,pgfkeys}
\usepackage{xkeyval}
\usepackage{expl3}  % for LaTeX3 key handling

\usepackage[table]{xcolor} % für Farben in Tabellen
\usepackage{pgf}           % für Farbinterpolation
\usepackage{colortbl}      % für farbige Tabellenzellen
\usepackage{booktabs}      % schönere Tabellenlinien
\usepackage{pgffor}         % für Schleifen
\usepackage{etoolbox}       % für Listen-Operationen
\usepackage{pifont}
\usepackage{graphicx}
\usepackage{array}
\usepackage{makecell}


% \translate{x}{y}{content}
\newcommand{\translate}[3]{%
  \hspace*{#1}% move horizontally
  \raisebox{#2}{#3}% move vertically and render content
}

\definecolor{brightRed}{RGB}{255,200,200}
\definecolor{brightGreen}{RGB}{200,255,200}
\newcommand{\cmark}{
  \edef\colorname{brightGreen}
  \expandafter\cellcolor\expandafter{\colorname}
  \textcolor{green}{\reflectbox{\rotatebox{-90}{\ding{51}}}}
  }  % grünes Häkchen
\newcommand{\xmark}{
  \edef\colorname{brightRed}
  \expandafter\cellcolor\expandafter{\colorname}
  \textcolor{red}{\reflectbox{\rotatebox{-90}{\ding{55}}}}
  }    % rotes Kreuz

% Dynamische Zellenfärbung basierend auf Bewertung (1=rot, 10=grün)
\newcommand{\colcell}[1]{%
  \pgfmathsetmacro{\pct}{round((#1-1)/9*100)}% Prozentwert berechnen
  \edef\colorname{green!\pct!red}% Farbe zusammensetzen
  \expandafter\cellcolor\expandafter{\colorname}\textbf{#1}%
}







% Define a new command with optional key-value parameters
\makeatletter

% Define keys with default values (initially undefined)
\define@key{myparams}{width}{\def\myparam@width{#1}}
\define@key{myparams}{height}{\def\myparam@height{#1}}
\define@key{myparams}{color}{\def\myparam@color{#1}}

% Command to check if parameter was set
\newcommand{\ifparamset}[3]{%
  \@ifundefined{myparam@#1}{#3}{#2}%
}

% Reset all parameters to undefined
\newcommand{\resetparams}{%
  \let\myparam@width\undefined
  \let\myparam@height\undefined
  \let\myparam@color\undefined
}

% Your main macro with optional parameters
\newcommand{\mymacro}[2][]{%
  \resetparams% Clear previous values
  \setkeys{myparams}{#1}% Parse the optional parameters
  %
  % Now use the parameters
  \textbf{Content: #2}\\
  %
  % Check if parameters were set and act accordingly
  \ifparamset{width}{Width was set to: \myparam@width\\}{Width not set\\}
  \ifparamset{height}{Height was set to: \myparam@height\\}{Height not set\\}
  \ifparamset{color}{Color was set to: \myparam@color\\}{Color not set\\}
}

\makeatother







\ExplSyntaxOn
% Define a key family "mymacro"
\keys_define:nn { mymacro }
  {
    width  .dim_set:N = \l_mymacro_width_dim,
    height .dim_set:N = \l_mymacro_height_dim,
    width  .initial:n = 1cm,
    height .initial:n = 1cm,
  }

% Define the command
\NewDocumentCommand{\mybox}{O{}}
 {
   % Parse the key=value options
   \keys_set:nn { mymacro } { #1 }
   % Example usage of parameters:
   \fbox{\parbox[c][\l_mymacro_height_dim][c]{\l_mymacro_width_dim}{Content}}
 }
\ExplSyntaxOff

\begin{document}

\subsection{No parameters}
\mymacro{Hello World}

\subsection{With width only}
\mymacro[width=5]{Hello World}

\subsection{With multiple parameters}
\mymacro[width=10, height=20, color=blue]{Hello World}

\subsection{With some parameters}
\mymacro[width=15, color=red]{Hello World}





\mybox[width=3cm, height=2cm]

\bigskip

\mybox[height=1cm, width=4cm] % order doesn’t matter!


\def\rowA{8,7,9,10}
\def\rowB{5,6,4,7}



\begin{table}[h]

%\newcommand{\cc}[1]{\translate{0.0cm}{0.0cm}{\reflectbox{\rotatebox{-90}{\colcell{#1}}}}}
\newcommand{\cc}[1]{\reflectbox{\rotatebox{-90}{\colcell{#1}}}}
\centering
\renewcommand{\arraystretch}{1.5}


\rotatebox{90}{
  \reflectbox{
\begin{tabular}{r|cccc|ccccc|ccc|cc|}
\toprule
\textbf{} &
\multicolumn{4}{c}{\reflectbox{Reglerverhalten}} &
\multicolumn{5}{c}{\reflectbox{Systemvoraussetzung}} &
\multicolumn{3}{c}{\reflectbox{Parametrisierung}} &
\multicolumn{2}{c}{\reflectbox{Automatisierungsgrad}} \\
\cline{2-13} % <-- horizontale Linie nur zwischen Spalte 2 und 4

\textbf{} &
\rotatebox{90}{\reflectbox{\textbf{Stabilität}}} &
\rotatebox{90}{\reflectbox{\textbf{Zeitverhalten}}} &
\rotatebox{90}{\reflectbox{\textbf{Stationär}}} &
\rotatebox{90}{\reflectbox{\textbf{Robustheit}}} &

\rotatebox{90}{\reflectbox{\textbf{Instabil möglich}}} &
\rotatebox{90}{\reflectbox{\textbf{MIMO fähig}}} &
\rotatebox{90}{\reflectbox{\textbf{Nichtlinear fähig}}} &
\rotatebox{90}{\reflectbox{\textbf{Zeitvariant fähig}}} &
\rotatebox{90}{\reflectbox{\textbf{Struktur offenheit}}} &

\rotatebox{90}{\reflectbox{\textbf{Benutzerfreundlichkeit}}} &
\rotatebox{90}{\reflectbox{\textbf{Reglerstruktur Vorgabe}}} &
\rotatebox{90}{\reflectbox{\textbf{Flexibilität der Optimierungsvorgaben}}} &

\rotatebox{90}{\reflectbox{\textbf{Tools/Technologien}}} &
\rotatebox{90}{\reflectbox{\textbf{Eingriff des Anwenders}}}\\
\midrule
\reflectbox{Lambda-Tuning}         & \cc{8}  & \cc{8}  & \cc{10}  & \cc{8} &   \xmark  & \cmark   & \xmark  & \xmark & \xmark  \\
\reflectbox{Pol-Platzierung}      \\
\reflectbox{LQR}    \\
\reflectbox{Ziegler-Nichols}  \\
\reflectbox{Matlab systune} \\
\reflectbox{Matlab looptune} \\
\reflectbox{Genetischen Algorithmen} \\
\reflectbox{Neuroevolution} \\

\bottomrule
\end{tabular}
}
}
\caption{Bewertungen (1 = rot, 10 = grün)}
\end{table}









\pagebreak
\begin{table}[h]

\newcommand{\cc}[1]{\translate{-0.5cm}{0cm}{\rotatebox{-90}{\colcell{#1}}}}
\centering
\renewcommand{\arraystretch}{1.5}

\begin{tabular}{r|cccc|ccccc|ccc|cc|}
\toprule
\textbf{} &
\multicolumn{4}{c}{Reglerverhalten} &
\multicolumn{5}{c}{Systemvoraussetzung} &
\multicolumn{3}{c}{Parametrisierung} &
\multicolumn{2}{c}{Automatisierungsgrad} \\
\cline{2-13} % <-- horizontale Linie nur zwischen Spalte 2 und 4

\textbf{} &
\rotatebox{-90}{\textbf{Stabilität}} &
\rotatebox{-90}{\textbf{Zeitverhalten}} &
\rotatebox{-90}{\textbf{Stationär}} &
\rotatebox{-90}{\textbf{Robustheit}} &

\rotatebox{-90}{\textbf{Instabil möglich}} &
\rotatebox{-90}{\textbf{MIMO fähig}} &
\rotatebox{-90}{\textbf{Nichtlinear fähig}} &
\rotatebox{-90}{\textbf{Zeitvariant fähig}} &
\rotatebox{-90}{\textbf{Struktur offenheit}} &

\rotatebox{-90}{\textbf{Benutzerfreundlichkeit}} &
\rotatebox{-90}{\textbf{Reglerstruktur Vorgabe}} &
\rotatebox{-90}{\textbf{Flexibilität der Optimierungsvorgaben}} &

\rotatebox{-90}{\textbf{Tools/Technologien}} &
\rotatebox{-90}{\textbf{Eingriff des Anwenders}}\\
\midrule
Lambda-Tuning         & \cc{8}  & \cc{8}  & \cc{10}  & \cc{8} &   \xmark  & \cmark   & \xmark  & \xmark & \xmark  \\
Pol-Platzierung      \\
LQR    \\
Ziegler-Nichols  \\
Matlab systune \\
Matlab looptune \\
Genetischen Algorithmen \\
Neuroevolution \\

\bottomrule
\end{tabular}

\caption{Bewertungen (1 = rot, 10 = grün)}
\end{table}


\begin{tabular}{|c|c|}
\hline
\textbf{Column 1} & \textbf{Column 2} \\ \hline
\makecell[c]{Centered \\ multi-line \\ text} & \makecell[c]{Also \\ centered} \\ \hline
\end{tabular}

%\def\rowA{8,7,9, 10}
%\def\rowB{5,6,4,7}
%\def\rowC{3,8,9,5}
%
%
%\begin{tabular}{|l|c|c|c|c|}
%\hline
%\textbf{Thema} & \textbf{K1} & \textbf{K2} & \textbf{K3} & \textbf{K4} \\
%\hline
%\AddRow{Design}{\rowA}
%\AddRow{Usability}{\rowB}
%\AddRow{Performance}{\rowC}
%\end{tabular}

\end{document}
