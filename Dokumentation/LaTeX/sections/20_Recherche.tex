\section{Recherche}
%\newcommand{\rechercheMiniPage}[2]{\minipagedOrBelowEachOther{#1}{#2}}
%\newcommand{\rechercheMiniPage}[3]{\minipagedOrBelowEachOther{#1#2}{#3}}
%\newcommand{\rechercheMiniPage}[3]{#1 #2 #3}

In der Recherche wird nach möglichen automatisierbaren Methoden oder Algorithmen für die Reglersynthese gesucht.

\subsection{Recherchierte Methoden}
Nachfolgend sind die recherchierten Methoden zusammenfassend aufgeführt und mit Vor- und Nachteilen gekennzeichnet.

\horizontalLine
\subsubsection{Lambda-Tuning}
\minipagedOrBelowEachOther
{
  Lambda-Tuning \cite{lambdaMethode} basiert auf dem Prinzip der \gls{Pol-Nullstelle-Kürzung}.
  Die Methode ermöglicht es, mit Hilfe eines einzigen Parameters, das Verhalten des \gls{System}s 
  einfach zu beeinflussen. 
  Um die Methode anwenden zu können, muss die Regelstrecke eine spezifische Form aufweisen.
  Zudem muss das offene System bereits stabil sein, da die Pol-Nullstelle-Kürzung ein instabiles System nicht
  stabilisieren kann.
  Dies schränkt den praktischen Gebrauch dieser Methode auf wenige Anwendungsgebiete ein.
}
{
  \begin{itemize}
    \item[+] Verwendet nur einen Optimierungsparameter: $\lambda$
    \item[+] Einfach automatisierbar
    \item[+] Benutzer muss nur einen Parameter einstellen. Dies kann mit einem GUI-Element wie z.B. einem Slider umgesetzt werden 
    \item[--] Die Regelstrecke muss stabil sein
    \item[--] Die Regelstrecke muss eine spezifische Form aufweisen
    \item[--] Die Regelstrecke muss linear und zeitinvariant sein
    \item[--] Nicht geeignet für \gls{MIMO}-Systeme
    \item[--] Reglerstruktur ist auf PID-Regler beschränkt
  \end{itemize}
}


\horizontalLine
\subsubsection{Polplatzierung}
\minipagedOrBelowEachOther
{
Mit Hilfe der \gls{PolePlacement} lassen sich die Pole des geschlossenen 
Regelkreises frei platzieren. Mit der Möglichkeit Pole frei zu platzieren kann ein guter Regler ausgelegt werden.
Die Schwierigkeit besteht jedoch darin, zu wissen wo die Pole platziert werden sollen.
Mit wenig Erfahrung lassen sich die richtigen Polstellen nicht finden.
Den besten Regler mit dieser Methode zu finden ist also sehr schwierig. 
Ausserdem wird voller Zugriff auf den gesamten Zustandsvektor benötigt, was nicht für jedes System immer möglich ist, 
weshalb in solchen Fällen eine zusätzliche Ergänzung eines \gls{Beobachter}s benötigt wird.
}
{
\begin{itemize}
  \item[+] Volle Kontrolle über die Pole des geschlossenen Regelkreises
  \item[+] Einfach automatisierbar 
  \item[+] Kann auf instabile Systeme angewendet werden 
  \item[+] Geeignet für MIMO-Systeme
  \item[+] Die Platzierung der Pole kann mit Hilfe eines GUI-Editors graphisch erfolgen
  \item[--] Benötigt ein intuitives Verständnis für den Einfluss der Polpositionen auf das System
  \item[--] Die Regelstrecke muss linear und zeitinvariant sein
\end{itemize}
}

\horizontalLine
\subsubsection{Linear Quadratic Regulator (LQR)}
\minipagedOrBelowEachOther
{
  Findet den optimalen Zustandsregler für die zu definierende, quadratische Kostenfunktion.
  Für die Kostenfunktion können die Zustandswerte, die Eingangssignale und das Kreuzprodukt 
  der Zustände mit den Eingängen, gewichtet verwendet werden.
  Es wird voller Zugriff auf den gesamten Zustandsvektor benötigt, was nicht für jedes System immer möglich ist, 
  weshalb in solchen Fällen eine zusätzliche Ergänzung eines Beobachters benötigt wird.
  Eine Schwierigkeit dieser Methode ist es, die passenden Gewichtungen der Kostenfunktion zu finden.
}
{
\begin{itemize}
  \item[+] Volle Kontrolle über die Pole des geschlossenen Regelkreises
  \item[+] Einfach automatisierbar 
  \item[+] Kann auf instabile Systeme angewendet werden 
  \item[+] Geeignet für MIMO-Systeme 
  \item[--] System sollte in Regelungsnormalform vorliegen damit das bestimmen der Kostenmatrizen einfacher ist
  \item[--] Bestimmung der Gewichtungsmatrizen ist nicht intuitiv 
  \item[--] Die Regelstrecke muss linear und zeitinvariant sein
\end{itemize}
}


\horizontalLine
\subsubsection{Ziegler-Nichols}
\minipagedOrBelowEachOther
{
  Mit Hilfe von Ziegler-Nichols und ähnlichen Varianten wie:
  \begin{itemize}
    \item Ziegler-Nichols mit Sprungantwort \cite{einfuehrungInDieRegelungstechnik}
    \item Ziegler-Nichols mit Hilfe der Stabilitätsgrenze \cite{zieglerNicholsFreq1} \cite{zieglerNicholsFreq2}
    \item Chien-Hrones-Reswick \cite{einfuehrungInDieRegelungstechnik}
    \item Latzel \cite{einfuehrungInDieRegelungstechnik}
  \end{itemize}
  kann ein einfacher PID-Regler mit empirischen Daten entworfen werden.
  Die Methoden sind sehr einfach in der Anwendung, jedoch nicht optimal und in der Praxis auch nicht immer umsetzbar.
  Ein System kann eventuell nicht immer, mit einem Sprung angeregt werden, um die benötigten Messwerte zu erhalten.
  Für MIMO-Systeme sind diese Methoden ungeeignet, weil der Aufwand sehr gross wäre, 
  für jede Übertragungsfunktion, von jedem Eingang zu jedem Ausgang, ein Regler mit diesen Methoden einzeln zu bestimmen. 
  Auch wenn dieser Aufwand betrieben wird, wären die gegenseitigen Einflüsse der Eingänge nicht berücksichtigt.
  Das würde nur bei Systemen klappen die sich komplett in mehrere \gls{SISO}-Systeme zerlegen lassen.
  Dann wäre das Problem auf mehrere SISO-Systeme aufgeteilt und könnten individuell betrachtet werden.
  Bei den meisten MIMO-Systemen sind die Zustände jedoch abhängig von mehreren Eingängen und Zuständen.
  Für instabile Systeme sind diese Methoden nicht anwendbar da die benötigten Daten nicht ermittelt werden können.
}
{
\begin{itemize}
  \item[+] Benötigt kein Modell der Regelstrecke
  \item[+] Einfach automatisierbar und mit einem Modell besonders einfach automatisierbar
  \item[+] Benötigt keine komplexe Mathematik
  \item[+] GUI-Gestaltung ist sehr einfach da es nur wenige tabellarische Auswahlmöglichkeiten gibt
  \item[--] Kann nicht auf instabile Systeme angewendet werden
  \item[--] Kann nur für \glspl{minimalphasiges System} angewendet werden
  \item[--] Ziegler-Nichols mit Hilfe der Stabilitätsgrenze kann ohne Modell am realen Regelstrecke kaum umgesetzt werden
  \item[--] Die Methoden dienen nur für einen ersten Reglerentwurf und sind keineswegs optimal
  \item[--] Nur Systeme der Form PT2 oder PT3 geeignet 
  \item[--] Bei schnellen Regelstrecken kaum umsetzbar
  \item[--] Nicht geeignet für MIMO-Systeme
  \item[--] Reglerstruktur ist auf PID-Regler beschränkt
\end{itemize}
}

\newpage
%\horizontalLine
\subsubsection{MATLAB \textit{systune}}
\label{sec:rechercheSystune}
\minipagedOrBelowEachOther
{
  Mit \textit{systune} \cite{matlabSystune} kann ein komplettes System, flexibel auf das gewünschte Verhalten
  optimiert werden. \textit{Systune} verwendet dafür ein $genss$-Modell \cite{matlabGenss}.
  Es handelt sich dabei um ein verallgemeinertes Zustandsraum Modell, das zusätzlich
  noch Optimierungsparameter beinhaltet, welche durch MATLAB optimiert werden.
  Für die Optimierung werden Optimierungsziele \cite{matlabTuningGoals} definiert,
  welche auf zwei Arten verwendet werden können.
  \begin{itemize}
    \item \textbf{Harte Ziele} \cite{matlabSystuneHardReq} werden priorisiert bei der Optimierung
    \item \textbf{Weiche Ziele} \cite{matlabSystuneSoftReq} können vernachlässigt werden um die \textbf{harten Ziele} zu erreichen
  \end{itemize}
  
  MATLAB \textit{systune} bietet als Ergebnis der Optimierung nicht nur die optimierten Parameter, 
  sondern auch eine Auswertung der Optimierung für jedes Optimierungsziel.
}
{
\begin{itemize}
  \item[+] Sehr flexibel
  \item[+] Einfach automatisierbar
  \item[+] Optimierung kann detailliert vorgenommen werden
  \item[+] Kann auf instabile Systeme angewendet werden 
  \item[+] Geeignet für MIMO-Systeme
  \item[+] Kann beliebig viele Parameter optimieren
  %\item[+] Grafischer Editor wird durch MATLAB bereitgestellt
  \item[--] Benötigt MATLAB
  \item[--] Das Modell muss als $genss$-Modell \cite{matlabGenss} vorliegen
  \item[--] GUI-Gestaltung zur Eingabe der Optimierungsziele ist komplex
  \item[--] Reglerstruktur muss vorgegeben werden
\end{itemize}
}

\horizontalLine
\subsubsection{MATLAB \textit{looptune}}
\minipagedOrBelowEachOther
{
  Mit \textit{looptune} \cite{matlabLooptune} wird eine Feedbackschleife optimiert.
  Looptune verwendet für die Optimierung zum einen die Ziel-crossover Frequenz/Frequenzbereich und zum anderen 
  können auch Optimierungsziele \cite{matlabTuningGoals} definiert werden.
  MATLAB \textit{looptune} bietet als Ergebnis der Optimierung nicht nur die optimierten Parameter, 
  sondern auch eine Auswertung der Optimierung für jedes Optimierungsziel.
}
{
\begin{itemize}
  \item[+] Kann auf instabile Systeme angewendet werden 
  \item[+] Geeignet für MIMO-Systeme
  \item[+] Kann beliebig viele Parameter optimieren
  \item[--] Benötigt MATLAB
  \item[--] Das Modell muss als $genss$-Modell \cite{matlabGenss} vorliegen 
  \item[--] GUI-Gestaltung zur Eingabe der Optimierungsziele ist komplex
  \item[--] Reglerstruktur muss vorgegeben werden
\end{itemize}
}

\horizontalLine
\subsubsection{Genetischer Algorithmus}
\minipagedOrBelowEachOther
{
  Ein genetischer Algorithmus \gls{GA} \cite{mirjalili2018genetic} ist eine Optimierungsmethode, basierend auf der natürlichen Evolution.
  Ein GA kann eine beliebige Anzahl an Parametern optimieren und benötigt dafür nur eine Bewertungsfunktion.
  Die Idee besteht darin, eine \gls{Population} an Individuen mit zufälligen Parametern zu erstellen und diese einzeln mit Hilfe der Bewertungsfunktion zu bewerten.
  Die besten Individuen werden ausgewählt und miteinander gekreuzt und mutiert um neue Individuen zu erzeugen.
  Dieser Vorgang wird über mehrere Generationen wiederholt bis ein Abbruchkriterium erfüllt ist.
  Diese Methode kann verwendet werden um die Parameter eines PID-Reglers oder auch eines Reglers mit beliebiger Struktur zu optimieren \cite{geneticPIDSynthesis}.
  GA ist grundsätzlich nur für \gls{Maximierungsproblem}e ausgelegt, mit ein paar Tricks können aber auch 
  \gls{Minimierungsproblem}e gelöst werden.
}
{
\begin{itemize}
  \item[+] Geeignet für MIMO-Systeme
  \item[+] Kann beliebig viele Parameter optimieren
  \item[+] Sehr flexibel bezüglich der Reglerstruktur
  \item[+] Kann auf nichtlineare Regelstrecken und Regler angewendet werden
  \item[+] Der Algorithmus allein kann ohne Modellwissen arbeiten
  \item[+] Gut automatisierbar 
  \item[+] Ableitung der Bewertungsfunktion wird nicht benötigt
  \item[--] Da die Optimierung auf reiner Simulation basiert, muss jedoch ein Modell der Regelstrecke vorliegen.
            Eine Optimierung am realen System wäre technisch zwar möglich, praktisch jedoch nicht umsetzbar.
  \item[--] Kann nicht auf zeitvariante Regelstrecken angewendet werden 
  \item[--] Instabile Systeme können stabilisiert werden, jedoch ist dies nicht garantiert
  \item[--] Benötigt eine gut definierte Bewertungsfunktion. Es kann sehr schwierig sein eine gute Bewertungsfunktion zu definieren:\\
  Gewünschtes Transientenverhalten, Überschwingen, Robustheitsanforderungen, etc. müssen während des Bewertens in eine Zahl zusammengefasst werden.
  \item[--] Optimierung kann lange dauern je nach Anzahl Parameter, Komplexität der Regelstrecke und Population/Generationen Grösse
  \item[--] Es kann nicht garantiert werden, einen optimalen Regler zu finden  
  \item[--] Reglerstruktur muss vorgegeben werden
\end{itemize}
}

\newpage
\subsubsection{Neuroevolution}
\minipagedOrBelowEachOther
{
  \gls{NEAT} ist ein Ansatz, der evolutionäre Algorithmen mit \gls{NN} kombiniert. 
  Dabei werden neuronale Netzwerke als Individuen in einer Population betrachtet und durch genetische Algorithmen optimiert. 
  Die Methode ist eine Erweiterung des klassischen GA und ermöglicht es, sowohl die Struktur als auch die Gewichtungen der Netzwerke zu optimieren.
  Der Anwender muss sich keine Reglerstruktur überlegen und vorgeben, der Algorithmus verändert die Struktur laufend um bessere Lösungen zu finden.
  Im Gegensatz zu einem vordefinierten NN, welches den Regler darstellen soll, 
  wird durch diesen Algorithmus ein unnötiges aufblasen des Netzwerks vermieden. \cite{trujillo2016neat}
  Dies ist ein häufiges Problem bei der Verwendung von NN, 
  einerseits ist zu Beginn nicht bekannt wie gross ein NN sein muss um die Aufgabe zu lösen, 
  andererseits kann ein zu grosses NN dazu führen, zu komplex für die Implementierung auf dem Zielsystem zu sein.
  NEAT startet bei einer minimalen Struktur und wächst nur so weit wie nötig um die Aufgabe zu lösen.
  Eine Beispielanwendung kann in der Literatur gefunden werden \cite{lauckner4644203neat}. (Nicht peer-reviewed)
}
{
\begin{itemize}
  \item[+] Geeignet für MIMO-Systeme
  \item[+] Kann beliebig viele Parameter optimieren
  \item[+] Sehr flexibel bezüglich der Reglerstruktur
  \item[+] Kann auf nichtlineare Regelstrecken angewendet werden
  \item[+] Der Algorithmus allein kann ohne Modellwissen arbeiten
  \item[+] Die Reglerstruktur muss nicht vorgegeben werden 
  \item[+] Sehr gut automatisierbar 
  \item[+] Ableitung der Bewertungsfunktion wird nicht benötigt
  \item[--] Da die Optimierung auf reiner Simulation basiert, muss jedoch ein Modell der Regelstrecke vorliegen.
            Eine Optimierung am realen System wäre technisch zwar möglich, praktisch jedoch nicht umsetzbar.
  \item[--] Kann nicht auf zeitvariante Regelstrecken angewendet werden 
  \item[--] Instabile Systeme können stabilisiert werden, jedoch ist dies nicht garantiert
  \item[--] Benötigt eine gut definierte Bewertungsfunktion. Es kann sehr schwierig sein eine gute Bewertungsfunktion zu definieren: \\
  Gewünschtes Transientenverhalten, Überschwingen, Robustheitsanforderungen, etc. müssen während des Bewertens in eine Zahl zusammengefasst werden.
  \item[--] Optimierung kann lange dauern je nach Anzahl Parameter, Komplexität der Regelstrecke und Population/Generationen Grösse
  \item[--] Es kann nicht garantiert werden, einen optimalen Regler zu finden
\end{itemize}
}


\horizontalLine
\subsubsection{Differential Evolution}
\minipagedOrBelowEachOther
{
  Der Differential Evolution Algorithmus \gls{DE} \cite{storn1997differentialEvolution} ist ein ähnlich wie der GA ein evolutionärer Algorithmus zur Optimierung von Parametern.
  Diese Methode hat wie der GA auch, den Vorteil, dass keine Ableitungen der Bewertungsfunktion benötigt wird.
  DE kann eine beliebige Anzahl an Parametern optimieren und benötigt dafür nur eine Bewertungsfunktion.
  Sowohl Maximierungs- als auch Minimierungsprobleme können mit dieser Methode gelöst werden.
  Ein Nachteil des DE ist jedoch, dass die Anfangswerte der Parameter 
  aus Sicht der ganzen Population, den gesamten Suchraum aufspannen müssen.
  Ist dies nicht der Fall, kann das optimale Ergebnis unter Umständen nicht gefunden werden da
  der Suchraum nicht in alle Richtungen erkundet werden kann.
}
{
\begin{itemize}
  \item[+] Geeignet für MIMO-Systeme
  \item[+] Kann beliebig viele Parameter optimieren
  \item[+] Sehr flexibel bezüglich der Reglerstruktur
  \item[+] Kann auf nichtlineare Regelstrecken und Regler angewendet werden
  \item[+] Der Algorithmus allein kann ohne Modellwissen arbeiten
  \item[+] Gut automatisierbar 
  \item[+] Ableitung der Bewertungsfunktion wird nicht benötigt
  \item[--] Da die Optimierung auf reiner Simulation basiert, muss jedoch ein Modell der Regelstrecke vorliegen.
            Eine Optimierung am realen System wäre technisch zwar möglich, praktisch jedoch nicht umsetzbar.
  \item[--] Kann nicht auf zeitvariante Regelstrecken angewendet werden 
  \item[--] Instabile Systeme können stabilisiert werden, jedoch ist dies nicht garantiert
  \item[--] Benötigt eine gut definierte Bewertungsfunktion. Es kann sehr schwierig sein eine gute Bewertungsfunktion zu definieren: \\
  Gewünschtes Transientenverhalten, Überschwingen, Robustheitsanforderungen, etc. müssen während des Bewertens in eine Zahl zusammengefasst werden.
  \item[--] Es kann nicht garantiert werden, einen optimalen Regler zu finden  
  \item[--] Reglerstruktur muss vorgegeben werden
\end{itemize}
}

\newpage
\subsection{Ergebnis}
Wie im Kapitel~\fullref{sec:Methodik} bereits erwähnt, wird für die Ermittlung der zu vertiefenden Methoden folgende Punkte betrachtet.
\methodeCriterias


\newpage
\subsubsection{Bewertungstabelle}
In der \ref{tab:Bewertungstabelle} ist die Bewertung der betrachteten Methoden in einer Tabelle dargestellt.
Es ist jedoch schwierig eine Quantisierung der einzelnen Kriterien vorzunehmen,
die Tabelle soll daher nur als grobe Orientierung dienen.


\begin{table}[H]
    \centering
        \includegraphics[width=0.98\linewidth]{images/Bewertung.pdf}
        \caption{Bewertungstabelle der betrachteten Methoden}
    \label{tab:Bewertungstabelle}
\end{table}
\newpage

\subsection{Kombination mehrerer Methoden}
\label{sec:KombinationMehrererMethoden}
\minipagedOrBelowEachOther
{
Es ist natürlich nicht vorgeschrieben nur eine Methode allein zu verwenden.
Das Kombinieren von mehreren Methoden kann sehr viele Vorteile bringen. 
}{}

\horizontalLine
\paragraph*{Beispiel 1:\\Ziegler-Nichols + genetischer Algorithmus für PID-Tuning}
\minipagedOrBelowEachOther
{
  Ein klassischer Ansatz zur schnellen Parametrierung eines PID-Reglers ist die Ziegler-Nichols Methode.
  Diese Methode liefert eine erste Abschätzung der Reglerparameter basierend auf dem offenen Regelkreisverhalten.
  Allerdings sind die mit Ziegler-Nichols bestimmten Parameter oft nicht optimal für spezifische Anforderungen.
}
{
  An dem Punkt kann ein \gls{GA} ansetzen, um die Reglerparameter weiter auf die gewünschten Anforderungen zu optimieren.
  Weil Ziegler-Nichols für \gls{MIMO} und instabile -\gls{System}e nicht geeignet ist, 
  unterliegt auch die Kombination dieser beiden Methoden dieser Einschränkung. 
}


\horizontalLine
\paragraph*{Beispiel 2:\\Ziegler-Nichols + diverse andere Optimierungsmethoden für PID-Tuning}
\minipagedOrBelowEachOther
{
Wie im vorherigen Beispiel beschrieben, kann Ziegler-Nichols eine gute erste Abschätzung der Reglerparameter liefern.
Für die weitere Optimierung der Regelparameter können aber auch andere Optimierungsmethoden verwendet werden:
}
{}
\begin{itemize}
    \item Differential Evolution \cite{storn1997differentialEvolution}
    \item Gradient Descent
    \item Gradient Descent with Momentum
\end{itemize}

\horizontalLine
\paragraph*{Beispiel 3:\\Evolutionärer Algorithmus + MATLAB \textit{systune} für komplexe Reglerstrukturen}
\minipagedOrBelowEachOther
{
Da für die Optimierung mit MATLAB \textit{systune} \cite{matlabSystune} ein initiales Modell mit Reglerstruktur benötigt wird,
kann eine Schwierigkeit darin bestehen, eine geeignete Reglerstruktur zu finden.
Ein evolutionärer Algorithmus wie \gls{NEAT} könnte verwendet werden, 
um verschiedene Reglerstrukturen zu generieren, welche anschliessend
mit MATLAB \textit{systune} optimiert werden.
}
{
Die Optimierungsergebnisse von \textit{systune} können in die Bewertungsfunktion des evolutionären Algorithmus einfliessen, 
um die besten Reglerstrukturen zu identifizieren.
}


\horizontalLine
\subsection{Auswahl der zu vertiefenden Methoden}
\subsubsection{MATLAB \textit{systune}}
%\minipagedOrBelowEachOther
{
  Aus allen betrachteten Methoden sticht~\fullref{sec:rechercheSystune} heraus, es handelt sich dabei nicht um einen spezifischen 
  Algorithmus für die \gls{Reglersynthese} sondern um ein Tool aus der MATLAB Control System Toolbox \cite{matlabControlSystemToolbox}.
  Mit diesem Tool können verschiedene Reglerstrukturen und Systemmodelle optimiert werden.
  Die Flexibilität und die einfache Automatisierung machen dieses Tool zu einer guten Wahl für die weitere Vertiefung.
}
{

}
\minipagedOrBelowEachOther
{
  \paragraph*{Voraussetzungen der Systeme}
  \begin{itemize}
    \item Das System muss als $genss$-Modell \cite{matlabGenss} modelliert sein
    \item Für die Optimierung können beliebig viele Parameter optimiert werden
    \item Die zu optimierenden Parameter können überall im Regelkreis liegen
    \item Das System sollte in der Lage sein, die definierten Optimierungsziele zu erreichen
    \item Stabile als auch instabile Systeme können optimiert werden
    \item SISO- als auch MIMO-Systeme können optimiert werden
    \item Nur lineare- und Zeitinvariante Systeme können optimiert werden
  \end{itemize}
}
{
\paragraph*{Ermöglichtes Reglerverhalten}
MATLAB \textit{systune} \cite{matlabSystune} ermöglicht die Definition von verschiedenen Optimierungszielen \cite{matlabTuningGoals},
welche spezifisch auf das gewünschte Reglerverhalten abgestimmt werden können.


\paragraph*{Die Schwierigkeit der Parametrisierung}
Die Parametrisierung erfolgt über Optimierungsziele \cite{matlabTuningGoals}.
Einerseits können beliebig viele Optimierungsziele definiert werden, andererseits können diese auch noch in harte und weiche Ziele
unterteilt werden.
Dadurch kann die Optimierung sehr flexibel gestaltet werden.
Jedes Optimierungsziel beinhaltet selbst noch weitere Parameter zur Konfiguration.
Das macht die Parametrisierung etwas komplexer, jedoch auch sehr mächtig.
}

%\paragraph{Ermöglichtes Reglerverhalten}
%\minipagedOrBelowEachOther
%{
%MATLAB systune \cite{matlabSystune} ermöglicht die Definition von verschiedenen Optimierungszielen \cite{matlabTuningGoals},
%welche sehr spezifisch auf das gewünschte Reglerverhalten abgestimmt werden können.
%}
%{}

%\paragraph{Voraussetzungen der Systeme}
%\begin{itemize}
%  \item Das \gls{System} muss als $genss$ \cite{matlabGenss} modelliert sein
%  \item Für die Optimierung können beliebig viele Parameter optimiert werden
%  \item Die zu optimierenden Parameter können überall im Regelkreis liegen
%  \item Das \gls{System} sollte in der Lage sein, die definierten Optimierungsziele zu erreichen
%  \item Stabile als auch instabile \gls{System}e können optimiert werden
%  \item \gls{SISO}- als auch \gls{MIMO}-\gls{System}e können optimiert werden
%  \item Nur lineare \gls{System}e können optimiert werden
%\end{itemize}

%\paragraph{Die Schwierigkeit der Parametrierung}
%\minipagedOrBelowEachOther
%{
%  Die Parametrierung erfolgt über Optimierungsziele \cite{matlabTuningGoals}.
%  Einerseits können beliebig viele Optimierungsziele definiert werden, andererseits können diese auch noch in harte und weiche Ziele
%  unterteilt werden. 
%}
%{
%  Dadurch kann die Optimierung sehr flexibel gestaltet werden.
%  Jedes Optimierungsziel beinhaltet selbst noch weitere Parameter zur Konfiguration.
%  Das macht die Parametrierung etwas komplexer, jedoch auch sehr mächtig.
%}

\newpage
\paragraph*{Automatisierungsgrad}
Um die Optimierung durchzuführen,
muss der Anwender einige Informationen bereitstellen.
Dazu gehören:
\begin{itemize}
  \item Definition des $genss$-Modells \cite{matlabGenss}, welches bereits den zu optimierenden Regler enthält
  \item Definition der Optimierungsziele \cite{matlabTuningGoals} und deren Parametrisierung
  \item Bestimmung welche Optimierungsziele hart und welche weich sind
  \item (Optional) Definition von Startwerten für die zu optimierenden Parameter
\end{itemize}
\minipagedOrBelowEachOther
{
Sobald diese Informationen bereitgestellt sind,
kann der Optimierungsprozess gestartet werden.
Das Tool übernimmt dann die komplette Optimierung und liefert ein 
$genss$-Modell \cite{matlabGenss} mit den optimierten Parametern zurück.
Ausserdem wird eine Auswertung der Optimierung für jedes Optimierungsziel bereitgestellt.
}
{

}




\horizontalLine
\subsubsection{Genetischer Algorithmus für PID-Tuning}
\label{sec:GAforPIDTuning}

\paragraph*{Ermöglichtes Reglerverhalten}
\minipagedOrBelowEachOther
{
Benutzerdefinierte Bewertungsfunktionen ermöglichen die Optimierung des PID-Reglers auf sehr spezifische Anforderungen.
}
{
Die Theorie zum GA ist im Kapitel\\
~\fullref{sec:Description_GenetischerAlgorithmus} beschrieben.
}

\paragraph*{Voraussetzungen der Systeme}
\begin{itemize}
  \item Modell benötigt für den \gls{GA}
  %\item System muss stabil sein
  %\item System muss \gls{minimalphasiges System} sein
  \item System kann \gls{MIMO} sein
  \item Reglerstruktur ist beliebig % auf PID-Regler beschränkt
\end{itemize}


\minipagedOrBelowEachOther
{
  \paragraph*{Die Schwierigkeit der Parametrisierung}
  Der Anwender muss eine Bewertungsfunktion definieren welche aus beliebigen Teilfunktionen bestehen kann.
}
{
  \paragraph*{Automatisierungsgrad}
  Mit bestehender Bewertungsfunktion und vordefiniertem Modell inklusive Reglerstruktur,
  kann die Optimierung gestartet werden.
  Die Optimierung ist ein iterativer Prozess welcher solange wiederholt wird, 
  bis das Ergebnis zufriedenstellend ist.
}





\horizontalLine
\subsubsection{Differential Evolution für PID-Tuning}
\minipagedOrBelowEachOther
{
  \gls{DE} hat ziemlich ähnliche Eigenschaften wie der \gls{GA}.
  Der Unterschied besteht hauptsächlich darin, wie die Selektion, Kreuzung und Mutation durchgeführt werden.\\


  Weil die beiden Algorithmen so ähnlich sind, wird hier nicht weiter auf die Eigenschaften eingegangen,
  da sie bereits im Abschnitt\\
  ~\fullref{sec:GAforPIDTuning} beschrieben wurden.
}
{
  Die Theorie zu DE ist im Kapitel\\
  ~\fullref{sec:Description_DifferentialEvolutionAlgorithmus} beschrieben.
}


