\subsection{Weiterführende Arbeiten}

\minipagedOrBelowEachOther
{
    \subsubsection{Vergleich der Methoden}
    Es war mir nicht möglich, die gleichen Optimierungskriterien für alle Methoden zu definieren, 
    weil die Umsetzung der Tuning-Goals in MATLAB nicht offen gelegt wird und mir daher nicht bekannt ist.
    Um einen fairen Vergleich der Methoden zu ermöglichen, wäre es jedoch wichtig, alle Methoden
    mit den gleichen Optimierungskriterien zu testen.
    Für eine weiterführende Arbeit könnte dies untersucht und gegebenenfalls umgesetzt werden.
}
{
    \subsubsection{Frequenzabhängige Optimierung}
    In dieser Arbeit wurden die Methoden \gls{GA} und \gls{DE} hauptsächlich im Zeitbereich angewendet.
    Die Berechnung der Verstärkungs- und Phasenreserve wurden ebenfalls im Zeitbereich approximativ durchgeführt.
    MATLAB eignet sich für Frequenzabhängige Analysen und Simulationen besser als die eigene Implementierung.
    Deshalb könnte in einer weiterführenden Arbeit vertiefter auf die unterschiedlichen,
    frequenzabhängigen Optimierungskriterien eingegangen werden.
}

\minipagedOrBelowEachOther
{
    \subsubsection{Vertiefung der Stabilitätsanalyse}
    In dieser Arbeit wurden bestimmte Stabilitätsanalysen erst in einem späten Stadium durchgeführt.
    Für eine weiterführende Arbeit sollten diverse Stabilitätskriterien bereits im Entwurf
    der Tuning-Goals und der Fitnessfunktionen verstärkt berücksichtigt werden.
}
{
    \subsubsection{Weitere Prozesse testen}
    In dieser Arbeit wurden die Methoden nur an zwei Prozessen getestet.
    Für eine weiterführende Arbeit, könnten die Methoden an weiteren Prozessen getestet werden,
    die auch erschwerte Bedingungen aufweisen um die Leistungsfähigkeit 
    der Methoden unter härteren Bedingungen zu untersuchen.
}


