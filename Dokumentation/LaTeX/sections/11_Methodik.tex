% =======================================================================================
% Methodik
% =======================================================================================
\subsection{Methodik}
\label{sec:Methodik}
\minipagedOrBelowEachOther
{
  Durch eine Literaturrecherche sollen automatisierbare Optimierungsverfahren ermittelt werden.
  Anschliessend werden die gefundenen Methoden verglichen.
  MATLAB dient in diesem Teil der Arbeit dazu, die Methoden grob zu testen und zu erproben.
  Für den Vergleich und die daraus folgende Auswahl der zu vertiefenden Methoden werden
  verschiedene Aspekte betrachtet.
}{
  Es muss sich die Frage gestellt werden, wie ein guter Regler definiert ist.
  Jede Anwendung hat unterschiedliche Anforderungen an einen guten Regler und demnach kann nicht 
  eindeutig bestimmt werden, wie das Verhalten eines guten Reglers aussieht. 
  Es können jedoch einige Eigenschaften des gesamten \gls{System}s gemessen werden, 
  um eine Aussage über die Qualität des Reglers zu treffen.
}

% ---------------------------------------------------------------------------------------
% Aspekte zur bewertung eines Reglers
% ---------------------------------------------------------------------------------------
\horizontalLine
\subsubsection{Aspekte zur Bewertung eines Reglers}
\minipagedOrBelowEachOther
{
  \subparagraph{Stabilität}
  \noindent
  \\ 
  Ein stabiler Regelkreis ist im Normalfall eine Grundvoraussetzung an ein System. 
  In sehr spezifischen Anwendungen kann jedoch auch ein instabiles Verhalten explizit gesucht sein.
  Im weiteren Verlauf dieser Arbeit wird jedoch nur auf stabile Regelkreise eingegangen.
  \begin{itemize}
    \item Der Regler muss in der Lage sein das System zu stabilisieren
    \item Der Regler muss in der Lage sein, ein System stabil zu halten
  \end{itemize}
}
{
  \subparagraph{Zeitliches Verhalten}
  \noindent
  \\ 
  Das zeitliche Verhalten des Systems beschreibt, 
  wie der Regler auf eine plötzliche Änderung des Sollwerts reagiert.
  Abhängig von der jeweiligen Anwendung kann gefordert sein, 
  dass der Regler sehr schnell oder eher langsam auf Sollwertänderungen anspricht.
  Schnelle Regler haben jedoch häufig den Nachteil, 
  dass sie weniger robust gegenüber Störungen und Modellabweichungen sind und stärker zu Schwingungen neigen.
  \begin{itemize}
    \item Wie schnell lässt sich die Regelstrecke auf den gewünschten Sollwert bringen?
    \item Wie verhält sich der Regler dabei? (Überschwingen?, Einschwingzeit?, etc.)
  \end{itemize}
}


\minipagedOrBelowEachOther
{
  \subparagraph{Stationäres Verhalten}
  \noindent
  \\ 
  Beim stationären Verhalten wird untersucht, wie sich das System verhält, 
  nachdem eine ausreichend lange Zeit vergangen ist.
  Dies ist entscheidend, um festzustellen, ob das System den Sollwert tatsächlich erreicht 
  oder ob dauerhaft eine bestimmte Abweichung bestehen bleibt.
  \begin{itemize}
    \item Wie genau wird der Sollwert im stationären Zustand erreicht/gehalten, wie klein wird der stationäre Fehler (Abweichung zum Sollwert)?
  \end{itemize}
}
{
  \subparagraph{Robustheit}
  \noindent
  \\ 
  Die Robustheit beschreibt die Fähigkeit des Reglers, 
  auch bei Störungen, Parameteränderungen oder Modellunsicherheiten ein stabiles und 
  gewünschtes Systemverhalten sicherzustellen.
  Ein robuster Regler gewährleistet dabei, 
  dass Leistungsanforderungen und Toleranzen trotz dieser Einflüsse eingehalten werden.
  \begin{itemize}
    \item Wie gut kann der Regler mit Störungen umgehen?
    \item Wie gut kann der Regler mit Modellabweichungen umgehen?
    \item Wie gut kann der Regler mit Änderungen eines realen Prozesses umgehen?\\
    Z.B. durch Verschleiss
    \item Wie gut kann der Regler mit den realen Nichtlinearitäten eines Prozesses umgehen?
    \item Betrachtung der Phasen- und Amplitudengangreserve als Indikator für die Robustheit
  \end{itemize}
}
\minipagedOrBelowEachOther
{
\subparagraph{Komplexität des Reglers}
\noindent
\\
Wie viele Parameter des Reglers müssen durch die Methode optimiert werden. 
Die Anzahl dieser Parameter, definiert die Dimension des Suchraums der Optimierung.

}{


}
% ---------------------------------------------------------------------------------------
% Methodenspezifische Aspekte
% ---------------------------------------------------------------------------------------
\horizontalLine

\subsubsection{Methodenspezifische Aspekte}
Neben den oben genannten Aspekten, die zur Bewertung eines Reglers dienen können, 
werden für die Auswahl der zu vertiefenden Methoden noch weitere Aspekte betrachtet.
Diese beziehen sich auf die Methode selbst und nicht auf den Regler. 

\subparagraph{Voraussetzungen der Systeme}
\noindent
\\
Voraussetzungen die das System erfüllen muss, um von der jeweiligen Methode optimiert zu werden.
\begin{itemize}
  \item Kann ein stabiles und instabiles System optimiert werden?
  \item Kann ein \gls{SISO}- und \gls{MIMO}-System optimiert werden?
  \item Kann nur ein lineares System optimiert werden, oder auch ein \gls{nichtlineares System} oder sogar \gls{zeitvariantes System}?
\end{itemize}


\subparagraph{Die Schwierigkeit der Parametrierung}
\noindent
\\
\begin{itemize}
  \item Wie komplex ist die Parametrierung der Methode?
  \item Wie viele und welche Informationen muss der Anwender der Methode bereitstellen, 
  um eine Optimierung durchzuführen?
  \item Lässt sich die Informationsbereitstellung einfach für den Anwender gestalten?
  \item Bestimmung der \gls{Hyperparameter} der Methode und deren Einflussbereich.
\end{itemize}



\subparagraph{Automatisierungsgrad}
\noindent
\\
\begin{itemize}
  \item Wie automatisiert läuft die Optimierung ab?
  \item Welche Tools/Technologien werden benötigt um die Methode anzuwenden?
  \item Wie viel Eingriff des Anwenders ist notwendig um eine Optimierung durchzuführen?
\end{itemize}

% ---------------------------------------------------------------------------------------
% Zusammenfassend
% ---------------------------------------------------------------------------------------
\newpage
\subsubsection{Zusammenfassend}
Insgesamt gibt es viele verschiedene Aspekte und Kriterien, die bei der Auswahl und Bewertung von 
Methoden zur Reglersynthese berücksichtigt werden müssen. 
Die oben genannten Punkte bieten eine gute Grundlage, um die Eignung einer Methode für ein bestimmtes System zu 
beurteilen. Die Auswahl der zu vertiefenden Methoden erfolgt demnach zusammenfassend anhand folgender Punkte:

\newcommand*{\methodeCriterias}  {
  \begin{itemize}
  \item Mögliches Verhalten das durch den resultierenden Regler erreicht werden kann.
  \begin{itemize}
    \item Stabilität
    \item Zeitliches Verhalten
    \item Stationäres Verhalten
    \item Robustheit
  \end{itemize}

  \item Voraussetzungen der Systeme, an denen die jeweilige Methode angewendet werden kann
  \begin{itemize}
    \item stabile/instabile Systeme
    \item SISO/MIMO Fähigkeit
    \item linearese/nichtlineare Systeme \& zeitinvariante/zeitvariante Systeme
    \item Regler Strukturen, Anzahl der zu optimierenden Parameter
  \end{itemize}

  \item Die Schwierigkeit der Parametrierung
  \begin{itemize}
    \item Benutzerfreundlichkeit der Parametrisierung
    \item Benötigtes Vorwissen des Anwenders bezüglich Regelungstechnik
  \end{itemize}

  \item Automatisierungsgrad
  \begin{itemize}
    \item Tools/Technologien
    \item Eingriffe des Anwenders
  \end{itemize}
\end{itemize}}
\methodeCriterias

Einige der besten Methoden, werden an verschiedenen Prozessen getestet und bewertet.






