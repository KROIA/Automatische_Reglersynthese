\newpage
\subsection{Ergebnis}
Wie im Kapitel~\fullref{sec:Methodik} bereits erwähnt, wird für die Ermittlung der zu vertiefenden Methoden folgende Punkte betrachtet.
\methodeCriterias


\newpage
\subsubsection{Bewertungstabelle}
In der \ref{tab:Bewertungstabelle} ist die Bewertung der betrachteten Methoden in einer Tabelle dargestellt.
Es ist jedoch schwierig eine Quantisierung der einzelnen Kriterien vorzunehmen,
die Tabelle soll daher nur als grobe Orientierung dienen.


\begin{table}[H]
    \centering
        \includegraphics[width=0.98\linewidth]{images/Bewertung.pdf}
        \caption{Bewertungstabelle der betrachteten Methoden}
    \label{tab:Bewertungstabelle}
\end{table}
\newpage

\subsection{Kombination mehrerer Methoden}
\label{sec:KombinationMehrererMethoden}
\minipagedOrBelowEachOther
{
Es ist natürlich nicht vorgeschrieben nur eine Methode allein zu verwenden.
Das Kombinieren von mehreren Methoden kann sehr viele Vorteile bringen. 
}{}

\horizontalLine
\paragraph*{Beispiel 1:\\Ziegler-Nichols + genetischer Algorithmus für PID-Tuning}
\minipagedOrBelowEachOther
{
  Ein klassischer Ansatz zur schnellen Parametrierung eines PID-Reglers ist die Ziegler-Nichols Methode.
  Diese Methode liefert eine erste Abschätzung der Reglerparameter basierend auf dem offenen Regelkreisverhalten.
  Allerdings sind die mit Ziegler-Nichols bestimmten Parameter oft nicht optimal für spezifische Anforderungen.
}
{
  An dem Punkt kann ein \gls{GA} ansetzen, um die Reglerparameter weiter auf die gewünschten Anforderungen zu optimieren.
  Weil Ziegler-Nichols für \gls{MIMO} und instabile -\gls{System}e nicht geeignet ist, 
  unterliegt auch die Kombination dieser beiden Methoden dieser Einschränkung. 
}


\horizontalLine
\paragraph*{Beispiel 2:\\Ziegler-Nichols + diverse andere Optimierungsmethoden für PID-Tuning}
\minipagedOrBelowEachOther
{
Wie im vorherigen Beispiel beschrieben, kann Ziegler-Nichols eine gute erste Abschätzung der Reglerparameter liefern.
Für die weitere Optimierung der Regelparameter können aber auch andere Optimierungsmethoden verwendet werden:
}
{}
\begin{itemize}
    \item Differential Evolution \cite{storn1997differentialEvolution}
    \item Gradient Descent
    \item Gradient Descent with Momentum
\end{itemize}

\horizontalLine
\paragraph*{Beispiel 3:\\Evolutionärer Algorithmus + MATLAB \textit{systune} für komplexe Reglerstrukturen}
\minipagedOrBelowEachOther
{
Da für die Optimierung mit MATLAB \textit{systune} \cite{matlabSystune} ein initiales Modell mit Reglerstruktur benötigt wird,
kann eine Schwierigkeit darin bestehen, eine geeignete Reglerstruktur zu finden.
Ein evolutionärer Algorithmus wie \gls{NEAT} könnte verwendet werden, 
um verschiedene Reglerstrukturen zu generieren, welche anschliessend
mit MATLAB \textit{systune} optimiert werden.
}
{
Die Optimierungsergebnisse von \textit{systune} können in die Bewertungsfunktion des evolutionären Algorithmus einfliessen, 
um die besten Reglerstrukturen zu identifizieren.
}


\horizontalLine
\subsection{Auswahl der zu vertiefenden Methoden}
\subsubsection{MATLAB \textit{systune}}
%\minipagedOrBelowEachOther
{
  Aus allen betrachteten Methoden sticht~\fullref{sec:rechercheSystune} heraus, es handelt sich dabei nicht um einen spezifischen 
  Algorithmus für die \gls{Reglersynthese} sondern um ein Tool aus der MATLAB Control System Toolbox \cite{matlabControlSystemToolbox}.
  Mit diesem Tool können verschiedene Reglerstrukturen und Systemmodelle optimiert werden.
  Die Flexibilität und die einfache Automatisierung machen dieses Tool zu einer guten Wahl für die weitere Vertiefung.
}
{

}
\minipagedOrBelowEachOther
{
  \paragraph*{Voraussetzungen der Systeme}
  \begin{itemize}
    \item Das System muss als $genss$-Modell \cite{matlabGenss} modelliert sein
    \item Für die Optimierung können beliebig viele Parameter optimiert werden
    \item Die zu optimierenden Parameter können überall im Regelkreis liegen
    \item Das System sollte in der Lage sein, die definierten Optimierungsziele zu erreichen
    \item Stabile als auch instabile Systeme können optimiert werden
    \item SISO- als auch MIMO-Systeme können optimiert werden
    \item Nur lineare- und Zeitinvariante Systeme können optimiert werden
  \end{itemize}
}
{
\paragraph*{Ermöglichtes Reglerverhalten}
MATLAB \textit{systune} \cite{matlabSystune} ermöglicht die Definition von verschiedenen Optimierungszielen \cite{matlabTuningGoals},
welche spezifisch auf das gewünschte Reglerverhalten abgestimmt werden können.


\paragraph*{Die Schwierigkeit der Parametrisierung}
Die Parametrisierung erfolgt über Optimierungsziele \cite{matlabTuningGoals}.
Einerseits können beliebig viele Optimierungsziele definiert werden, andererseits können diese auch noch in harte und weiche Ziele
unterteilt werden.
Dadurch kann die Optimierung sehr flexibel gestaltet werden.
Jedes Optimierungsziel beinhaltet selbst noch weitere Parameter zur Konfiguration.
Das macht die Parametrisierung etwas komplexer, jedoch auch sehr mächtig.
}

%\paragraph{Ermöglichtes Reglerverhalten}
%\minipagedOrBelowEachOther
%{
%MATLAB systune \cite{matlabSystune} ermöglicht die Definition von verschiedenen Optimierungszielen \cite{matlabTuningGoals},
%welche sehr spezifisch auf das gewünschte Reglerverhalten abgestimmt werden können.
%}
%{}

%\paragraph{Voraussetzungen der Systeme}
%\begin{itemize}
%  \item Das \gls{System} muss als $genss$ \cite{matlabGenss} modelliert sein
%  \item Für die Optimierung können beliebig viele Parameter optimiert werden
%  \item Die zu optimierenden Parameter können überall im Regelkreis liegen
%  \item Das \gls{System} sollte in der Lage sein, die definierten Optimierungsziele zu erreichen
%  \item Stabile als auch instabile \gls{System}e können optimiert werden
%  \item \gls{SISO}- als auch \gls{MIMO}-\gls{System}e können optimiert werden
%  \item Nur lineare \gls{System}e können optimiert werden
%\end{itemize}

%\paragraph{Die Schwierigkeit der Parametrierung}
%\minipagedOrBelowEachOther
%{
%  Die Parametrierung erfolgt über Optimierungsziele \cite{matlabTuningGoals}.
%  Einerseits können beliebig viele Optimierungsziele definiert werden, andererseits können diese auch noch in harte und weiche Ziele
%  unterteilt werden. 
%}
%{
%  Dadurch kann die Optimierung sehr flexibel gestaltet werden.
%  Jedes Optimierungsziel beinhaltet selbst noch weitere Parameter zur Konfiguration.
%  Das macht die Parametrierung etwas komplexer, jedoch auch sehr mächtig.
%}

\newpage
\paragraph*{Automatisierungsgrad}
Um die Optimierung durchzuführen,
muss der Anwender einige Informationen bereitstellen.
Dazu gehören:
\begin{itemize}
  \item Definition des $genss$-Modells \cite{matlabGenss}, welches bereits den zu optimierenden Regler enthält
  \item Definition der Optimierungsziele \cite{matlabTuningGoals} und deren Parametrisierung
  \item Bestimmung welche Optimierungsziele hart und welche weich sind
  \item (Optional) Definition von Startwerten für die zu optimierenden Parameter
\end{itemize}
\minipagedOrBelowEachOther
{
Sobald diese Informationen bereitgestellt sind,
kann der Optimierungsprozess gestartet werden.
Das Tool übernimmt dann die komplette Optimierung und liefert ein 
$genss$-Modell \cite{matlabGenss} mit den optimierten Parametern zurück.
Ausserdem wird eine Auswertung der Optimierung für jedes Optimierungsziel bereitgestellt.
}
{

}




\horizontalLine
\subsubsection{Genetischer Algorithmus für PID-Tuning}
\label{sec:GAforPIDTuning}

\paragraph*{Ermöglichtes Reglerverhalten}
\minipagedOrBelowEachOther
{
Benutzerdefinierte Bewertungsfunktionen ermöglichen die Optimierung des PID-Reglers auf sehr spezifische Anforderungen.
}
{
Die Theorie zum GA ist im Kapitel\\
~\fullref{sec:Description_GenetischerAlgorithmus} beschrieben.
}

\paragraph*{Voraussetzungen der Systeme}
\begin{itemize}
  \item Modell benötigt für den \gls{GA}
  %\item System muss stabil sein
  %\item System muss \gls{minimalphasiges System} sein
  \item System kann \gls{MIMO} sein
  \item Reglerstruktur ist beliebig % auf PID-Regler beschränkt
\end{itemize}


\minipagedOrBelowEachOther
{
  \paragraph*{Die Schwierigkeit der Parametrisierung}
  Der Anwender muss eine Bewertungsfunktion definieren welche aus beliebigen Teilfunktionen bestehen kann.
}
{
  \paragraph*{Automatisierungsgrad}
  Mit bestehender Bewertungsfunktion und vordefiniertem Modell inklusive Reglerstruktur,
  kann die Optimierung gestartet werden.
  Die Optimierung ist ein iterativer Prozess welcher solange wiederholt wird, 
  bis das Ergebnis zufriedenstellend ist.
}





\horizontalLine
\subsubsection{Differential Evolution für PID-Tuning}
\minipagedOrBelowEachOther
{
  \gls{DE} hat ziemlich ähnliche Eigenschaften wie der \gls{GA}.
  Der Unterschied besteht hauptsächlich darin, wie die Selektion, Kreuzung und Mutation durchgeführt werden.\\


  Weil die beiden Algorithmen so ähnlich sind, wird hier nicht weiter auf die Eigenschaften eingegangen,
  da sie bereits im Abschnitt\\
  ~\fullref{sec:GAforPIDTuning} beschrieben wurden.
}
{
  Die Theorie zu DE ist im Kapitel\\
  ~\fullref{sec:Description_DifferentialEvolutionAlgorithmus} beschrieben.
}


