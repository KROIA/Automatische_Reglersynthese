\section{Einleitung}

\minipagedOrBelowEachOther
{
    \subsection{Problemstellung}
    Für die \gls{Reglersynthese} gibt es
    eine Vielzahl von Methoden. Diese unterscheiden sich zum
    einen durch ihre mathematische Herangehensweise und zum
    anderen durch das benötigte \gls{Modellwissen} über den \gls{Prozess}.
    Ausserdem gibt es Unterschiede wie automatisiert die Reglersynthese
    abläuft und wie die Methode parametrisiert wird.
}{
    \subsection{Aufgabenstellung}
    Der Fokus dieser Arbeit soll auf der Parametrisierung der
    Methode liegen, also darauf, wie und in welcher Form der Methode 
    kommuniziert wird, welches Verhalten der geschlossene
    Regelkreis haben soll. Diese Arbeit soll aufzeigen, welche Methoden 
    zur automatischen Reglersynthese existieren und für
    welche Anwendungsfälle diese geeignet sind.

    Die besten Methoden sollen implementiert und an verschiedenen Prozessen getestet werden.
    Zuletzt soll eine interaktive GUI erstellt werden mit der die unterschiedlichen Methoden
    auf einem Prozess angewendet werden können.
    Das Ziel der Arbeit ist es eine Aussage darüber zu treffen, welche
    Methoden, je nach Prozess und gewünschtem Verhalten, 
    zur automatischen Reglersynthese geeignet sind, wie diese funktionieren
    und welche Vor- und Nachteile diese haben.
    Zur besseren Vermittlung des gewonnenen Wissens sollen die Methoden
    in einer GUI auf einem Beispielprozess implementiert werden.
}

\horizontalLine
\subsection{Aufbau des Dokuments}
    Das Dokument besteht aus 4 Teilen.
    \begin{itemize}
        \item Recherche zu verschiedenen Methoden für die automatische Reglersynthese
        \item Implementierung und Testen der ausgewählten Methoden an zwei Prozessen
        \begin{itemize}
            \item DC-Motor
            \item Motor-mit-Schwungmasse
        \end{itemize}
        \item Schlussteil mit Fazit
        \item Anhang mit diversen theoretischen Erklärungen zu verschiedenen Themen die in der Arbeit vorkommen
    \end{itemize}
    
\minipagedOrBelowEachOther
{
    \subsubsection{Abkürzungen}
    \printglossary[type=acronyms, style=longheader]
}
{
    \subsubsection{Randnotiz zu den getesteten Systemen}
    In dieser Arbeit werden zwei verschiedene \gls{Prozess}e mit einem PID-Regler geregelt.\\
    Mir ist durchaus bewusst, dass ein PID-Regler nicht zwingend die beste Wahl ist.
    Es geht in dieser Arbeit jedoch nicht darum, einen optimalen Regler für die Prozesse zu entwerfen,
    sondern die verschiedenen Methoden der automatischen Reglersynthese anzuwenden und zu vergleichen.
    Die Wahl eines nicht für alle Prozesse optimalen Reglers, kann als Challenge gesehen werden,
    ein möglichst gutes Ergebnis mit den gegebenen Mitteln zu erzielen. 
}



\horizontalLine
\subsection{Verwendete Tools}
In dieser Arbeit wurden folgende Tools verwendet:
\begin{itemize}
    \item \textbf{LaTeX} ~\cite{latexProject} für die Dokumentation
    \item \textbf{MATLAB/Simulink} ~\cite{matlabAPP} für die Simulation der Systeme und die Verwendung der \textit{systune} Methode
    \item \textbf{Visual Studio Community 2022} ~\cite{visualStudioAPP} \\
        für die C++ Implementierung der Optimierungsalgorithmen \gls{GA} und \gls{DE} und die GUI.\\
        \begin{itemize}
            \item Verwendeter Compiler: \textbf{MSVC}
            \item Build System: \textbf{CMake} ~\cite{cmakeAPP}
            \item GUI Framework: \textbf{Qt 5.15.2} ~\cite{qtFrameworkLibrary}
            \item SFML: \textbf{SFML 2.6.1} ~\cite{sfmlCPPLibrary} für die Visualisierung der Simulation
            \item ImGui: \textbf{ImGui 1.91.5} ~\cite{imguiCPPLibrary} als Grundlage für ImPlot
            \item ImPlot: \textbf{ImPlot 0.16} ~\cite{implotCPPLibrary} für die Plot Elemente in der GUI
        \end{itemize}
    \item \textbf{Visual Studio Code} ~\cite{visualStudioCodeAPP} für die Bearbeitung von LaTeX Dateien
    \item \textbf{Processing} ~\cite{processingAPP} für die Erprobung und Visualisierung vom Differential Evolution Algorithmus in einer einfachen Umgebung
    \item \textbf{Git} ~\cite{gitAPP} für die Versionskontrolle des Codes und der Dokumentation
    \item \textbf{ChatGPT, ClaudeAI und GitHub Copilot}  ~\cite{llmChatGPT} ~\cite{llmClaude} ~\cite{llmGitHubCopilot} 
    für die Unterstützung bei der Code Ergänzung und Hilfe
    bei der Erstellung von MATLAB-Skripte und LaTeX Makros.
\end{itemize}

%\horizontalLine
%\newpage
\subsection{Quellcode und Dokumentation}
Der Quellcode und die Dokumentation dieser Arbeit sind im GitHub Repository \cite{baRepository} verfügbar.
Dort sind auch alle MATLAB-Skripte und Simulink-Modelle zu finden, die in dieser Arbeit verwendet wurden.

