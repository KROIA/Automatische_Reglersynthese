\formulaWithDerivationDeclaration{formulaDCMotorCModel}
{   % Dargestellte Formel
    \mathFunction{y}{t}= \frac{1}{T} \cdot \int_{}^{}(K_{\mathbf{1}} \cdot \mathFunction{u}{t} - \mathFunction{y}{t} \cdot (1+K_{\mathbf{2}} \cdot K_{\mathbf{3}} \cdot \mathFunction{l}{t}))dt
}
{   % Herleitung der Formel
    Als Ausgangslage dient das Modell aus dem Skript \cite{regT12Script} Kapitel 3.3, Gleichung (34):
    \begin{equation*}
        T \cdot \mathFunctionDot{y}{t} + \mathFunction{y}{t} = K_{\mathbf{1}} \cdot \mathFunction{u}{t} - K_{\mathbf{2}} \cdot \mathFunction{l}{t}
    \end{equation*}

    \begin{table}[H]
        \center
        \begin{tabular}{@{}rl@{}}
                \toprule
                \textbf{Parameter}              & \textbf{Beschreibung} \\
                \midrule
                $K_{\mathbf{1}}$                & Substitution mehrerer Motor spezifischer Konstanten \\[0.5ex]
                $K_{\mathbf{2}}$                & Substitution mehrerer Motor spezifischer Konstanten \\[0.5ex]
                $t$                 & Zeit \\[0.5ex]
                $T$                 & Zeitkonstante des Motors \\[0.5ex]
                $\mathFunction{u}{t}$           & Motor Eingangsspannung \\[0.5ex]
                $\mathFunction{y}{t}$           & Motor Ausgangsdrehzahl als Spannung abgebildet \\[0.5ex]
                \bottomrule
        \end{tabular}
        \caption{Symbole für die Herleitung des DC-Motor Modells}
    \end{table}

    \begin{table}[h]
    \begin{tabular}{m{0.5\textwidth} m{0.5\textwidth}}
        %\centering
        \hline
        Als Modifikation wird die Last $l$ mit dem Ausgang $y$ multipliziert, 
        da das Lastmoment der Wirbelstrombremse abhängig von der Drehzahl des Motors ist.
        Dabei wird der Faktor $K_{\mathbf{3}}$ als Proportionalitätsfaktor eingeführt.
        $K_{\mathbf{3}}$ gibt an, wie stark die Last vom Ausgang abhängt.
        Somit ergibt sich die rechts dargestellte modifizierte Differential Gleichung.
        &
        \begin{equation*}
        T \cdot \mathFunctionDot{y}{t} + \mathFunction{y}{t} = K_{\mathbf{1}} \cdot \mathFunction{u}{t} - K_{\mathbf{2}} \cdot \mathFunction{l}{t} \cdot K_{\mathbf{3}} \cdot \mathFunction{y}{t}
        \end{equation*} \\
        \hline

        Umstellen nach $\mathFunctionDot{y}{t}$
        &
        \begin{equation*}
        T \cdot \mathFunctionDot{y}{t} = K_{\mathbf{1}} \cdot \mathFunction{u}{t} - K_{\mathbf{2}} \cdot \mathFunction{l}{t} \cdot K_{\mathbf{3}} \cdot \mathFunction{y}{t} - \mathFunction{y}{t}
        \end{equation*}
        \begin{equation*}
        \mathFunctionDot{y}{t} = \frac{1}{T} (\cdot K_{\mathbf{1}} \cdot \mathFunction{u}{t} - K_{\mathbf{2}} \cdot \mathFunction{l}{t} \cdot K_{\mathbf{3}} \cdot \mathFunction{y}{t} - \mathFunction{y}{t})
        \end{equation*} \\
        \hline

        Durch die Integration erhält man $\mathFunction{y}{t}$.
        Diese Gleichung lässt sich in code implementieren.
        &
        \begin{equation*}
            \mathFunction{y}{t}= \frac{1}{T} \cdot \int_{}^{}(K_{\mathbf{1}} \cdot \mathFunction{u}{t} - \mathFunction{y}{t} \cdot (1+K_{\mathbf{2}} \cdot K_{\mathbf{3}} \cdot \mathFunction{l}{t}))dt
        \end{equation*}\\
        \hline
        
    \end{tabular}
    \end{table}

    Da die Implementierung diskretisiert erfolgt, muss die Formel noch diskretisiert werden.
    Dies wird in diesem Teil nicht weiter beschrieben, da die Integration im code auf unterschiedliche Arten
    implementiert worden ist. (Backward Euler, Forward Euler und Bilinear Transformation)
    Die folgende Formel dient deshalb nur zur Verdeutlichung des Prinzips der Implementierung im code.
    Für jede Integrationsmethode muss die Transformation entsprechend durchgeführt werden.
    Dies kann im C++ Code eingesehen werden. \cite{baRepository}
    \begin{equation*}
        \mathFunction{y}{t}= \frac{1}{T} \cdot \int_{}^{}(K_{\mathbf{1}} \cdot \mathFunction{u}{t} - \mathFunction{y}{t} \cdot (1+K_{\mathbf{2}} \cdot K_{\mathbf{3}} \cdot \mathFunction{l}{t}))dt
    \end{equation*}

    \newpage
}