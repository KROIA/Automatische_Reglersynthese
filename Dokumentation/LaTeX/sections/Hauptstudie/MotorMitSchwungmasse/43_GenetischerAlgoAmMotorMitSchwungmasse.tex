\section{GA am Motor mit Schwungmasse}

% Commands already define in GA am DC-Motor
%\newcommand{\individuumI}{\overrightarrow{I_{\mathbf{i}}}}
%\newcommand{\funcGOfk}[1]{\mathFunction{g_\mathbf{#1}}{\individuumI,k}}
%\newcommand{\funcGOf}[1]{\mathFunction{g_\mathbf{#1}}{\individuumI}}

\minipagedOrBelowEachOther
{
    Weil der \gls{GA} kein MATLAB voraussetzt, ist diese Methode auch ohne MATLAB implementiert worden.
	Das hat den Vorteil, dass ein System auch ohne MATLAB Lizenz optimiert werden kann.
}
{
	Die Funktionsweise des GA ist im Kapitel\\
	\fullref{sec:Description_GenetischerAlgorithmus} beschrieben.
	
}

\subsection{System Definieren}
\begin{figure}[H]
    \centering
    \begin{tikzpicture}[auto, node distance=2cm,>=latex', scale=1]
    \newcommand{\labeloffset}{0.2}
    % Draw grid
    %\draw[step=1cm,gray,very thin] (0,0) grid (10,10);
    
    % Draw blocks
    \sumNode{sum}{1,0};

    %\saturationBlockNode{saturation1}{-1, 0}{-10}{10};
    \saturationBlockNode{saturation3}{4.5, 2}{-10}{10};
    \blockNode{controller}{2.5,0}{$PID$};
    \saturationBlockNode{saturation2}{4.5,0}{-10}{10};
    \gainNode[east]{u1gain}{$(saturation2.east) + (1,0)$}{300};
    \gainNode[east]{u2gain}{$(saturation3.east) + (1,0)$}{300};
    \blockNode{plant}{$(u1gain.east) + (2,-0.25)$}{M=$|$w$|$=M};

    % Draw connections
    %\connect{(-2.5,0)             (saturation1.west)          }{->};
    \connect[label={-3.6,\labeloffset}{$\mathbf{d}$}              ]{(0,2)             (saturation3.west)          }{->};
    \connect[label={-0.5,\labeloffset}{$\mathbf{r}$}              ]{(0,0)             (sum.west)          }{->};
    \connect[label={-0.5,\labeloffset}{$\mathbf{e}$}   ]{(sum.east)        (controller.west)   }{->};
    \connect{(controller.east) (saturation2.west)   }{->};
    
    \connect{(saturation2.east) (u1gain.west)        }{->};
    \connect[label={-0.25, \labeloffset}{$\mathbf{u_{1}}$}   ]{(u1gain.east) ($(plant.west) + (0, 0.25)$)        }{->};
    \connectionNode{output}{$(plant.east) + (0.5,-0.25)$};
    \connect[label={0.2,0}{$\mathbf{y_{2}}$}               ]{\vectorAdd{plant.east}{0,-0.25} (output.center) \vectorAdd{output}{0.5,0}}{->};
    \connect[label={0.2,0}{$\mathbf{y_{1}}$}               ]{\vectorAdd{plant.east}{0,0.25} \vectorAdd{plant.east}{1,0.25}}{->};
    \connect{(saturation3.east) (u2gain.west)}{->};
    \connect[label={-0.25, -\labeloffset}{$\mathbf{u_{2}}$}]{(u2gain.east) ($(plant.west) + (-0.6, 2.25)$) ($(plant.west) + (-0.6, -0.25)$) ($(plant.west) + (0, -0.25)$)}{->};

    \feedback[sign={-0.3,-0.1}              ]{output.center}{-1}{sum.south}{->};
\end{tikzpicture}
    \caption{Regelkreis Motor mit Schwungmasse}
    \label{fig:MotorMitSchwungmasseTikzDiagram_genetic}
\end{figure}
In der C++ Anwendung werden die Teilsysteme objektorientiert implementiert und anschliessend 
als ein gesamtheitliches Objekt zusammengeführt und verbunden.
Der Kernteil der Implementierungslogik spielt sich dabei in den Update-Funktionen der jeweiligen Teilsysteme ab.
Aufgrund des Codes-Umfangs werden deshalb in den nachfolgenden Unterkapiteln nur die Update-Funktionen gezeigt.


\horizontalLine
\subsubsection{Motor definieren}
Die Implementierung im C++ Code ist auf der Grundlage der Modellidentifikation des Systems im Kapitel
\fullref{sec:MotorMitSchwungmasseModellidentifikation} durchgeführt worden.
Die dazugehörige C++ Implementierung ist unter
\fullref{lst:MotorMitSchwungmasseDefinitionCPP} zu finden.
%\minipagedOrBelowEachOther
%{
%	
%}{
%	
%}



\horizontalLine
\subsubsection{PID-Regler definieren}
Der PID-Regler wurde bereits im Kapitel \fullref{sec:ErweiterterPIDRegler} beschrieben und 
auch die C++ Implementierung wird dort aufgeführt. Deshalb wird hier nicht weiter darauf eingegangen.




\horizontalLine
\subsubsection{Alle Teilsysteme verbinden}
\begin{cppcode}{Update-Funktion des Gesamtsystems im Code}{lst:MotorMitSchwungmasseRegelkreisCPP}
// Update Funktion des Gesamtsystems im C++ Code. class TestSystem
void update(double deltaTime) override
{
	double y = m_motorWithMass.getOutputs()[1];	// y(t)
    double measurementNoise = AutoTuner::Solver::getRandomDouble(-1, 1) * 0.1;
    m_errorValue = m_referenceValue - y + measurementNoise;				// e(t) = r(t) - y(t)

    m_pidController.setInput(m_errorValue);

	m_pidController.update(deltaTime);
    m_pidOutputValue = m_pidController.getOutput();
    m_motorWithMass.setInputSignal(0, m_pidOutputValue);
    m_motorWithMass.setInputs(1, m_disturbanceValue);
    m_motorWithMass.update(deltaTime);
}
\end{cppcode}
\minipagedOrBelowEachOther{
	Wie im Code ersichtlich ist, wird dem System ein zusätzliches Messrauschen hinzugefügt.
	Dieses begünstigt die Eigenschaft der Rauschunterdrückung des Reglers durch die Bewertungsteilfunktion
	$\mathFunction{g_\mathbf{1}}{\individuumI}$ (Stellwertänderungsrate).
}{

}


\newpage
\subsection{Optimierungsziele definieren}
\begin{figure}[H]
    \centering
    \includegraphics[width=\linewidth]{images/MotorMitSchwungmasse/Stimuli.pdf}
    \caption{Anregungssignale für die Systemsimulation}
    \label{fig:MotorMitSchwungmasseStimuli}
\end{figure}
Um die Bewertungsfunktion zu definieren wird eine zeitliche Simulation des Systems benötigt.
Dazu wird ein vordefiniertes Anregungssignal verwendet.
Das Anregungssignal besteht aus $r$ (Ziel-Winkel) und $d$ (Störungsmoment).


\newpage
\begin{table}[H]
\centering
\begin{tabular}{@{}clrc@{}}
		\toprule
		\textbf{Symbol} & \textbf{Beschreibung} & \textbf{Wert} & \textbf{Einheit} \\
		\midrule
		$Ts$                & Abtastzeit &0.01& s\\[0.5ex]
		$t_\mathbf{start}$  & Startzeit &0& s\\[0.5ex]
		$t_\mathbf{end}$    & Endzeit &50& s\\[0.5ex]
		$u_\mathbf{satHigh}$   & Obere Sättigungsgrenze der Stellgrösse &10& \\[0.5ex]
		$u_\mathbf{satLow}$   & Untere Sättigungsgrenze der Stellgrösse &-10& \\[0.5ex]
		%$N$                 & Anzahl Individuen in der Population &30\\[0.5ex]
		$K$                 & Anzahl Zeitschritte in der Simulation & 5000\\[0.5ex]
		$D$				  & Anzahl der Optimierungs-Parameter & 4 & \\[0.5ex]
		$\individuumI$    & \gls{Individuum} $\mathbf{i}$ in der \gls{Population} & $\individuumI \in \mathbb{R}^D$ \\[0.5ex]
		%$\mathFunction{f}{\individuumI}$ & Maximierungs-Fitness-Funktion.& $f(\individuumI) \in \mathbb{R}^{+}$ \\[0.5ex]
		$\mathFunction{g}{\individuumI}$ & Fehler-Minimierungs-Funktion. & $\mathFunction{g}{\individuumI} \in \mathbb{R}$ \\[0.5ex]
		$a_\mathbf{o}$      & Gewichtungsfaktor für das Optimierungsziel $\mathbf{o}$ & $a_\mathbf{o} \in \mathbb{R}^{+_0}$ \\[0.5ex]
		$O$                 & Anzahl Optimierungsziele & 5 \\[0.5ex]
		$\funcGOfk{o}$ & Optimierungsfunktion für das Ziel $\mathbf{o}$ zum Zeitpunkt $k$ & $\funcGOfk{o} \in \mathbb{R}$ \\[0.5ex]
		\bottomrule
\end{tabular}
\caption{Symbole für die Simulation am Motor mit Schwungmasse}
\end{table}


Für die Abtastzeit $Ts$ wurde ein realistischer Wert von 10ms gewählt.\\


\minipagedOrBelowEachOther
{
	Die Anzahl Zeitschritte $K$ wird aus der Abtastzeit $T_\mathbf{s}$ und 
	dem Simulationszeitraum $[t_\mathbf{start}, t_\mathbf{end}]$ berechnet.  
}{
	\centering
    \begin{equation}
		K = \frac{t_\mathbf{end}-t_\mathbf{start}}{T_\mathbf{s}}
	\end{equation}
}

\horizontalLine

\minipagedOrBelowEachOther
{
	Der \gls{GA} ist für \gls{Maximierungsproblem}e ausgelegt. 
	Für diese Anwendung wäre es aber sinnvoller die Optimierungsziele als \gls{Minimierungsproblem}e zu formulieren.
	Es gibt jedoch eine Möglichkeit zur Umrechnung von \gls{Minimierungsproblem}en in Maximierungsprobleme.
	Dies ist im Kapitel \fullref{sec:Description_GenetischerAlgorithmus_FitnessFunction_TransformationSteps} beschrieben.
	Dank dieser Umrechnung können die Optimierungsziele als Minimierungsproblem formuliert werden.

	Die Fehler-Minimierungs-Funktion $\mathFunction{g}{\individuumI}$
	wird aus mehreren Teilsummen zusammengesetzt. Die Teilsummen bewerten verschiedene Aspekte des Regelverhaltens.
	Allgemein lässt sich die Fehler-Minimierungs-Funktion $\mathFunction{g}{\individuumI}$ wie in der 
\ref{form:MotorMitSchwungmasseGeneticAlgorithmErrorMinimizationFunction} darstellen.
}{
	
    \begin{equation}
		\label{form:MotorMitSchwungmasseGeneticAlgorithmErrorMinimizationFunction}
		\mathFunction{g}{\individuumI} = 
		\frac{1}{K} \cdot \sum_{k=0}^{K-1}\bigg(
				\sum_{o=0}^{O-3}\Big( 
					a_\mathbf{o} \cdot \funcGOfk{o}
				\Big)
			\bigg) + a_\mathbf{3} \cdot \funcGOf{\mathbf{3}}
			  + a_\mathbf{4} \cdot \funcGOf{\mathbf{4}}
	\end{equation}
	
	\begin{itemize}
		\item $\frac{1}{K}$ dient zur Normierung der Fehlerbewertung auf die Anzahl der Zeitschritte.
		\item $\sum_{k=0}^{K-1}$ summiert die Fehlerbewertung über alle Zeitschritte der Simulation.
		\item $\sum_{o=0}^{O-3}\left( 
					a_\mathbf{o} \cdot \funcGOfk{o}
				\right) $ summiert die gewichteten Optimierungsziele $\funcGOfk{o}$ für den Zeitschritt $k$.
		\item $a_\mathbf{3} \cdot \funcGOf{\mathbf{3}}$ bewertet die Verstärkungsreserve des Systems.
		\item $a_\mathbf{4} \cdot \funcGOf{\mathbf{4}}$ bewertet die Phasenreserve des Systems.
	\end{itemize}

	$\funcGOf{\mathbf{3}}$ und $\funcGOf{\mathbf{4}}$ 
	werden nicht über die Zeitschritte während der Simulation summiert, 
	da sie nicht abhängig von der Laufzeit der Simulation sind, 
	sondern approximiert im Frequenzbereich berechnet werden.
}







%\begin{equation}
%	\mathFunction{g}{\individuumI} = 
%	\frac{1}{t_\mathbf{end}-t_\mathbf{start}}\sum_{k=0}^{K}(
%		a_\mathbf{1} \cdot \mathFunction{g_\mathbf{1}}{k} + 
%		a_\mathbf{2} \cdot \mathFunction{g_\mathbf{2}}{k}
%		)
%\end{equation}

%\horizontalLine



\newpage
Folgende Optimierungsziele wurden bereits im Kapitel 
~\fullref{sec:Description_GenetischerAlgorithmus_FitnessFunction_OptimizationGoals_SimpleMotor} 
im Zusammenhang mit dem DC-Motor beschrieben und werden hier deshalb nicht nochmals erklärt:
\begin{itemize}
    \item $\funcGOfk{\mathbf{0}}$: Absoluter Regelungsfehler aber ohne Sättigungsbedingung 
    \item $\funcGOfk{\mathbf{1}}$: Stellwertänderungsrate
    \item $\funcGOfk{\mathbf{2}}$: Positives Überschwingen unterdrücken
\end{itemize}

\horizontalLine
\subsubsection{$\funcGOf{\mathbf{3}}$: Verstärkungsreserve}
\minipagedOrBelowEachOther
{
	$\funcGOf{\mathbf{3}}$ bewertet die quadratische Abweichung der Verstärkungsreserve des Systems
    von dem gewünschten Wert von 2. 	      
}{
	Diese Bewertung wurde für den Test deaktiviert ($a_\mathbf{3} = 0$), 
    da die Optimierung dem System von sich aus eine weitaus bessere Verstärkungsreserve verpasst als gefordert und so
    wie die Bewertung formuliert ist, würde der \gls{Solver} versuchen, die Verstärkungsreserve wieder zu verschlechtern.
}
\vspace{0.1cm}
\horizontalLine
\subsubsection{$\funcGOf{\mathbf{4}}$: Phasenreserve}
\minipagedOrBelowEachOther
{
	$\funcGOf{\mathbf{4}}$ bewertet die quadratische Abweichung der Phasenreserve des Systems
    von dem gewünschten Wert von 90°.
	Die Bewertung wird erst ab Epoche 100 durchgeführt, damit der Algorithmus
	zuerst eine grobe Lösung finden kann, bevor die Stabilitätskriterien berücksichtigt werden.
	Die Frequenzanalyse wird mit dem nicht linearisierten System im Zeitbereich durchgeführt.
	Zu Beginn der Simulation ist das System noch instabil, was zu einer falsch berechneten Frequenzanalyse führt 
	und somit ist die Frequenzanalyse erst ab Epoche 100 sinnvoll.
}{
    Es wurde 90° gewählt, weil damit versucht wird, möglichst nah an 90° zu kommen.
    \textit{Systune} hat gezeigt, dass es möglich ist, nah an die 90° zu kommen, ohne das System zu träge zu machen.
}


\horizontalLine
\subsubsection{$\mathFunction{g}{\individuumI}$: Gesamte Fehlerbewertung}
\minipagedOrBelowEachOther
{
	Die Teilbewertungen werden jeweils mit einem \\
	Gewichtungsfaktor $a_\mathbf{o}$ skaliert. Diese müssen vom Anwender\\
	passend gewählt werden, um das gewünschte Verhalten zu erzielen.
	Es ist zu beachten, dass $\funcGOf{\mathbf{4}}$ nicht über 
	die Simulationszeit summiert wird, sondern nur einmal pro Individuum berechnet wird,
	weil sich der Frequenzgang im Verlauf einer Simulationsdurchführung nicht ändert.
}{
	\centering
    \begin{equation}
		\mathFunction{g}{\individuumI} = 
		\frac{1}{K} \cdot \sum_{k=0}^{K-1}\Big(
			a_\mathbf{0} \cdot \funcGOfk{\mathbf{0}} + 
			a_\mathbf{1} \cdot \funcGOfk{\mathbf{1}} +
			a_\mathbf{2} \cdot \funcGOfk{\mathbf{2}}
			\Big)
             +
            a_\mathbf{4} \cdot \funcGOf{\mathbf{4}}
	\end{equation}
}



\newpage
\subsection{Optimierung durchführen}
\begin{figure}[H]
	\center
    %\vspace{-1.6cm}
    \includegraphics[width=\linewidth]{images/MotorMitSchwungmasse/Genetic_SimulationSoftware.png}
    \caption{Software der Systemsimulation für die Optimierung}
    \label{fig:MotorMitSchwungmasseGenetic_SimulationSoftware}
\end{figure}

\minipagedOrBelowEachOther
{
	Wie beim Beispiel mit dem DC-Motor \ref{fig:SimpleMotorGenetic_SimulationSoftware} wird die Optimierung 10-mal durchgeführt
	und bestimmte Hyperparameter werden willkürlich gewählt.

}
{
	Mehr zur Bestimmung der Hyperparameter ist im Kapitel\\
	\fullref{sec:objektivesFazitIterativesTuning} beschrieben.
}

\begin{table}[H]
\centering
\vspace{-0.2cm}
\begin{tabular}{@{}clrc@{}}
		\toprule
		\textbf{Parameter} & \textbf{Beschreibung} & \textbf{Wert} \\
		\midrule
		$D$                 & Anzahl der einzelnen Optimierungsdurchläufe &10\\[0.5ex]
		$N$                 & Anzahl Individuen in der Population &30\\[0.5ex]
		$G$                 & Anzahl Generationen &1000\\[0.5ex]
		$C$                 & Mutations-Wahrscheinlichkeit &0.05\\[0.5ex]
		$\mathFunction{M}{k}$ & Mutationsstärke &$1 \cdot 0.999^{k}$\\[0.5ex]
		Anti-Windup       & Verwendete Anti-Windup Methode & Clamping \\[0.5ex]
		\hline
		\textbf{Optimierungsziele} &  &  & \\[0.5ex]
		$a_\mathbf{0}$     & Absoluter Regelungsfehler &2.8\\[0.5ex]
		$a_\mathbf{1}$     & Stellwertänderungsrate &0.1\\[0.5ex]
		$a_\mathbf{2}$     & Positives Überschwingen &111\\[0.5ex]
        $a_\mathbf{3}$     & Verstärkungsreserve (6dB) &0\\[0.5ex]
        $a_\mathbf{4}$     & Phasenreserve (90°) &99 Ab Epoche 100\\[0.5ex]
		\hline
		\textbf{Konstante Systemparameter} &  &  & \\[0.5ex]
		$u_\mathbf{satHigh}$         & PID Ausgangs-Sättigung oberer Wert &10\\
		$u_\mathbf{satLow}$          & PID Ausgangs-Sättigung unterer Wert &-10\\
		$I_{\mathbf{SatHigh}}$	     & PID-Integrator-Sättigung oberer Wert &100000000\\
		$I_{\mathbf{SatLow}}$	     & PID-Integrator-Sättigung unterer Wert &-100000000\\
		\hline
		\textbf{PID-Startparameter} &  &  & \\[0.5ex]
		$Kp$     & Proportionalitätsfaktor &$0 + \mathFunction{randG}{-1,1}*$\\[0.5ex]
		$Ki$     & Integrationsfaktor &$0 + \mathFunction{randG}{-1,1}*$\\[0.5ex]
		$Kd$     & Ableitungsfaktor &$0 + \mathFunction{randG}{-1,1}*$\\[0.5ex]
		$Kn$     & Filterkonstante &$0 + \mathFunction{randG}{-1,1}*$\\[0.5ex]
		\bottomrule
\end{tabular}
\caption{Hyper- \& Tuningparameter für Optimierung mit dem genetischen Algorithmus}
\label{tab:MotorMitSchwungmasse_OptimizationParameters}
\end{table}

\[
    *\mathFunction{randG}{a,b} \text{ ist eine gleichverteilte Zufallsvariable im Intervall [a, b] } 
\]
\newpage

%\horizontalLine
\subsection{Optimierung auswerten}


\begin{figure}[H]
	\center
    %\vspace{-1.6cm}
    \includegraphics[width=\linewidth]{images/MotorMitSchwungmasse/Genetic_Simulation_LearningHistory_1.pdf}
    \caption{Lernverlauf der Optimierung mit dem genetischen Algorithmus bei 10 Durchläufen}
    \label{fig:MotorMitSchwungmasseGenetic_SimulationLearningHistory_1}
\end{figure}

\minipagedOrBelowEachOther
{
    Bis die Bewertungsfunktion der Phasen- und Verstärkungsreserve aktiviert werden, konvergiert der \gls{GA} sehr zuverlässig.
    Der Sprung bei Epoche 100 ist zu erwarten, da dort die Gewichtung der Phasenreserve in der Bewertungsfunktion aktiviert wird.
    Ein Grossteil der Durchläufe finden ein schlechtes Ergebnis, besonders die Phasen- und Verstärkungsreserve-Kriterien bringen
    die Optimierung oft aus dem Gleichgewicht.
}
{
    Die Bewertungen der Phasen- und Verstärkungsreserve wird erst nach 100 Epochen aktiviert, da die Frequenzanalyse des
    im Code implementierten, zeitlichen und nichtlinearen Systems bei nicht optimalen Reglerparametern
    falsche und sprunghafte Ergebnisse liefert. 
    Dies führt zu einem Chaos in der Bewertungsfunktion und kann die Optimierung stark behindern.
    Nach ca. 100 Epochen sind die Reglerparameter meistens so gut, dass die Frequenzanalyse auch im Zeitbereich
    verlässliche Ergebnisse liefert und deshalb in die Bewertungsfunktion aufgenommen werden kann.
}

%\newpage
\horizontalLine
\subsection{Optimierten Regler am realen System testen}
\begin{figure}[H]
	\center
    %\vspace{-1.6cm}
    \includegraphics[width=\linewidth]{images/MotorMitSchwungmasse/Genetic_realResult.pdf}
    \caption{Antwort auf Stimulation des optimierten Reglers am Motor mit Schwungmasse, mit dem durch \gls{GA} optimierten PID-Regler}
    \label{fig:MotorMitSchwungmasseGenetic_realResult}
\end{figure}
\newpage