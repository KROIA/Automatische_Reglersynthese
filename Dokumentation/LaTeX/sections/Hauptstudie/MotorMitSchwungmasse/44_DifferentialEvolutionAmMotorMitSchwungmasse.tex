\section{DE Algorithmus am Motor mit Schwungmasse}
\minipagedOrBelowEachOther
{
    Im Gegensatz zum \gls{GA} wird beim \gls{DE} kein abklingender \gls{Hyperparameter} für 
    die Mutationsstärke $\mathFunction{M}{k}$ verwendet, da der Algorithmus von sich aus kleinere Mutationen durchführt,
    während die Population konvergiert.
}
{
    Die Funktionsweise des DE Algorithmus ist im Kapitel\\
	\fullref{sec:Description_DifferentialEvolutionAlgorithmus} beschrieben.
}




\subsection{Optimierung durchführen}


\minipagedOrBelowEachOther
{
	Die Software führt die Optimierung 10-mal durch, weil der \gls{DE}
	aufgrund der zufälligen Startwerte nicht immer zum gleichen Ergebnis führt.
	Bei der Optimierung wurden die Einstellungen aus der ~\ref{tab:MotorMitSchwungmasseDifferential_OptimizationParameters} verwendet.
}{
	
}

\begin{table}[H]
\centering
\begin{tabular}{@{}clrc@{}}
		\toprule
		\textbf{Parameter} & \textbf{Beschreibung} & \textbf{Wert} \\
		\midrule
		$D$                 & Anzahl der einzelnen Optimierungsdurchläufe &10\\[0.5ex]
		$N$                 & Anzahl Individuen in der Population &30\\[0.5ex]
		$G$                 & Anzahl Generationen &1000\\[0.5ex]
		$C$                 & Crossover-Wahrscheinlichkeit &0.7\\[0.5ex]
		$M$                 & Mutationsstärke &0.5\\[0.5ex]
		Anti-Windup         & Verwendete Anti-Windup Methode & Clamping \\[0.5ex]
		\hline
		\textbf{Optimierungsziele} &  &  & \\[0.5ex]
		$a_\mathbf{1}$     & Absoluter Regelungsfehler &2.8\\[0.5ex]
		$a_\mathbf{2}$     & Stellwertänderungsrate &0.1\\[0.5ex]
		$a_\mathbf{3}$     & Positives Überschwingen &111\\[0.5ex]
        $a_\mathbf{4}$     & Verstärkungsreserve (6dB) &0\\[0.5ex]
        $a_\mathbf{5}$     & Phasenreserve (90°) &99 Ab Epoche 100\\[0.5ex]
		\hline
		\textbf{Konstante Systemparameter} &  &  & \\[0.5ex]
		$u_\mathbf{satHigh}$ & PID Ausgangs Sättigung oberer Wert &10\\
		$u_\mathbf{satLow}$  & PID Ausgangs Sättigung unterer Wert &-10\\
		$I_{\mathbf{SatHigh}}$	     & PID Integrator Sättigung oberer Wert &100000000\\
		$I_{\mathbf{SatLow}}$	     & PID Integrator Sättigung unterer Wert &-100000000\\
		\hline
		\textbf{PID-Startparameter} &  &  & \\[0.5ex]
		$Kp$     & Proportionalitätsfaktor &$0 + \mathFunction{randG}{-1,1}*$\\[0.5ex]
		$Ki$     & Integrationsfaktor &$0 + \mathFunction{randG}{-1,1}*$\\[0.5ex]
		$Kd$     & Ableitungsfaktor &$0 + \mathFunction{randG}{-1,1}*$\\[0.5ex]
		$Kn$     & Filterkonstante &$0 + \mathFunction{randG}{-1,1}*$\\[0.5ex]
		\bottomrule
\end{tabular}
\caption{Hyper- \& Tuningparameter für Optimierung mit dem Differential Evolution Algorithmus}
\label{tab:MotorMitSchwungmasseDifferential_OptimizationParameters}
\end{table}

\[
    *\mathFunction{randG}{a,b} \text{ ist eine gleichverteilte Zufallsvariable im Intervall [a, b] } 
\]

%\horizontalLine
\newpage
\subsection{Optimierung auswerten}


\begin{figure}[H]
	\center
    %\vspace{-1.6cm}
    \includegraphics[width=\linewidth]{images/MotorMitSchwungmasse/Differential_Simulation_LearningHistory_1.pdf}
    \caption{Lernverlauf der Optimierung mit dem Differential Evolution Algorithmus bei 10 Durchläufen}
    \label{fig:MotorMitSchwungmasseDifferential_SimulationLearningHistory_1}
\end{figure}

\minipagedOrBelowEachOther
{
    Sehr auffällig ist, dass der \gls{DE} Algorithmus im Vergleich zum \gls{GA}
    schöner konvergiert und kein stochastisches Verhalten zeigt.
    Abgesehen von einem Durchlauf, finden alle anderen Durchläufe nahezu die gleiche Lösung.
}
{

}

\horizontalLine
\subsection{Optimierten Regler am realen System testen}
\begin{figure}[H]
	\center
    %\vspace{-1.6cm}
    \includegraphics[width=\linewidth]{images/MotorMitSchwungmasse/Differential_realResult.pdf}
    \caption{Antwort auf Stimulation des optimierten Reglers am Motor mit Schwungmasse, mit dem durch \gls{DE} optimierten PID-Regler}
    \label{fig:MotorMitSchwungmasseDifferential_realResult}
\end{figure}

\minipagedOrBelowEachOther
{

}
{
	
}
