\section{\textit{Systune} am Motor mit Schwungmasse}
\label{sec:SystuneAmMotorMitSchwungmasse}
Wie bereits im Beispiel \fullref{sec:SystuneAmDCMotor} beschrieben, 
wird hier der gleiche Ablauf, aber diesmal für das Motor-mit-Schwungmasse-System, durchgeführt.
Deshalb wird darauf verzichtet jeden Schritt in Detail erneut zu beschreiben.
Das komplette MATLAB-Skript kann im GitHub Repository \cite{baRepository} eingesehen werden.


\subsection{Regelkreis Definieren}
\minipagedOrBelowEachOther
{
    %\vspace{0.8cm}
    Der verwendete Regelkreis ist im Kapitel\\
    ~\fullref{sec:MotorMitSchwungmasseModellidentifikation} genauer beschrieben.
	Da der Regelkreis für \textit{systune} linear sein muss, werden die nichtlinearen Sättigungsblöcke vernachlässigt und 
    erst beim Testen des optimierten Systems wieder berücksichtigt.
}{
	%\centering
    \begin{figure}[H]
    \centering
    \begin{tikzpicture}[>=latex', scale=1]
    % Manually set bounding box
    \useasboundingbox (0,0) rectangle (8,-1);
    \newcommand{\labeloffset}{0.2}
    % Draw grid
    %\draw[step=1cm,gray,very thin] (0,0) grid (10,10);

    \newcommand{\plantSys}{M=$|$w$|$=M};
    \newcommand{\controllerSys}{$\mathbf{PID}$};
    
    % Draw blocks
    \sumNode{sum}{0.5,0};

    \blockNode{controller}{2,0}{\controllerSys};
    \connectionNode{analysisPoint}{0.5,-1};
    \gainNode[east]{u1gain}{$(controller.east) + (1,0)$}{300};
    \blockNode{plant}{$(u1gain.east) + (1.7,-0.25)$}{\plantSys};

    % Draw connections
    \connect[label={-0.6,0}{$\mathbf{r}$}              ]{(-0.2,0)             (sum.west)          }{->};
    \connect[label={-0.5,\labeloffset}{$\mathbf{e}$}   ]{(sum.east)        (controller.west)   }{->};
    %\connect[label={-0.5,\labeloffset}{$\mathbf{u_{Out}}$}]{(controller.east) (analysisPoint.center)}{-};
    \connect{(controller.east) (u1gain.west)   }{->};
    \connect[label={-0.3,\labeloffset}{$\mathbf{u_{1}}$}]{(u1gain.east) ($(plant.west) + (0,0.25)$)   }{->};
    %\connect[label={0, \labeloffset}{$\mathbf{u_{AP}}$}]{(analysisPoint.center) ($(analysisPoint.center) + (0, 0.5)$)   }{->, dashed};
    \connectionNode{output}{$(plant.east) + (0.5,-0.25)$};
    \connect[label={0.2,0}{$\mathbf{y_2}$}               ]{\vectorAdd{plant.east}{0,-0.25} (output.center) \vectorAdd{output}{0.5,0}}{->};
    \connect[label={0.2,0}{$\mathbf{y_1}$}               ]{\vectorAdd{plant.east}{0,0.25} \vectorAdd{plant.east}{1,0.25}}{->};
    \connect[label={-0.3,-\labeloffset}{$\mathbf{u_2}$}  ]{\vectorAdd{plant.west}{-0.5,-0.25} \vectorAdd{plant.west}{0,-0.25}}{->};
    \connect[label={0, \labeloffset}{$\mathbf{u_{AP}}$}]{(analysisPoint.center) ($(analysisPoint.center) + (-0.5, 0)$)   }{->, dashed};

    \feedback[sign={-0.3,-0.1}              ]{output.center}{-0.5}{sum.south}{->};
\end{tikzpicture}
    \caption{Regelkreis des Motors mit Schwungmasse}
    \label{fig:MotorMitSchwungmasseTikzDiagram_systune}
    \end{figure}
}



\horizontalLine
\subsection{Optimierungsziele definieren}
Bei den optimierungszielen können verschiedene Kriterien verwendet werden. 
In diesem Beispiel werden drei Ziele definiert:
\begin{itemize}
    \item \textbf{Step-Tracking}: Das System soll der Sprungantwort eines PT2-Systems folgen.
    \item \textbf{Overshoot}: Das Überschwingen soll begrenzt werden.
    \item \textbf{Phasen- und Verstärkungsreserve}: Die Stabilitätsreserven sollen eingehalten werden.
\end{itemize}

\subsubsection{TuningGoal.StepTracking \cite{matlabTuningGoals_StepTracking}}
\minipagedOrBelowEachOther
{
    Das Step-Tracking Ziel sorgt dafür, dass das System einer vordefinierten Sprungantwort folgt.
    Dafür wird ein PT2-System als Referenz verwendet.
    In diesem Beispiel wird eine Zeitkonstante von $\tau = 0.2$ und eine Dämpfung von $D = 1$ verwendet.
    Die Wahl dieser Parameter entspricht approximativ den Parametern aus der Impulsantwort welche bei der 
    Systemidentifikation in Kapitel~\fullref{sec:MotorMitSchwungmasseModellidentifikation} ermittelt wurde.
}{
	\begin{equation}
        H_{\mathbf{ref}}(s) = \frac{1}{\tau^2 \cdot s + 2 \cdot D \cdot \tau \cdot s + 1} \quad \text{mit} \quad \tau = 0.2, D = 1
    \end{equation}
}
\begin{matlabcode}{Definition des Step-Tracking Ziels}{lst:motorMitSchwungmasseStepTrackingZielDefinition_systune}
pt2T = 0.2;
pt2D  = 1;
pt2Step = tf([1],[pt2T*pt2T 2*pt2D*pt2T 1]);
TR1 = TuningGoal.StepTracking('r', 'y2', pt2Step);  
\end{matlabcode}




\subsubsection{TuningGoal.Overshoot \cite{matlabTuningGoals_Overshoot}}
Das Overshoot Ziel begrenzt das Überschwingen des Systems.
Ziel ist es, vom Referenzsignal $r$ zur Ausgangsgrösse $y2$ nicht mehr als 10\% Überschwingen zuzulassen.

\begin{matlabcode}{Definition des Overshoot Ziels}{lst:motorMitSchwungmasseOvershootZielDefinition_systune}
TR2 = TuningGoal.Overshoot('r', 'y2', 10);  
\end{matlabcode}




\subsubsection{TuningGoal.Margins \cite{matlabTuningGoals_Margins}}
\minipagedOrBelowEachOther{
    Das Margins-Ziel sorgt dafür, 
    dass die Stabilitätsreserven eingehalten werden.
    In diesem Beispiel sollen eine Phasenreserve von mindestens 45° und 
    eine Verstärkungsreserve von mindestens 6dB eingehalten werden.
}{
    Der angegebene Analysis-Point 'uAP' steht für den Teil des Feedbacks, 
    welcher für die Stabilitätsanalyse aufgetrennt wird 
    um die Messung der Phasen- und Verstärkungsreserven zu ermöglichen.
}


\begin{matlabcode}{Definition des Margins Ziels}{lst:motorMitSchwungmasseMarginsZielDefinition_systune}
TR4 = TuningGoal.Margins('uAP', 6, 45);
\end{matlabcode}


\subsubsection{Optimierungsziele kombinieren}
\begin{matlabcode}{Kombinieren der Optimierungsziele}{lst:motorMitSchwungmasseOptimierungszieleKombinieren_systune}
targetSoftGoals = [TR1  TR2]; % Step-Tracking und Overshoot
targetHardGoals = [TR4];      % Phasen- und Verstärkungsreserven
\end{matlabcode}





\newpage
\subsection{Regler extrahieren}
Nach der Optimierung kann der optimierte Regler aus dem Gesamtsystem extrahiert werden.
\begin{matlabcode}{Regler extrahieren}{lst:DCMotorReglerExtrahieren_systune}
    controllerSysTuned = getBlockValue(sysTuned, 'controllerSys');
    disp(controllerSysTuned);
\end{matlabcode}
\begin{matlabcode}{Optimierter Regler}{lst:motorMitSchwungmasseControllerSysTunedOutput_systune}
  pid with properties:

              Kp: -0.4108
              Ki: -4.6926e-08
              Kd: -0.7332
              Tf: 0.0802
              ...
\end{matlabcode}



\horizontalLine
\subsection{Optimierung auswerten}
Die Ziele wurden gut erreicht, da alle unter 1 liegen.
\begin{matlabcode}{TuningGoals Auswertung}{lst:motorMitSchwungmasseTuningGoalsAuswertungOutput_systune}
Final soft objective: 0.3065 (< 1 is good)
Final soft objective: 0.6893 (< 1 is good)
Final hard objective: 0.6066 (< 1 is good)

%=== Performance Comparison ===
%                  Initial    Tuned
%Rise Time:        NaN s    0.660 s
%Settling Time:    NaN s    1.180 s
%Overshoot:        NaN %    0.31 %
%
%Gain Margin:      Na  dB   19.83 dB
%Phase Margin:      <0 °    75.71 °
\end{matlabcode}






\newpage
\subsubsection{Step-Tracking Ergebnis visualisiert}
\begin{matlabcode}{Step-Tracking Diagramm}{lst:motorMitSchwungmasseOptimierungsanalyseOutput_systuneStepTracking}
viewGoal(targetSoftGoals(1), sysTuned);
\end{matlabcode}
\minipagedOrBelowEachOther
{
    \vspace{1cm}
    Das Diagramm in~\ref{fig:motorMitSchwungmasseSystuneTuningGoalStepTrackingResult} zeigt die Sprungantwort des 
    optimierten Systems. Die Kurve mit dem Namen \textit{Desired} entspricht der gewünschten Sprungantwort,
    die durch das Step-Tracking Ziel definiert wurde: 
    $H_{\mathbf{ref}}(s) = \frac{1}{\tau^2 \cdot s + 2 \cdot D \cdot \tau \cdot s + 1} \quad \text{mit} \quad \tau = 0.2, D = 1$.
    In blau ist die Sprungantwort des optimierten Systems dargestellt.
    Diese deckt sich hauptsächlich kurz nach dem Start sehr gut mit der gewünschten Sprungantwort.
}{
	%\centering
   % \vspace{-1cm}
    \begin{figure}[H]
        %\centering
        \includegraphics[width=\linewidth]{images/MotorMitSchwungmasse/SystuneTuningGoalSoft_1.pdf}
        \caption{Step-Tracking Optimierungsergebnis mit \textit{systune} am Motor mit Schwungmasse}
        \label{fig:motorMitSchwungmasseSystuneTuningGoalStepTrackingResult}
    \end{figure}
}

\horizontalLine
\subsubsection{Overshoot Ergebnis visualisiert}
\begin{matlabcode}{Overshoot Diagramm}{lst:motorMitSchwungmasseOptimierungsanalyseOutput_systuneOvershoot}
viewGoal(targetSoftGoals(2), sysTuned);
\end{matlabcode}
\minipagedOrBelowEachOther
{
    \vspace{0.6cm}
	Das Diagramm in~\ref{fig:motorMitSchwungmasseSystuneTuningGoalOvershootResult} zeigt den Frequenzgang des optimierten geschlossenen Systems.
    Der orange Bereich stellt den Tiel der Verstärkung dar, welcher grösser als die erlaubten 10\% Überschwingen ist.
    Die blaue Kurve zeigt die Verstärkung des optimierten Systems, diese ist im gesamten Frequenzbereich unterhalb der
    10\% Grenze, was zeigt, dass das Overshoot Ziel erfolgreich erfüllt wurde.
}{
	%\centering
    \begin{figure}[H]
        %\vspace{-1cm}
        \includegraphics[width=\linewidth]{images/MotorMitSchwungmasse/SystuneTuningGoalSoft_2.pdf}
        \caption{Overshoot Optimierungsergebnis mit \textit{systune} am Motor mit Schwungmasse}
        \label{fig:motorMitSchwungmasseSystuneTuningGoalOvershootResult}
    \end{figure}
}

\newpage
\subsubsection{Phasen- und Verstärkungsreserven Ergebnis visualisiert}
\begin{matlabcode}{Phasen- und Verstärkungsreserven Diagramm}{lst:motorMitSchwungmasseOptimierungsanalyseOutput_systunePhaseAndGainMargin}
viewGoal(targetHardGoals(1), sysTuned);
\end{matlabcode}
\minipagedOrBelowEachOther
{
    %\centering
    \begin{figure}[H]
        \centering
        \includegraphics[width=\linewidth]{images/MotorMitSchwungmasse/SystuneTuningGoalHard_1.pdf}
        \caption{Phasen- und Verstärkungsreserven Optimierungsergebnis mit \textit{systune} am Motor mit Schwungmasse}
        \label{fig:motorMitSchwungmasseSystuneTuningGoalPhaseAndGainMarginResult}
    \end{figure}
    %\vspace{-1cm}
	Das Diagramm in \ref{fig:motorMitSchwungmasseSystuneTuningGoalPhaseAndGainMarginResult} zeigt die Verstärkungs- und 
    Phasenreserven des optimierten geschlossenen Systems.
    Die blaue Kurve zeigt jeweils den Verstärkungsreserven (oben) und die Phasenreserve (unten) des optimierten Systems.
    Beide Kurven liegen über den geforderten Mindestreserven (oranger Bereich), 
    was zeigt, dass das Margins-Ziel erfolgreich erfüllt wurde.  

    \vspace{0.3cm}
    Mehr Informationen darüber, wie die Optimierungsergebnisse visualisiert werden,
    ist in der MATLAB Dokumentation zu den Tuning-Goals beschrieben. \cite{matlabTuningGoals_VisualizeTuningGoals}
}{
	%\centering
    \begin{figure}[H]
        \centering
        \includegraphics[width=0.838\linewidth]{images/MotorMitSchwungmasse/SystuneStabilityDelayMarginsPlot.pdf}
        \caption{Totzeit \& Verstärkungsreserve Diagramm des optimierten Systems}
        \label{fig:motorMitSchwungmasseSystuneStabilityMarginsPlot}
    \end{figure}

    Das von MATLAB generierte Diagramm kann verwirren und ist nicht einfach zu interpretieren.
    Eine bessere Darstellung über die resultierten Phasen- und Verstärkungsreserven ist in
    \ref{fig:motorMitSchwungmasseSystuneStabilityMarginsPlot} zu sehen.
    Wie diese Abbildung interpretiert wird, ist in folgendem Kapitel erklärt:\\
\fullref{sec:StabilitaetsDelayMarginPlot}


    Das Diagramm zeigt, dass das System über eine grosse Phasenreserve
    verfügt. Das zeigt sich durch die Toleranz der möglichen Totzeiten mit bis zu 0.6s
    bei einer Verstärkung von 1.
    Ausserdem besitzt das System eine gute Verstärkungsreserve mit bis zu 20dB\\
    (Verstärkungsfaktor = 10).
}
\newpage

%\minipagedOrBelowEachOther
%{
%    
%}{
%	
%}



%\horizontalLine
\subsection{Optimierten Regler am realen System testen}

%\subsubsection{Reales System Resultat}
\begin{figure}[H]
    \centering
    \includegraphics[width=\linewidth]{images/MotorMitSchwungmasse/Systune_realResult.pdf}
    \caption{Antwort auf Stimulation des optimierten Reglers am Motor mit Schwungmasse, mit dem durch \textit{systune} optimierten PID-Regler}
    \label{fig:MotorMitSchwungmasseSystune_realResult}
\end{figure}

\minipagedOrBelowEachOther
{
    Für den Test am realen Prozess sind im Simulink Modell unterschiedliche Sollwerte vorbereitet. 
    Auch die Sättigungen sind wieder eingebaut.
    Da der PID-Regler am Ausgang einen Sättigungsblock besitzt, entsteht das Problem des Integrator Windups.
    Die \textit{Clamping} Anti-Windup Methode wird verwendet, da diese keinen weiteren Tuning-Parameter erzeugt.
    \textit{Systune} kann leider keinen Anti-Windup Tuning-Parameter optimieren.
}
{
    Der Regler verstärkt an seinem Ausgang $u1$ das Messrauschen des Encoders, welches zu starken 
    Stromschwankungen am Motoreingang führen. 
    Es ist zu erkennen, dass der Ist-Winkel $y2$, sobald die Störgrösse dazu geschaltet wird, davon driftet.
    Der Regler besitzt deshalb keine guten Eigenschaften um Störgrössen zu kompensieren.
}
\newpage