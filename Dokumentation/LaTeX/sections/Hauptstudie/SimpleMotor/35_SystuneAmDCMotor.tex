\section{\textit{Systune} am DC-Motor}
\label{sec:SystuneAmDCMotor}
Für \textit{systune} wird MATLAB benötigt, nachfolgend wird Schritt für Schritt aufgezeigt,
wie die Optimierung in MATLAB umgesetzt wird.


\subsection{Regelkreis Definieren}
\minipagedOrBelowEachOther
{
    %\vspace{0.8cm}
    Der verwendete Regelkreis ist im Kapitel\\~\fullref{sec:DCMotorModell} genauer beschrieben.
	Da der Regelkreis für \textit{systune} linear sein muss, werden die nichtlinearen Sättigungsblöcke vernachlässigt und 
    erst beim Testen des optimierten Systems wieder berücksichtigt.
    Das komplette MATLAB-Skript kann im GitHub Repository \cite{baRepository} eingesehen werden.
}{
	%\centering
    \begin{figure}[H]
    \centering
    \begin{tikzpicture}[>=latex', scale=1]
    % Manually set bounding box
    \useasboundingbox (0,0) rectangle (8,-1);
    \newcommand{\labeloffset}{0.2}
    % Draw grid
    %\draw[step=1cm,gray,very thin] (0,0) grid (10,10);

    \newcommand{\plantSys}{$\mathbf{Motor}=\frac{0.9}{0.14 \cdot s + 1}$};
    \newcommand{\controllerSys}{$\mathbf{PID}$};
    
    % Draw blocks
    \sumNode{sum}{0.5,0};

    \blockNode{controller}{2,0}{\controllerSys};
    \connectionNode{analysisPoint}{$(controller.east) + (1,0)$};
    \blockNode{plant}{$(analysisPoint.center) + (2,0)$}{\plantSys};

    % Draw connections
    \connect[label={-0.6,0}{$\mathbf{r}$}              ]{(-0.2,0)             (sum.west)          }{->};
    \connect[label={-0.5,\labeloffset}{$\mathbf{e}$}   ]{(sum.east)        (controller.west)   }{->};
    \connect[label={-0.5,\labeloffset}{$\mathbf{u_{Out}}$}]{(controller.east) (analysisPoint.center)}{-};
    \connect[label={-0.5,\labeloffset}{$\mathbf{u_{In}}$}]{(analysisPoint.center) (plant.west)   }{->};
    \connect[label={0, \labeloffset}{$\mathbf{u_{AP}}$}]{(analysisPoint.center) ($(analysisPoint.center) + (0, 0.5)$)   }{->, dashed};
    \connectionNode{output}{$(plant.east) + (0.5,0)$};
    \connect[label={0.2,0}{$\mathbf{y}$}               ]{(plant.east) \vectorAdd{output}{0.5,0}}{->};

    \feedback[sign={-0.3,-0.1}              ]{output.center}{-1}{sum.south}{->};
\end{tikzpicture}
    \caption{Regelkreis für den DC-Motor}
    \label{fig:SimpleMotorTikzDiagram_systune}
    \end{figure}
}

\horizontalLine
\subsubsection{Motor definieren}
\begin{matlabcode}{Definition des DC-Motors}{lst:DCMotorDC_MotorDefinition_systune}
T = 0.14;  % Zeitkonstante
K1 = 0.9;  % DC-Verstärkung
plantSys = tf(K1, [T, 1]);
plantSys.InputName = 'uIn';
plantSys.OutputName = 'y';
plantSys.Name = 'plantSys';
\end{matlabcode}

\horizontalLine
\subsubsection{Tunable PID-Regler definieren}
Mit der Funktion \textit{tunablePID} aus der Control System Toolbox \cite{matlabControlSystemToolbox}
kann ein PID-Regler erstellt werden, der sogenannte \textit{tunable} Parameter besitzt.
Diese Parameter können anschliessend mit \textit{systune} optimiert werden.

\begin{matlabcode}{Definition des PID-Reglers}{lst:DCMotorPID_ReglerDefinition_systune}
controllerSysInitial = tunablePID('controllerSys', 'PID');
% Set initial values (optional)
controllerSysInitial.Kp.Value = 1;
controllerSysInitial.Ki.Value = 1;

% Name the controller inputs and outputs
controllerSysInitial.InputName = 'e';
controllerSysInitial.OutputName = 'uOut';
\end{matlabcode}


\horizontalLine
\subsubsection{Analysis-Point definieren}
Ein Analysis-Point kann verwendet werden um während der Optimierung
bestimmte Signale zu überwachen oder zu beeinflussen \cite{matlabSystune}.
Dieser Analysis-Point kann später bei den~\fullref{sec:Optimierungsziele_definieren} verwendet werden.
\begin{matlabcode}{Definition des Analysis-Points}{lst:DCMotorAnalysisPointDefinition_systune}
G_ap = AnalysisPoint('uAP');
G_ap.InputName = 'uOut';
G_ap.OutputName = 'uIn';
\end{matlabcode}


\horizontalLine
\subsubsection{Feedback Summation definieren}
\begin{matlabcode}{Definition des Feedback Summation Blocks}{lst:DCMotorFeedbackSummationDefinition_systune}
sum_block = sumblk('e = r - y');
\end{matlabcode}


\horizontalLine
\subsubsection{Alle Teilsysteme verbinden}
\begin{matlabcode}{Definition des Gesamtsystems}{lst:DCMotorGesamtsystemDefinition_systune}
% System inputs: 'r'
% System outputs: 'y', 'uIn', 'e'
sys0 = connect(plantSys, G_ap, controllerSysInitial, sum_block, {'r'}, {'y', 'uIn', 'e'});
sys0.Name = 'Untuned system';
\end{matlabcode}
$sys0$ ~\ref{fig:SimpleMotorTikzDiagram_systune} ist nun das nicht optimierte \gls{Gesamtsystem}, welches den DC-Motor, 
den PID-Regler, den Analysis-Point und das Feedback beinhaltet. 




\newpage
\subsection{Optimierungsziele definieren}
\label{sec:Optimierungsziele_definieren}
Bei den Optimierungszielen können verschiedene Kriterien verwendet werden. 
In diesem Beispiel werden zwei Ziele definiert:
\begin{itemize}
    \item \textbf{Step-Tracking}: Das System soll der Sprungantwort des Motors folgen.
    \item \textbf{Overshoot}: Das Überschwingen soll begrenzt werden.
\end{itemize}
Diese Ziele benötigen keinen Analyis-Point, deshalb wurde dieser nur zur Demonstration definiert.
Es gibt aber andere Optimierungsziele, die auf einen Analysis-Point angewiesen sind.


\subsubsection{TuningGoal.StepTracking \cite{matlabTuningGoals_StepTracking}}
\minipagedOrBelowEachOther
{
    Das Step-Tracking Ziel sorgt dafür, dass das \gls{Gesamtsystem} einer Sprungantwort folgt.
    In diesem Fall wird nur eine Zahl angegeben, welche vom Step-Tracking als PT1 System interpretiert wird:  
}{
	\begin{equation}
        H_{\mathbf{ref}}(s) = \frac{\frac{1}{\tau}}{s + \frac{1}{\tau}} \quad \text{mit} \quad \tau = 0.014
    \end{equation}
}

% math
%\minipagedOrBelowEachOther{
Das Referenzsystem $H_{\mathbf{ref}}(s)$ entspricht, abgesehen von der DC-Verstärkung, der Übertragungsfunktion des Motors.
Damit wird ein möglichst schnelles Ansprechverhalten des geschlossenen Regelkreises angestrebt.
%}
%{

%}

\begin{matlabcode}{Definition des Step-Tracking Ziels}{lst:DCMotorStepTrackingZielDefinition_systune}
TR1 = TuningGoal.StepTracking('r', 'y', 0.014);  
\end{matlabcode}


\subsubsection{TuningGoal.Overshoot \cite{matlabTuningGoals_Overshoot}}
Das Overshoot Ziel begrenzt das Überschwingen des Systems.
Ziel ist es, vom Referenzsignal $r$ zur Ausgangsgrösse $y$ nicht mehr als 10\% Überschwingen zuzulassen.

\begin{matlabcode}{Definition des Overshoot Ziels}{lst:DCMotorOvershootZielDefinition_systune}
TR2 = TuningGoal.Overshoot('r', 'y', 10);  
\end{matlabcode}

\subsubsection{Optimierungsziele kombinieren}
Die beiden definierten Ziele werden in zwei Vektoren gesammelt, es können beliebig viele Ziele definiert werden.
Die Ziele werden in harte und weiche Ziele unterteilt.
Harte Ziele müssen zwingend erfüllt werden, während weiche Ziele nur angestrebt werden.
\begin{matlabcode}{Kombinieren der Optimierungsziele}{lst:DCMotorOptimierungszieleKombinieren_systune}
targetSoftGoals = [TR1]; % Step Tracking
targetHardGoals = [TR2]; % Overshoot
\end{matlabcode}

\horizontalLine
\subsection{Optimierung durchführen}
\label{sec:Optimierung_durchfuehren}
Mit dem Befehl \textit{systune} wird die Optimierung durchgeführt \cite{matlabSystune}.
Als Eingabe werden das nicht optimierte System $sys0$ sowie die definierten Optimierungsziele übergeben.
Als Ausgabe wird das optimierte System $sysTuned$ sowie die Optimierungsergebnisse der 
weichen $fSoft$ und harten $fHard$ Ziele zurückgegeben.
\begin{matlabcode}{Optimierung durchführen}{lst:DCMotorOptimierungDurchfuehren_systune}
    [sysTuned, fSoft, fHard] = systune(sys0, targetSoftGoals, targetHardGoals);
    sysTuned.Name = 'Tuned system';
\end{matlabcode}

\subsubsection{Regler extrahieren}
Nach der Optimierung kann der optimierte Regler aus dem Gesamtsystem extrahiert werden.
\begin{matlabcode}{Regler extrahieren}{lst:DCMotorReglerExtrahieren_systune}
    controllerSysTuned = getBlockValue(sysTuned, 'controllerSys');
    disp(controllerSysTuned);
\end{matlabcode}



\begin{matlabcode}{Optimierter Regler}{lst:DCMotorcontrollerSysTunedOutput_systune}
  pid with properties:

              Kp: 11.1112
              Ki: 79.3650
              Kd: -1.8759e-04
              Tf: 1.9757
              ...
\end{matlabcode}


\newpage
\subsection{Optimierung auswerten}
\label{sec:Optimierung_auswerten}



\textit{Systune} liefert für jedes Optimierungsziel einen Wert zurück, der angibt, wie gut das jeweilige Ziel erfüllt wurde.
\begin{matlabcode}{TuningGoals auswerten}{lst:DCMotorTuningGoalsAuswerten_systune}
% Display final objective value
fprintf('Final soft objective: %.4f (< 1 is good)\n', fSoft);
fprintf('Final hard objective: %.4f (< 1 is good)\n', fHard);
\end{matlabcode}



Das erste Soft-Ziel (Step-Tracking) wurde sehr gut erfüllt, da der Wert nahe bei 0 liegt.
Das zweite Hard-Ziel (Overshoot) wurde ebenfalls erfüllt, da der Wert unter 1 liegt.
\begin{matlabcode}{TuningGoals Auswertung}{lst:DCMotorTuningGoalsAuswertungOutput_systune}
Final soft objective: 0.0000 (< 1 is good)
Final hard objective: 0.6893 (< 1 is good)
\end{matlabcode}



Es können auch noch mehr Analysen durchgeführt werden, um die Optimierung zu bewerten.
\begin{matlabcode}{Optimierung analysieren}{lst:DCMotorOptimierungAnalysieren_systune}
% Create closed-loop systems
CL_initial = feedback(controllerSysInitial*plantSys, 1);
CL_tuned   = feedback(controllerSysTuned*plantSys, 1);

% Get step response info
info_initial = stepinfo(CL_initial);
info_tuned = stepinfo(CL_tuned);

fprintf('\n=== Performance Comparison ===\n');
fprintf('                  Initial    Tuned\n');
fprintf('Rise Time:        %.3f s    %.3f s\n', info_initial.RiseTime, info_tuned.RiseTime);
fprintf('Settling Time:    %.3f s    %.3f s\n', info_initial.SettlingTime, info_tuned.SettlingTime);
fprintf('Overshoot:        %.2f %%    %.2f %%\n', info_initial.Overshoot, info_tuned.Overshoot);

% Check margins
[Gm_i, Pm_i] = margin(plantSys*controllerSysInitial);
[Gm_t, Pm_t] = margin(plantSys*controllerSysTuned);

fprintf('\nGain Margin:      %.2f dB   %.2f dB\n', 20*log10(Gm_i), 20*log10(Gm_t));
fprintf('Phase Margin:     %.2f °    %.2f °\n', Pm_i, Pm_t);
\end{matlabcode}

\begin{matlabcode}{Optimierungsanalyse Ausgabe}{lst:DCMotorOptimierungsanalyseOutput_systune}
%=== Performance Comparison ===
%                  Initial    Tuned
%Rise Time:        3.370 s    0.031 s
%Settling Time:    6.662 s    0.055 s
%Overshoot:        0.00 %     0.00 %
%
%Gain Margin:      Inf dB     Inf dB
%Phase Margin:     136.74 °   90.00 °
\end{matlabcode}



\newpage
\subsubsection{Step-Tracking Ergebnis visualisiert}
\begin{matlabcode}{Step-Tracking Diagramm}{lst:DCMotorOptimierungsanalyseOutput_systuneStepTracking}
viewGoal(targetSoftGoals(1), sysTuned);
\end{matlabcode}
\minipagedOrBelowEachOther
{
    \vspace{1.1cm}
    Das Diagramm in \ref{fig:SimpleMotorSystuneTuningGoalStepTrackingResult} zeigt die Sprungantwort des 
    optimierten Systems. Die Kurve mit dem Namen \textit{Desired} entspricht der gewünschten Sprungantwort,
    die durch das Step-Tracking Ziel definiert wurde: $H_{\mathbf{ref}}(s) = \frac{1}{\tau \cdot s + 1}$ mit $\tau = 0.014$.\\
    In blau ist die Sprungantwort des optimierten Systems dargestellt.
    Diese deckt sich sehr gut mit der gewünschten Sprungantwort, was zeigt, dass das Step-Tracking Ziel sehr gut erfüllt wurde.


    
}{
	%\centering
    %\vspace{-1cm}
    \begin{figure}[H]
        %\centering
        \includegraphics[width=\linewidth]{images/SimpleMotor/SystuneTuningGoalSoft_1.pdf}
        \caption{Step-Tracking Optimierungsergebnis mit \textit{systune} am DC-Motor}
        \label{fig:SimpleMotorSystuneTuningGoalStepTrackingResult}
    \end{figure}
}

\horizontalLine
\subsubsection{Overshoot Ergebnis visualisiert}
\begin{matlabcode}{Overshoot Diagramm}{lst:DCMotorOptimierungsanalyseOutput_systuneOvershoot}
viewGoal(targetHardGoals(1), sysTuned);
\end{matlabcode}
\minipagedOrBelowEachOther
{
    \vspace{0.7cm}
	Das Diagramm in \ref{fig:SimpleMotorSystuneTuningGoalOvershootResult} zeigt den Frequenzgang des optimierten geschlossenen Systems.
    Der orange Bereich stellt den Teil der Verstärkung dar, welcher grösser als die erlaubten 10\% Überschwingen ist.
    Die blaue Kurve zeigt die Verstärkung des optimierten Systems, diese ist im gesamten Frequenzbereich unterhalb der
    10\% Grenze, was zeigt, dass das Overshoot Ziel erfolgreich erfüllt wurde.\\

    %\begin{matlabcode}{Overshoot Diagramm}{lst:DCMotorOptimierungsanalyseOutput_systuneOvershoot}
    %viewGoal(targetHardGoals(1), sysTuned);
    %\end{matlabcode}

    Mehr Informationen darüber, wie die Optimierungsergebnisse visualisiert werden,
    ist in der MATLAB Dokumentation zu den Tuning-Goals beschrieben. \cite{matlabTuningGoals_VisualizeTuningGoals}

    
}{
	%\centering
    \begin{figure}[H]
        %\vspace{-1cm}
        \includegraphics[width=\linewidth]{images/SimpleMotor/SystuneTuningGoalHard_1.pdf}
        \caption{Overshoot Optimierungsergebnis mit systune am DC-Motor}
        \label{fig:SimpleMotorSystuneTuningGoalOvershootResult}
    \end{figure}
}




\newpage
\subsection{Optimierten Regler am realen System testen}
%\subsubsection{Simulink Resultat}
%\label{sec:Optimierten_Regler_testen}
%\begin{figure}[H]
%    \centering
%    \includegraphics[width=\linewidth]{images/MotorSystemSystune_simulinkResult.pdf}
%    \caption{Simulink Simulationsergebnis des optimierten Reglers am DC-Motor}
%    \label{fig:MotorSystemSystune_simulinkResult}
%\end{figure}
%
%\minipagedOrBelowEachOther
%{
%    Für den Test sind im Simulink Modell unterschiedliche Sprünge vorbereitet. Auch die Sättigungen sind wieder eingebaut.
%    Da der PID-Regler am Ausgang einen Sättigungsblock besitzt, entsteht das Problem des Integrator Windups.
%    Deshalb wird die \textit{Clamping} Anti-Windup Methode verwendet, da diese keinen weiteren Tuning-Parameter benötigt.
%    Mit systune kann leider kein Anti-Windup Tuning-Parameter optimieren.
%}
%{
%    Mit der Clamping Methode des Simulink PID-Blocks entstehen aber schwingungen am ausgang des Reglers wenn $r$ auf 0 wechselt.
%    Dies ist im~\ref{fig:MotorSystemSystune_simulinkResult} bei den Zeiten um 3s oder 5s zu erkennen.
%    Mehr zu diesem Verhalten und wie man dies verhindern kann, ist im~\fullref{sec:Description_SimulinkPID_ClampingLogic_Problem} beschrieben.
%}

%\subsubsection{Reales System Resultat}
\begin{figure}[H]
    \centering
    \includegraphics[width=\linewidth]{images/SimpleMotor/Systune_realResult.pdf}
    \caption{Antwort auf Stimulation des optimierten Reglers am DC-Motor, mit dem durch \textit{systune} optimierten PID-Regler}
    \label{fig:SimpleMotorSystune_realResult}
\end{figure}

\minipagedOrBelowEachOther
{
    Für den Test am Realen Prozess sind im Simulink Modell unterschiedliche Sprünge vorbereitet. 
    Das Signal $r$ definiert eine Spannung die proportional für eine gewünschte Drehzahl steht.
    Die Last $l$ steuert zwei Wirbelstrombremsen an, wobei ein Wert von 1 bedeutet, 
    dass nur eine Bremse aktiv ist und ein Wert von 2 bedeutet, dass beide Bremsen aktiv sind.
    Auch die Sättigungen sind wieder eingebaut.\\


    Da der PID-Regler am Ausgang einen Sättigungsblock besitzt, entsteht das Problem des Integrator Windups.
    Die \textit{Clamping} Anti-Windup Methode wird verwendet, da diese keinen weiteren Tuning-Parameter erzeugt.
    \textit{Systune} kann leider keinen Anti-Windup Tuning-Parameter optimieren.\\


    Mit der Clamping Methode des Simulink PID-Blocks entsteht ein ungewöhnliches Verhalten am Ausgang des Reglers, wenn $r$ auf 0 wechselt.
    Dies ist im~\ref{fig:SimpleMotorSystune_realResult} z.B. bei der Zeit um 3s zu erkennen.
    Mehr zu diesem Verhalten und wie man dies verhindern kann, ist in folgendem Kapitel beschrieben:\\
    ~\fullref{sec:Description_SimulinkPID_ClampingLogic_Problem}\\


    Was deutlich auffällt ist, dass der Regler-Ausgang $u$ sehr starkes Rauschen aufweist.
    Dieses Rauschen wird an den Motor weitergegeben und ist deshalb auch in der 
    Drehzahl $y$ zu erkennen. 
}
{
    

    \begin{figure}[H]
		\centering
		\input{images/SimpleMotor/SimpleMotorTikzDiagram_realSystem}
		\caption{Regelkreis des DC-Motors}
		\label{fig:SimpleMotorTikzDiagram_realSystem}
	\end{figure}
}
\newpage
