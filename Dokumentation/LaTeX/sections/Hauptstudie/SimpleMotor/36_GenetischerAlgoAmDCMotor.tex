\section{Genetischer Algorithmus am DC-Motor}

\newcommand{\individuumI}{\overrightarrow{I_{\mathbf{i}}}}
\newcommand{\funcGOfk}[1]{\mathFunction{g_\mathbf{#1}}{\individuumI,k}}
\newcommand{\funcGOf}[1]{\mathFunction{g_\mathbf{#1}}{\individuumI}}

\minipagedOrBelowEachOther
{
%Leider lässt sich der PID nicht mit Ziegler Nichols Vorbestimmen weil der DC-Motor ein PT1-Verhalten hat.
%Deshalb wird direkt mit dem Genetischen Algorithmus optimiert.
Weil der genetische Algorithmus kein MATLAB voraussetzt, ist diese Methode auch ohne MATLAB implementiert worden.
Das hat den Vorteil, dass ein System auch ohne MATLAB Lizenz optimiert werden kann.
}
{
	Die Funktionsweise des \gls{GA} ist im Kapitel\\
	\fullref{sec:Description_GenetischerAlgorithmus} beschrieben.

}

\subsection{System Definieren}
\minipagedOrBelowEachOther
{
	%\vspace{0.8cm}
	In der C++ Anwendung werden die Teilsysteme objektorientiert implementiert und anschliessend 
	als ein gesamtheitliches Objekt zusammengeführt und verbunden.
	Der Kernteil der Implementierungslogik spielt sich dabei in den Update-Funktionen der jeweiligen Teilsysteme ab.
	Aufgrund des Codes-Umfangs werden deshalb in den nachfolgenden Unterkapiteln nur die Update-Funktionen gezeigt.
}{
	\vspace{-1cm}
	%\centering
    \begin{figure}[H]
		\centering
		\begin{tikzpicture}[auto, node distance=2cm,>=latex', scale=1]
    \useasboundingbox (0,0.5) rectangle (6.5,-1.5);
    \newcommand{\labeloffset}{0.2}
    % Draw grid
    %\draw[step=1cm,gray,very thin] (0,0) grid (10,10);

    \newcommand{\plantSys}{$\mathbf{Motor}$};
    \newcommand{\controllerSys}{$\mathbf{PID}$};
    \newcommand{\delaySys}{$\mathbf{z^{-1}}$};
    
    % Draw blocks
    \sumNode{sum}{0.5,0};

    \blockNode{controller}{2,0}{\controllerSys};
    \blockNode{plant}{5,0}{\plantSys};
    \blockNode{delay}{3.5,-1}{\delaySys};

    % Draw connections
    \connect[label={-0.45,0}{$\mathbf{r}$}              ]{(0,0)             (sum.west)          }{->};
    \connect[label={-0.5,\labeloffset}{$\mathbf{e}$}   ]{(sum.east)        (controller.west)   }{->};
    \connect[label={-1,\labeloffset}{$\mathbf{u}$}]   {(controller.east) (plant.west)   }{->};
    \connectionNode{output}{$(plant.east) + (0.5,0)$};
    \connect[label={0.2,0}{$\mathbf{y}$}               ]{(plant.east) \vectorAdd{output}{0.5,0}}{->};
    \connect[label={-5.18,0.5}{$\mathbf{l}$}              ]{(0,1) (5,1) (plant.north)}{->};

    \feedback[sign={-0.3,-0.1}              ]{delay.west}{0}{sum.south}{->};
    \connect{(output.center) ($(output.center) + (0,-1)$) (delay.east)}{->};
    %\connect{(delay.west) ($(sum.center) + (0,-1)$) (sum.south)}{->};

    %\feedback[sign={-0.3,-0.1}              ]{output.center}{-1}{sum.south}{->};
\end{tikzpicture}
		\caption{Regelkreis des DC-Motors}
		\label{fig:SimpleMotorTikzDiagram_genetic}
	\end{figure}
}




\horizontalLine
\subsubsection{Motor definieren}
\minipagedOrBelowEachOther
{
	Die Implementierung im C++ Code erfolgt natürlich diskretisiert.
	Wie genau die Diskretisierte Implementierung aussieht wird nicht weiter erläutert und kann im Code eingesehen werden.
	Als Grundlage für die Implementierung dient die ~\ref{form:formulaDCMotorCModel}.\\
	Die dazugehörige C++ Implementierung ist im folgenden Codeausschnitt ersichtlich:\\
	~\fullref{lst:DC_MotorDefinitionCPP}
}{
	\centering
    \formulaWithDerivationShow{formulaDCMotorCModel} % Show formula 
}





\horizontalLine
\subsubsection{PID-Regler definieren}
Der PID-Regler wurde bereits im Kapitel \fullref{sec:ErweiterterPIDRegler} beschrieben und 
auch die C++ Implementierung wird dort aufgeführt. Deshalb wird hier nicht weiter darauf eingegangen.



\horizontalLine
\subsubsection{Alle Teilsysteme verbinden}

\begin{cppcode}{Update-Funktion des Gesamtsystems im Code}{lst:DC_MotorRegelkreisCPP}
// Update Funktion des Gesamtsystems im C++ Code. class TestSystem
void update(double deltaTime) override
{
	double y = m_dcMotorSystem.getAngularVelocity();	// y(t) * z^{-1}, Letzter Ausgang des Motors
	m_errorValue = m_referenceValue - y;				// e(t) = r(t) - y(t)

	m_pidController.setInput(m_errorValue);
	m_pidController.update(deltaTime);
	m_pidOutputValue = m_pidController.getOutput();	    // u(t)

	m_dcMotorSystem.setInputs(m_pidOutputValue, m_disturbanceValue);
	m_dcMotorSystem.update(deltaTime);
}
\end{cppcode}



\newpage
\subsection{Optimierungsziele definieren}
\begin{figure}[H]
    \centering
    \includegraphics[width=\linewidth]{images/SimpleMotor/Stimuli.pdf}
    \caption{Anregungssignale für die Systemsimulation}
    \label{fig:SimpleMotorStimuli}
\end{figure}

\minipagedOrBelowEachOther
{
	Um die Bewertungsfunktion zu definieren wird eine zeitliche Simulation des Systems benötigt.
	Dazu wird ein vordefiniertes Anregungssignal verwendet.
	Das Anregungssignal besteht aus $r$ (Ziel-Drehzahl) und $l$ (Last/Störgrösse) und ist 
	so gewählt, dass verschiedene Aspekte des Regelverhaltens bewertet werden können.
}
{
	Einerseits ermöglichen die unterschiedlichen Sprünge im Ziel-Drehzahl $r$ eine Beurteilung 
	über das Verhalten des Reglers auf den Prozess, der eine nichtlineare DC-Verstärkung hat.
	Andererseits wird durch die Nutzung der Wirbelstrombremsen $l$, die Einwirkung von Störgrössen mitberücksichtigt.
	Mit diesem Signal ist es dem \gls{GA} möglich, den Regler auf verschiedene Situationen zu optimieren.
}
\newpage

\begin{table}[H]
\centering
\begin{tabular}{@{}clrc@{}}
		\toprule
		\textbf{Symbol} & \textbf{Beschreibung} & \textbf{Wert} & \textbf{Einheit} \\
		\midrule
		$Ts$                & Abtastzeit &0.01& s\\[0.5ex]
		$t_\mathbf{start}$  & Startzeit &0& s\\[0.5ex]
		$t_\mathbf{end}$    & Endzeit &20& s\\[0.5ex]
		$u_\mathbf{satHigh}$   & Obere Sättigungsgrenze der Stellgrösse &10& V\\[0.5ex]
		$u_\mathbf{satLow}$   & Untere Sättigungsgrenze der Stellgrösse &0& V\\[0.5ex]
		%$N$                 & Anzahl Individuen in der Population &30\\[0.5ex]
		$K$                 & Anzahl Zeitschritte in der Simulation & 2000\\[0.5ex]
		$D$				  & Anzahl der Optimierungs-Parameter & 4 & \\[0.5ex]
		$\individuumI$    & \gls{Individuum} $\mathbf{i}$ in der \gls{Population} & $\individuumI \in \mathbb{R}^D$ \\[0.5ex]
		%$\mathFunction{f}{\individuumI}$ & Maximierungs-Fitness-Funktion.& $f(\individuumI) \in \mathbb{R}^{+}$ \\[0.5ex]
		$\mathFunction{g}{\individuumI}$ & Fehler-Minimierungs-Funktion. & $g(\individuumI) \in \mathbb{R}$ \\[0.5ex]
		$a_\mathbf{o}$      & Gewichtungsfaktor für das Optimierungsziel $\mathbf{o}$ & $a_\mathbf{o} \in \mathbb{R}^{+_0}$ \\[0.5ex]
		$O$                 & Anzahl Optimierungsziele & 3 \\[0.5ex]
		$\funcGOfk{o}$ & Optimierungsfunktion für das Ziel $\mathbf{o}$ zum Zeitpunkt $k$ & $\funcGOfk{o} \in \mathbb{R}$ \\[0.5ex]
		\bottomrule
\end{tabular}
\caption{Symbole für die Simulation am DC-Motor}
\end{table}

%\horizontalLine
Für die Abtastzeit $Ts$ wurde ein realistischer Wert von 10ms gewählt.\\




\minipagedOrBelowEachOther
{
	Die Anzahl Zeitschritte $K$ wird aus der Abtastzeit $T_\mathbf{s}$ und 
	dem Simulationszeitraum $[t_\mathbf{start}, t_\mathbf{end}]$ berechnet.  
}{
	\centering
    \begin{equation}
		K = \frac{t_\mathbf{end}-t_\mathbf{start}}{T_\mathbf{s}}
	\end{equation}
}




\horizontalLine

\minipagedOrBelowEachOther
{
	Der \gls{GA} ist für \gls{Maximierungsproblem}e ausgelegt. 
	Für diese Anwendung wäre es aber sinnvoller die Optimierungsziele als \gls{Minimierungsproblem}e zu formulieren.
	Es gibt jedoch eine Möglichkeit zur Umrechnung von \gls{Minimierungsproblem}en in Maximierungsprobleme.
	Dies ist im Kapitel~\fullref{sec:Description_GenetischerAlgorithmus_FitnessFunction_TransformationSteps} beschrieben.
	Dank dieser Umrechnung können die Optimierungsziele als Minimierungsproblem formuliert werden.

	Die Fehler-Minimierungs-Funktion $g(\individuumI)$
	wird aus mehreren Teilsummen zusammengesetzt. Die Teilsummen bewerten verschiedene Aspekte des Regelverhaltens.
	Allgemein lässt sich die Fehler-Minimierungs-Funktion $g(\individuumI)$ wie in der
    \ref{form:SimpleMotorGeneticAlgorithmErrorMinimizationFunction} darstellen.
}{
	\centering
	\begin{equation}
        \label{form:SimpleMotorGeneticAlgorithmErrorMinimizationFunction}
		\mathFunction{g}{\individuumI} = 
		\frac{1}{K} \cdot \sum_{k=0}^{K-1}\bigg(
				\sum_{o=0}^{O-1}\Big(
					a_\mathbf{o} \cdot \funcGOfk{o}
				\Big)
			\bigg)
	\end{equation}
	
	\begin{itemize}
		\item $\frac{1}{K}$ dient zur Normierung der Fehlerbewertung auf die Anzahl der Zeitschritte.
		\item $\sum_{k=0}^{K-1}$ summiert die Fehlerbewertung über alle Zeitschritte der Simulation.
		\item $\sum_{o=0}^{O-1}\left(
					a_\mathbf{o} \cdot \funcGOfk{o}
				\right)$ summiert die gewichteten Optimierungsziele $\funcGOfk{o}$ für den Zeitschritt $k$.
	\end{itemize}
}






%\begin{equation}
%	\mathFunction{g}{\individuumI} = 
%	\frac{1}{t_\mathbf{end}-t_\mathbf{start}}\sum_{k=0}^{K}(
%		a_\mathbf{1} \cdot \mathFunction{g_\mathbf{1}}{k} + 
%		a_\mathbf{2} \cdot \mathFunction{g_\mathbf{2}}{k}
%		)
%\end{equation}

%\horizontalLine



\newpage
\label{sec:Description_GenetischerAlgorithmus_FitnessFunction_OptimizationGoals_SimpleMotor}
\subsubsection{$\funcGOfk{0}$: Absoluter Regelungsfehler mit Sättigungsbedingung}
\minipagedOrBelowEachOther
{
	$\funcGOfk{0}$ zielt darauf ab, den Absoluten Regelungsfehler zu minimieren.
	An dieser Stelle könnte berechtigterweise die Frage aufkommen, warum nicht das Quadratische Fehlerintegral verwendet wird?
	Tests haben gezeigt, dass es grundsätzlich egal ist ob das absolute oder das quadratische Fehlerintegral
	als Optimierungsziel verwendet wird, der Unterschied liegt in der Anwenderfreundlichkeit.
	Für die Optimierung mit den verschiedenen Zielen müssen passende Gewichtungen gewählt werden damit das gewünschte
	Verhalten erreicht wird. Bei der Verwendung des quadratischen Fehlerintegrals
	ist es schwieriger passende Gewichtungen zu finden als beim absoluten Fehlerintegral.
	Mit dem absoluten Fehlerintegral konvergiert der genetische Algorithmus in einem grösseren Bereich 
	des Gewichtes für das Optimierungsziel des Fehlerintegrals.

	      
}{
	\begin{equation}
		\label{form:SimpleMotorGeneticAlgorithmAbsoluteErrorWithSaturationCondition}
		\begin{aligned}
			ignoreError ={}&
			\Big(\mathFunction{u}{\individuumI, k} \ge u_{\mathbf{sat+}} \land \mathFunction{r}{k} > \mathFunction{y}{\individuumI, k}\Big) \lor \\
			&\Big(\mathFunction{u}{\individuumI, k} \le u_{\mathbf{sat-}} \land \mathFunction{r}{k} < \mathFunction{y}{\individuumI, k}\Big)
		\end{aligned}
	\end{equation}
	Die Bedingung ~\ref{form:SimpleMotorGeneticAlgorithmAbsoluteErrorWithSaturationCondition} dient als Fairness-Ausgleich bei der Bestrafung des Fehlers. 
	Im Fall, wo der Regler in die Sättigung läuft, weil eine grosse Änderung der Referenz vorliegt,
	wird der Fehler nicht weiter aufaddiert, da der Regler in diesen Fällen nicht dazu beitragen kann,
	dass die Fläche des Fehlers kleiner wird. Der Motor gibt bereits alles was er kann.
	Durch diese Bedingung konzentriert sich der die Fehlerbewertung auf die Zeiträume in denen der Regler
	den Fehler tatsächlich beeinflussen kann.
	%\centering
    \begin{equation}
		\funcGOfk{0} =
		\begin{cases}
		0 & \text{wenn } ignoreError=1\\
		|\mathFunction{e}{\individuumI,k}| & \text{sonnst}
		\end{cases}
	\end{equation}
}
\vspace{0.1cm}
\horizontalLine
\subsubsection{$\funcGOfk{1}$: Stellwertänderungsrate}
\minipagedOrBelowEachOther
{
	$\funcGOfk{1}$ bewertet grosse Stellwertänderungen als schlecht.
	Ohne dieses Optimierungsziel tendiert der \gls{GA} dazu, den Regler sehr aggresiv zu machen, was 
	sich negativ auf das Stellverhalten auswirkt. Zur Schonung des Motors wird deshalb mit diesem
	Optimierungsziel versucht, grosse Stellwertänderungen zu vermeiden.
	Dank diesem Ziel wird auch automatisch die Rauschunterdrückung des Reglers gefördert.
}{
	\centering
    \begin{equation}
		\funcGOfk{1} = \frac{| \mathFunction{u}{\individuumI,k} - \mathFunction{u}{\individuumI,k-1}|}{T_\mathbf{s}}
	\end{equation}
}

\horizontalLine
\subsubsection{$\funcGOfk{2}$: Positives Überschwingen unterdrücken}
\minipagedOrBelowEachOther
{
	\vspace{0.5cm}
	$\funcGOfk{2}$ fokussiert sich auf das Überschwingen des Systems in die positive Richtung.
	Es wird dabei nur die erste Überschwingung bewertet nach einer Referenzänderung,
	weil es schwierig zu definieren ist, ab wann Signal als nicht mehr schwingend zu betrachten ist.
	Das Kriterium für das Überschwingen ist in~\ref{fig:StepresponseOvershootingCriteriaPlot}
	als Fläche zwischen der Referenz $r$ und der Systemantwort $y$ dargestellt.
	Diese Fläche wird noch normiert mit $u_\mathbf{sat+}$.
}{
	\centering
    
	\begin{figure}[H]
	    \centering
	    \begin{tikzpicture}
\begin{axis}[
    width=8cm,
    height=4cm,
    axis lines = middle,
    xlabel = $t$,
    ylabel = {$ $},
    xmin = 0, xmax = 6,
    ymin = 0, ymax = 1.4,
    grid = both,
    legend pos = south east,
]

% Step signal r(t)
\addplot[
    thick,
    orange,
    name path=ref,
    domain=0:6,
    samples=400,
] {x < 1 ? 0 : 1};
\addlegendentry{$r$}

% Output response y(t)
\addplot[
    thick,
    blue,
    name path=resp,
    samples=300,
    domain=0:6,
] {x < 1 ? 0 : 1 - exp(-(x-1)) * (cos(deg(2.5*(x-1))) + 0.2*sin(deg(2.5*(x-1))))};
\addlegendentry{$y$}

% --- Filled overshoot area (first bump) ---
\addplot[
    blue!40,
    opacity=0.4,
] 
fill between[
    of=resp and ref,
    soft clip={domain=1.72:2.95},
];

\end{axis}
\end{tikzpicture}

	    \caption{Kriterium für das Überschwingen in positive Richtung}
	    \label{fig:StepresponseOvershootingCriteriaPlot}
	\end{figure}


}

\horizontalLine


\subsubsection{$\mathFunction{g}{\individuumI}$: Gesamte Fehlerbewertung}
\minipagedOrBelowEachOther
{
	Die Teilbewertungen werden jeweils mit einem \\
	Gewichtungsfaktor $a_\mathbf{o}$ skaliert. Diese müssen vom Anwender\\
	passend gewählt werden, um das gewünschte Verhalten zu\\
	erzielen.
}{
	\centering
    \begin{equation}
		\mathFunction{g}{\individuumI} = 
		\frac{1}{K} \cdot \sum_{k=0}^{K-1}\Big(
			a_\mathbf{0} \cdot \funcGOfk{0} + 
			a_\mathbf{1} \cdot \funcGOfk{1} +
			a_\mathbf{2} \cdot \funcGOfk{2}
			\Big)
	\end{equation}
}
\newpage


\subsection{Optimierung durchführen}
\begin{figure}[H]
	\center
    \vspace{-0.5cm}
    \includegraphics[width=\linewidth]{images/SimpleMotor/Genetic_SimulationSoftware.png}
    \caption{Software der Systemsimulation für die Optimierung}
    \label{fig:SimpleMotorGenetic_SimulationSoftware}
\end{figure}

\minipagedOrBelowEachOther
{
	Die Optimierung wird 10-mal mit dem \gls{GA} durchgeführt, 
	weil der GA aufgrund seiner stochastischen Natur nicht immer zum gleichen Ergebnis führt.
	Bei der Optimierung wurden die Einstellungen aus der~\ref{tab:SimpleMotorGenetic_OptimizationParameters} verwendet.\\

	Für den \gls{Hyperparameter} der Mutationsstärke $\mathFunction{M}{k}$ wird ein
	abklingender Wert verwendet, um zu Beginn der Optimierung eine grössere Diversität in der Population zu ermöglichen.
	Am Ende der Optimierung soll die Population jedoch weniger divers sein, um eine bessere \gls{Konvergenz} zu ermöglichen.
}
{
	Die Wahl der PID Ausgangs-Sättigung $u_\mathbf{satHigh}$ und $u_\mathbf{satLow}$
	ist aufgrund der realen Spannungsbegrenzung des DC-Motors notwendig.
	Die PID-Integrator-Sättigung $I_{\mathbf{SatHigh}}$ und $I_{\mathbf{SatLow}}$ hingegen
	sind, wie viele andere dieser Hyperparameter, willkürlich gewählt und können auch andere Werte annehmen. 
	Mehr zur Bestimmung der Hyperparameter ist im Kapitel
	~\fullref{sec:objektivesFazitIterativesTuning} beschrieben.
}

\begin{table}[H]
\centering
\begin{tabular}{@{}clrc@{}}
		\toprule
		\textbf{Parameter} & \textbf{Beschreibung} & \textbf{Wert} \\
		\midrule
		$D$                 & Anzahl der einzelnen Optimierungsdurchläufe &10\\[0.5ex]
		$N$                 & Anzahl Individuen in der Population &30\\[0.5ex]
		$G$                 & Anzahl Generationen &5000\\[0.5ex]
		$C$                 & Mutations-Wahrscheinlichkeit &0.05\\[0.5ex]
		$\mathFunction{M}{k}$ & Mutationsstärke &$1 \cdot 0.999^{k}$\\[0.5ex]
		Anti-Windup       & Verwendete Anti-Windup Methode & Clamping \\[0.5ex]
		\hline
		\textbf{Optimierungsziele} &  &  & \\[0.5ex]
		$a_\mathbf{0}$     & Absoluter Regelungsfehler &3.8\\[0.5ex]
		$a_\mathbf{1}$     & Stellwertänderungsrate &0.038\\[0.5ex]
		$a_\mathbf{2}$     & Positives Überschwingen &111\\[0.5ex]
		\hline
		\textbf{Konstante Systemparameter} &  &  & \\[0.5ex]
		$u_\mathbf{satHigh}$ & PID Ausgangs-Sättigung oberer Wert &10\\
		$u_\mathbf{satLow}$  & PID Ausgangs-Sättigung unterer Wert &0\\
		$I_{\mathbf{SatHigh}}$	     & PID-Integrator-Sättigung oberer Wert &10\\
		$I_{\mathbf{SatLow}}$	     & PID-Integrator-Sättigung unterer Wert &-10\\
		\hline
		\textbf{PID-Startparameter} &  &  & \\[0.5ex]
		$Kp$     & Proportionalitätsfaktor &$0 + \mathFunction{randG}{-10,10}*$\\[0.5ex]
		$Ki$     & Integrationsfaktor &$0 + \mathFunction{randG}{-10,10}*$\\[0.5ex]
		$Kd$     & Ableitungsfaktor &$0 + \mathFunction{randG}{-10,10}*$\\[0.5ex]
		$Kn$     & Filterkonstante &$0 + \mathFunction{randG}{-10,10}*$\\[0.5ex]
		\bottomrule
\end{tabular}
\caption{Hyper- \& Tuningparameter für Optimierung mit dem genetischen Algorithmus}
\label{tab:SimpleMotorGenetic_OptimizationParameters}
\end{table}
\[
    *\mathFunction{randG}{a,b} \text{ ist eine gleichverteilte Zufallsvariable im Intervall [a, b] } 
\]
\newpage

%\horizontalLine
\subsection{Optimierung auswerten}


\begin{figure}[H]
	\center
    %\vspace{-1.6cm}
    \includegraphics[width=\linewidth]{images/SimpleMotor/Genetic_Simulation_LearningHistory_1.pdf}
    \caption{Lernverlauf der Optimierung mit dem genetischen Algorithmus bei 10 Durchläufen}
    \label{fig:SimpleMotorGenetic_SimulationLearningHistory_1}
\end{figure}

\minipagedOrBelowEachOther
{
	Die \ref{fig:SimpleMotorGenetic_SimulationLearningHistory_1} zeigt den Lernverlauf
	der 10 Optimierungsdurchläufe mit dem \gls{GA}.
	Ohne den abklingenden Mutations-Parameter $\mathFunction{M}{k}$ würde die Lernkurve eher linear verlaufen.
}
{
	Die hohe Mutationsstärke zu Beginn der Optimierung sorgt dafür, dass sich die Population schneller in Richtung
	eines guten Lösungsbereichs bewegt.
	Am Ende der Optimierung ist die Mutationsstärke klein, was eine feinere Suche im nach einer guten Lösung ermöglicht.
}

\horizontalLine
\subsection{Optimierten Regler am realen System testen}
\begin{figure}[H]
	\center
    %\vspace{-1.6cm}
    \includegraphics[width=\linewidth]{images/SimpleMotor/Genetic_realResult.pdf}
    \caption{Antwort auf Stimulation des optimierten Reglers am DC-Motor, mit dem durch \gls{GA} optimierten PID-Regler}
    \label{fig:SimpleMotorGenetic_realResult}
\end{figure}

\minipagedOrBelowEachOther
{

}
{
	
}

%\subsubsection{Die Optimierung führt nicht immer zu einem guten Ergebnis}


%\begin{figure}[H]
%	\center
%    %\vspace{-1.6cm}
%    \includegraphics[width=\linewidth]{images/MotorSystemGenetic_Simulation_LearningHistory_2.pdf}
%    \caption{Lernverlauf der Optimierung mit dem genetischen Algorithmus führt nicht immer zu einem guten Ergebnis}
%    \label{fig:MotorSystemGenetic_SimulationLearningHistory_2}
%\end{figure}
%
%\begin{figure}[H]
%	\center
%    %\vspace{-1.6cm}
%    \includegraphics[width=\linewidth]{images/MotorSystemGenetic_Simulation_StepResponse_2.pdf}
%    \caption{Sprungantwort der Systeme mit den Reglern + schlecht optimierter Regler}
%    \label{fig:MotorSystemGenetic_SimulationStepResponse_2}
%\end{figure}
