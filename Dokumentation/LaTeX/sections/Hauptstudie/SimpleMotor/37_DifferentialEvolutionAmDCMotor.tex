\section{Differential Evolution Algorithmus am DC-Motor}
\minipagedOrBelowEachOther
{
    Da der \gls{DE} Algorithmus dem \gls{GA} sehr ähnlich ist, lässt sich der \gls{Solver} im C++ Projekt 
    sehr einfach austauschen. Deshalb kann der grösste Teil der Implementierung vom Test des GA wiederverwendet werden und
    wird hier nicht nochmals im Detail beschrieben.
}
{
    Im Gegensatz zum GA wird beim DE kein abklingender \gls{Hyperparameter} für 
    die Mutationsstärke $\mathFunction{M}{k}$ verwendet, da der Algorithmus von sich aus kleinere Mutationen durchführt,
    während die Population konvergiert. Die Funktionsweise des DE Algorithmus ist im Kapitel\\
	\fullref{sec:Description_DifferentialEvolutionAlgorithmus} beschrieben.
}




\subsection{Optimierung durchführen}


\minipagedOrBelowEachOther
{
	Die Software führt die Optimierung 10-mal durch, weil der DE
	aufgrund der zufälligen Startwerte nicht immer zum gleichen Ergebnis führt.
	Bei der Optimierung wurden die Einstellungen aus der \ref{tab:SimpleMotorDifferential_OptimizationParameters} verwendet.
}{

	Wie bereits im Beispiel mit dem GA \ref{fig:SimpleMotorGenetic_SimulationSoftware} werden diverse Hyperparameter
	willkürlich gewählt.
	Mehr zur Bestimmung der Hyperparameter ist im Kapitel\\
	\fullref{sec:objektivesFazitIterativesTuning} beschrieben.
	
}

\begin{table}[H]
\centering
\begin{tabular}{@{}clrc@{}}
		\toprule
		\textbf{Parameter} & \textbf{Beschreibung} & \textbf{Wert} \\
		\midrule
		$D$                 & Anzahl der einzelnen Optimierungsdurchläufe &10\\[0.5ex]
		$N$                 & Anzahl Individuen in der Population &30\\[0.5ex]
		$G$                 & Anzahl Generationen &5000\\[0.5ex]
		$C$                 & Crossover-Wahrscheinlichkeit &0.7\\[0.5ex]
		$M$                 & Mutationsstärke &0.5\\[0.5ex]
		Anti-Windup         & Verwendete Anti-Windup Methode & Clamping \\[0.5ex]
		\hline
		\textbf{Optimierungsziele} &  &  & \\[0.5ex]
		$a_\mathbf{1}$     & Absoluter Regelungsfehler &3.8\\[0.5ex]
		$a_\mathbf{2}$     & Stellwertänderungsrate &0.038\\[0.5ex]
		$a_\mathbf{3}$     & Positives Überschwingen &111\\[0.5ex]
		\hline
		\textbf{Konstante Systemparameter} &  &  & \\[0.5ex]
		$u_\mathbf{satHigh}$ & PID Ausgangs Sättigung oberer Wert &10\\
		$u_\mathbf{satLow}$  & PID Ausgangs Sättigung unterer Wert &0\\
		$I_{\mathbf{SatHigh}}$	     & PID Integrator Sättigung oberer Wert &10\\
		$I_{\mathbf{SatLow}}$	     & PID Integrator Sättigung unterer Wert &-10\\
		\hline
		\textbf{PID-Startparameter} &  &  & \\[0.5ex]
		$Kp$     & Proportionalitätsfaktor &$0 + \mathFunction{randG}{-10,10}*$\\[0.5ex]
		$Ki$     & Integrationsfaktor &$0 + \mathFunction{randG}{-10,10}*$\\[0.5ex]
		$Kd$     & Ableitungsfaktor &$0 + \mathFunction{randG}{-10,10}*$\\[0.5ex]
		$Kn$     & Filterkonstante &$0 + \mathFunction{randG}{-10,10}*$\\[0.5ex]
		\bottomrule
\end{tabular}
\caption{Hyper- \& Tuningparameter für Optimierung mit dem Differential Evolution Algorithmus}
\label{tab:SimpleMotorDifferential_OptimizationParameters}
\end{table}

\[
    *\mathFunction{randG}{a,b} \text{ ist eine gleichverteilte Zufallsvariable im Intervall [a, b] } 
\]

\newpage
%\horizontalLine
\subsection{Optimierung auswerten}


\begin{figure}[H]
	\center
    %\vspace{-1.6cm}
    \includegraphics[width=\linewidth]{images/SimpleMotor/Differential_Simulation_LearningHistory_1.pdf}
    \caption{Lernverlauf der Optimierung mit dem Differential Evolution Algorithmus bei 10 Durchläufen}
    \label{fig:SimpleMotorDifferential_SimulationLearningHistory_1}
\end{figure}

\minipagedOrBelowEachOther
{
    Sehr auffällig ist, dass der DE Algorithmus im Vergleich zum GA
    schöner konvergiert und kein stochastisches Verhalten zeigt.
    Weiter fällt auf, dass von den 10 durchgeführten Optimierungen nur 1 eine noch bessere Lösung gefunden hat.
    Man kann gut sehen, dass wenn sich die Population auf eine gute Lösung einigt, 
    sie nicht mehr aus dem lokalen Minimum herausfindet und somit nicht immer die beste Lösung findet.
}
{

}


\horizontalLine
\subsection{Optimierten Regler am realen System testen}
\begin{figure}[H]
	\center
    %\vspace{-1.6cm}
    \includegraphics[width=\linewidth]{images/SimpleMotor/Differential_realResult.pdf}
    \caption{Antwort auf Stimulation des optimierten Reglers am DC-Motor, mit dem durch DE optimierten PID-Regler}
    \label{fig:SimpleMotorDifferential_realResult}
\end{figure}
\newpage