\subsubsection{Regelkreis für Prozess: DC-Motor}
\label{sec:DCMotorModell}
\begin{figure}[H]
    \centering
    \begin{tikzpicture}[auto, node distance=2cm,>=latex', scale=1]
    \newcommand{\labeloffset}{0.2}
    % Draw grid
    %\draw[step=1cm,gray,very thin] (0,0) grid (10,10);
    
    % Draw blocks
    \sumNode{sum}{1,0};

    \saturationBlockNode{saturation1}{-1, 0}{0}{10};
    \blockNode{controller}{2.5,0}{$G_c(s)$};
    \saturationBlockNode{saturation2}{$(controller.east) + (1.5,0)$}{0}{10};
    \blockNode{plant}{$(saturation2.east) + (2,0)$}{$G(s)=\frac{K1}{T \cdot s + 1}$};

    % Draw connections
    \connect{(-2.5,0)             (saturation1.west)          }{->};
    \connect[label={-0.5,\labeloffset}{$\mathbf{r}$}              ]{(saturation1.east)             (sum.west)          }{->};
    \connect[label={-0.5,\labeloffset}{$\mathbf{e}$}   ]{(sum.east)        (controller.west)   }{->};
    \connect{(controller.east) (saturation2.west)   }{->};
    \connect[label={-0.4,\labeloffset}{$\mathbf{u}$}   ]{(saturation2.east) (plant.west)        }{->};
    \connectionNode{output}{$(plant.east) + (0.5,0)$};
    \connect[label={0.2,0}{$\mathbf{y}$}               ]{(plant.east) \vectorAdd{output}{0.5,0}}{->};
    \connect[label={-6.95,0.7}{$\mathbf{l}$}              ]{(-2.5,1) ($(plant.north)+ (0, 0.5)$) (plant.north)}{->};

    \feedback[sign={-0.3,-0.1}              ]{output.center}{-1}{sum.south}{->};
\end{tikzpicture}
    
  %  \begin{table}[h!]
    \renewcommand{\arraystretch}{1.0}
    \vspace{0.5cm}   % adds 0.5 cm of white space
    \begin{tabular}{@{}llr@{}}
        \toprule
        \textbf{Symbol} & \textbf{Beschreibung} & \textbf{Wert} \\
        \midrule
        $K1$        & DC-Verstärkung                & 1  \\[0.5ex]
        $T$         & Zeitkonstante                 & 0.14 \\[0.5ex]
        \bottomrule
    \end{tabular}
    \caption{Regelkreis für Prozess: DC-Motor}
    %\label{tab:Spannungsfall_Legende}
    %\end{table}
    \label{fig:SimpleMotorTikzDiagram}
\end{figure}

\minipagedOrBelowEachOther
{
  Das Beispielsystem ist in~\ref{fig:SimpleMotorTikzDiagram} dargestellt.
  Es handelt sich dabei um einen einfachen Regelkreis mit einem DC-Motor als Regelstrecke.
  Der Motor wird mit einer Stellgrösse $u$ (Spannung) angesteuert und die Drehgeschwindigkeit $y$
  wird als Spannung gemessen und zurückgeführt. 
  Der Motor wird so gehandhabt als wäre es nicht möglich, ihn in entgegengesetzte Richtung zu betreiben,
  daher kommt die untere Sättigung der Stellgrösse $u$ bei 0V zum Tragen.
  Als Regler wird ein PID-Regler verwendet.
  Der Aufbau des PID-Reglers ist in~\ref{fig:ErweiterterPID} dargestellt.
  Die Last $l$ treibt eine Wirbelstrombremse an, die eine bremsende Wirkung auf den Motor hat.
  Das System wird im Skript von Regelungstechnik 1-2 \cite{regT12Script} genauer beschrieben.
  Im gegensatz zu dem im Skript definierten $K1=0.91$ wurde hier $K1=1$ verwendet,
  da der im Skript verwendete Sprung von 4V für die Systemidentifikation nur für Sprünge von 4V geeignet ist.
  Dies liegt daran, dass die DC-Verstärkung des Motors sehr stark nichtlinear ist. Die Wahl des korrekten $K1$
  ist deshalb nicht sehr kritisch für die Optimierung des Reglers.
}
{
  Das Modell des DC-Motors ist als PT1-Glied modelliert.
  Die Übertragungsfunktion der Regelstrecke lautet:
  \begin{equation}
    \label{eq:DCMotorPT1}
    \mathFunction{G}{s} = \frac{K1}{T \cdot s + 1}
  \end{equation}
  Die Wirbelstrombremse wird im Simulink-Modell abhängig von der Drehzahl des Motors modelliert.
  Da dies ein nichtlineares Verhalten abbildet und die Last bei der Optimierung von \textit{systune} nicht berücksichtigt wird,
  genügt die Übertragungsfunktion in \ref{eq:DCMotorPT1} als Modell für die Optimierung.\\


  Bei den Optimierungen mit dem \gls{GA} und \gls{DE} ist es jedoch möglich, ein nichtlineares Modell zu verwenden,
  weshalb dort die Last berücksichtigt wird. 


}
\minipagedOrBelowEachOther
{
  Der Regelkreis verfügt über zwei Eingänge:
  \begin{itemize}
    \item Sollwert $r$: Gewünschte Drehgeschwindigkeit des Motors
    \item Störgrösse $l$: Externe Störung die auf den Motor wirkt.\\
    Für die Last wird eine Wirbelstrombremse verwendet.
  \end{itemize}
}
{

}

\paragraph{Implementierung im C++ Code}

\begin{cppcode}{Update Funktion des DC-Motors im Code}{lst:DC_MotorDefinitionCPP}
// Update Funktion des DC-Motors im C++ Code. class DCMotorSystem
void update(double deltaTime) override
{
    // Variablen Aliasen für bessere Lesbarkeit in der Formel
	double& y = m_outputAngularVelocity;
	double& u = m_inputVoltage;
	double& l = m_disturbance;
	double& k1 = m_k1;
	double& k2 = m_k2;
	double& k3 = m_k3;
	double& invT = m_invTimeConstant;

    // Signal welches integriert wird
    // invT = 1 / T wird bereits vor der Integration dazu multipliziert, anders als in der Formel oben dargestellt
	double preIntSignal = (k1 * u - y * (1.0 + k2 * k3 * l)) * invT;

    switch (getIntegrationSolver())
	{
		case IntegrationSolver::ForwardEuler:
		{
			m_integratorOutput = getIntegrated_forwardEuler(m_lastPreIntegratorSignal, m_integratorOutput, deltaTime);
			break;
		}
		case IntegrationSolver::BackwardEuler:
		{
			m_integratorOutput = getIntegrated_backwardEuler(preIntSignal, m_integratorOutput, deltaTime);
			break;
		}
		case IntegrationSolver::Bilinear:
		{
			m_integratorOutput = getIntegrated_Bilinear(m_lastPreIntegratorSignal, preIntSignal, m_integratorOutput, deltaTime);
			break;
		}
	}
	m_outputAngularVelocity = m_integratorOutput;
	m_lastPreIntegratorSignal = preIntSignal;
} 
\end{cppcode}


\newpage
\paragraph{Implementierung des DC-Gains im Simulink Modell}
Im Simulink-Modell ist der nichtlineare DC-Gain des Motors so modelliert, 
dass das Modell dem realen Motor möglichst nahe kommt.
Dies ist eigentlich für die Optimierung nicht relevant und diente nur dazu,
die Regler im Simulink-Modell vorgängig zu testen bevor sie am realen Prozess ausprobiert wurden.
Wie die nichtlinear DC-Verstärkung modelliert ist, wird nachfolgend erklärt.

\horizontalLine

\minipagedWithHFillOrBelowEachOther
{



Eine Systemidentifikation mit Sprüngen von 0V - 10V in 1V Schritten wurde durchgeführt,
um unterschiedliche DC-Verstärkungen an den unterschiedlichen Arbeitspunkten zu ermitteln.\\


Die ermittelten DC-Verstärkungen waren linear mit Offset in Abhängigkeit der Eingangsspannung $u$.
Die Geradengleichung wurde mit Hilfe von Linearer Regression ermittelt.
Die Geradengleichung lautet:
\begin{equation}
  \mathFunction{f_\mathbf{2}}{x} = m \cdot x - q
\end{equation}

\begin{figure}[H]
    \centering
    
  %  \begin{table}[h!]
    \renewcommand{\arraystretch}{1.0}
    %\vspace{0.5cm}   % adds 0.5 cm of white space
    \begin{tabular}{@{}cc@{}}
        \toprule
        \textbf{Symbol}  & \textbf{Wert} \\
        \midrule
        $m$       & 1.07  \\[0.5ex]
        $q$         & -0.43   \\[0.5ex]
        \bottomrule
    \end{tabular}
    \caption{Parameter der Geradengleichung}
    %\label{tab:Spannungsfall_Legende}
    %\end{table}
\end{figure}



Im Simulink-Modell wird für $x$ der Ausgangswert des PT1-Glieds eingesetzt.
Das ist möglich, weil für das PT1-Glied eine Verstärkung von 1 verwendet wird.
obwohl sich das $x$ in der Geradengleichung eigentlich auf das Eingangssignal bezieht.\\




}
{
Da der Offset von -0.43 dazu führen würde, dass der Motor auch bei einer Eingangsspannung von 0V eine negative Drehzahl annehmen würde,
muss ein Teil der Geradengleichung durch ein Polynom abgerundet werden.
Der Wechsel der beiden Funktionen wird bei $x=1$ vorgenommen.\\


Um dieses Polynom zu finden wurde ein Gleichungssystem aufgestellt:
\begin{equation}
  \mathFunction{f_\mathbf{1}}{x} = a \cdot x^2 + b \cdot x
\end{equation}
\begin{equation}
  \left|
    \begin{aligned}
      m &= \mathFunctionDot{f_{\mathbf{1}}}{1} \\
      \mathFunction{f_{\mathbf{1}}}{x} &= \mathFunction{f_{\mathbf{2}}}{1}
    \end{aligned}
  \right|
\end{equation}
Mit diesem Gleichungssystem werden die Koeffizienten $a$ und $b$ des Polynoms bestimmt.
Die Lösung des Gleichungssystems ist:
\begin{figure}[H]
    \centering
    
  %  \begin{table}[h!]
    \renewcommand{\arraystretch}{1.0}
    %\vspace{0.5cm}   % adds 0.5 cm of white space
    \begin{tabular}{@{}cc@{}}
        \toprule
        \textbf{Symbol}  & \textbf{Wert} \\
        \midrule
        $a$       & 0.43  \\[0.5ex]
        $b$       & 0.21   \\[0.5ex]
        \bottomrule
    \end{tabular}
    \caption{Parameter des Polynoms}
    %\label{tab:Spannungsfall_Legende}
    %\end{table}
\end{figure}
}



\horizontalLine

\minipagedOrBelowEachOther
{
  \vspace{0.5cm}   % adds 0.5 cm of white space
  Die ermittelten Funktionen $\mathFunction{f_\mathbf{1}}{x}$ und $\mathFunction{f_\mathbf{2}}{x}$
  sind im Blockdiagramm im Simulink-Modell umgesetzt und sorgen für eine relistischere Simulation des DC-Motors.
  Die Kombination der beiden Funktionen ist in~\ref{fig:SimpleMotorTikzDiagram_nonlinearDCGain} dargestellt,
  wobei $x$ der Ausgang des PT1-Glieds ist, welcher ohne diese Anpassung, der Winkelgeschwindigkeit des Motors entsprechen würde.
  Der neue Wert $y$ wird dann als Drehgeschwindigkeit des Motors verwendet und an den Ausgang des Simulink-Modells weitergegeben. 
}
{
  \begin{figure}[H]
    \centering
    \includegraphics[width=\linewidth]{images/SimpleMotor/DcGainPlot.pdf}
    %\begin{tikzpicture}[scale=2]
    % Draw axes
    \draw[thick,->] (0,0) -- (3,0) node[right] {$x$};
    \draw[thick,->] (0,0) -- (0,3) node[above] {$y$};
    
    % Draw the first part of the function (x < 1): y = 0.43*x^2 + 0.21*x
    % Using cyan color for segment 1
    \draw[cyan, thick, domain=0:1, samples=50] 
        plot (\x, {0.43*\x*\x + 0.21*\x});
    
    % Draw the second part of the function (x >= 1): y = 1.07*x - 0.43
    % Using orange color for segment 2
    \draw[orange, thick, domain=1:2.8, samples=50] 
        plot (\x, {1.07*\x - 0.43});
    
    % Mark the transition point at x = 1
    \fill (1, 0.64) circle (0.5pt);
    
    % Add labels for the segments
    \node[cyan] at (0.5, 0.4) {\textbf{1}};
    \node[orange] at (2, 1.9) {\textbf{2}};
    
    % Add tick mark at x = 1
    \draw (1, -0.05) -- (1, 0.05);
    \node[below] at (1, -0.05) {\small $1$};
\end{tikzpicture}
    \caption{Annäherung der DC-Verstärkung des Motors im Simulink-Modell}
    \label{fig:SimpleMotorTikzDiagram_nonlinearDCGain}
  \end{figure}
}


%\dot{}

