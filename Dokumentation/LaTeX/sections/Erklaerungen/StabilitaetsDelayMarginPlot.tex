\subsubsection{Totzeit- \& Verstärkungsreserve Diagramm}
\label{sec:StabilitaetsDelayMarginPlot}

\minipagedOrBelowEachOther
{
    Das Totzeit- \& Verstärkungsreserve Diagramm in~\ref{fig:stabilitaetsDelayMarginPlot} 
    zeigt die Verstärkungs- und Totzeitreserven des optimierten geschlossenen Systems.
    Der Rote Bereich stellt die instabilen Verstärkungs- und Totzeit- Paare dar.
    Der Grüne Bereich stellt die stabilen Verstärkungs- und Totzeit- Paare dar.
    Das Diagramm gibt an, wie viel Verstärkung und Totzeit dem open loop System $G$ hinzugefügt werden kann,
    bis das closed loop System instabil wird.\\


    Ermittelt wird das Diagramm, indem das zu testende System $G$ mit einem komplexen Faktor $K$ multipliziert wird
    und anschliessend geprüft wird, ob sich ein Pol des geschlossenen Systems in der rechten Halbebene befindet.
    \[
        G_{test}(s) = \frac{K \cdot G(s)}{1 + K \cdot G(s)}
    \]
    \[
        K = (r \cdot e^{-s \cdot T}) \quad \text{mit} \quad r \in (0, \infty), T \in [0, \infty)
    \]



    \begin{figure}[H]
        \centering
        \begin{tikzpicture}[auto, node distance=2cm,>=latex', scale=1]
    \newcommand{\labeloffset}{0.2}
    % Draw grid
    %\draw[step=1cm,gray,very thin] (0,0) grid (10,10);
    
    % Draw blocks
    \sumNode{sum}{1,0};
    \blockNode{kGain}{2.5,0}{$K$};
    \blockNode{plant}{$(kGain.east) + (1,0)$}{$G(s)$};

    % Draw connections
    \connect{(0,0)             (sum.west)          }{->};
    \connect{(sum.east)        (kGain.west)   }{->};
    \connect{(kGain.east) (plant.west)   }{->};
    \connectionNode{output}{$(plant.east) + (0.5,0)$};
    \connect{(plant.east) (output.center)   ($(output.center) + (0.5,0)$)}{->};
    
    \feedback[sign={-0.3,-0.1}              ]{output.center}{-1}{sum.south}{->};
\end{tikzpicture}
        \caption{Testsystem für Totzeit- \& Verstärkungsreserve Diagramm}
        \label{fig:stabilitaetsDelayMarginPlotSystem}
    \end{figure}
    Für jeden möglichen Wert von $K$ wird die Stabilität des geschlossenen Systems überprüft.
    Wenn es stabil ist, wird der Punkt $K$ im Diagramm mit grüner Farbe markiert, 
    andernfalls mit rot.
    Weil es natürlich nicht möglich ist alle möglichen Werte von $K$ zu testen,
    wird der Bereich in einem Raster getestet.
    Damit ein möglichst grosser Bereich abgedeckt und dargestellt werden kann, wird die Verstärkung
    in dB dargestellt.
}{
    %\centering
    \begin{figure}[H]
        \centering
        \includegraphics[width=\linewidth]{images/MotorMitSchwungmasse/GeneticStabilityDelayMarginsPlot.pdf}
        \caption{Totzeit- \& Verstärkungsreserve Diagramm für System $G$}
        \label{fig:stabilitaetsDelayMarginPlot}
    \end{figure}
}

\paragraph{Wofür ist dieses Diagramm nun nützlich?}
\minipagedOrBelowEachOther
{
    Wird der Vorwärtspfad des Regelkreises mit diesem Diagramm analysiert,
    lässt sich ablesen, wie viel Verstärkung und Totzeit das System verträgt.

    In der Praxis bietet besonders dieses Diagramm mit der Totzeitinformation einen grossen Vorteil
    gegenüber vielen anderen Robustheitsanalysen.
    Eine Totzeit schleicht sich in realen Systemen sehr schnell ein und kann sich je nach System
    stark auf die Stabilität auswirken.
}
{
    Dieses Diagramm erleichtert es, im Designprozess bereits frühzeitig zu erkennen,
    ob die realen Zeitverzögerungen zu Stabilitätsproblemen führen können oder nicht.
    Gegebenenfalls kann die benötigte Totzeitreserve als Anforderung in den Optimierungsprozess
    mit einbezogen werden.
}