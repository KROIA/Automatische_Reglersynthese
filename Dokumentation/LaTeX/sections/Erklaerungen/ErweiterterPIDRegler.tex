\subsubsection{Reglerstruktur: Erweiterter PID-Regler}
\label{sec:ErweiterterPIDRegler}
\minipagedOrBelowEachOther
{
    Für die Tests der unterschiedlichen Methoden wird auf ein Simulink Modell mit der in~\fullref{fig:SimpleMotorTikzDiagram} 
    dargestellten Regelstrecke zurückgegriffen. Der darin verwendete PID-Regler Block beinhaltet nicht nur die üblich bekannten
    $Ki$, $Kp$ und $Kd$ Anteile, sondern auch noch diverse weitere Einstellmöglichkeiten und Parameter.
    Einige dieser zusätzlichen PID-Regler Erweiterungen sind in~\fullref{fig:ErweiterterPID} dargestellt.
    Für die Tests werden die unterschiedlichen Einstellungen so gut es geht berücksichtigt. 
    Der Grund warum gleich mehrere Einstellungsmöglichkeiten getestet werden ist, dass einige Einstellungen zu einer
    grösseren Anzahl an Parametern führen, welche optimiert werden müssen.
}
{
    Dies ermöglicht es, ein besseres Ergebnis zu erzielen und 
    gleichzeitig zeigt es auch die Probleme der getesteten Methoden besser auf.
    Der PID-Regler wird im diskreten Zeitbereich angewendet um Faire Bedingungen der unterschiedlichen Methoden zu gewährleisten.
    Deshalb sind die dargestellten Übertragungsfunktionen auch im Z-Transformationsbereich angegeben.

    \vspace{0.2cm}
    Der PID-Block von Simulink hat jedoch für den Anti-Windup $Clamping$ Mechanismus 
    ein falsches Verhalten, dies ist in den Testresultaten ersichtlich.
    Mehr dazu kann im~\fullref{sec:Description_SimulinkPID_ClampingLogic_Problem} nachgelesen werden.
}


\begin{figure}[H]
    \centering
    \begin{tikzpicture}[auto, node distance=2cm,>=latex', scale=1]
    %\begin{scope}[shift={(0,5)}]   
        \newcommand{\labeloffset}{0.2}
        \definecolor{lightGrayColor}{rgb}{0.6,0.6,0.6};

        
        \newcommand{\varPIDTs}{Ts}
        \newcommand{\varPIDKb}{Kb}
        \newcommand{\varPIDKp}{Kp}
        \newcommand{\varPIDKi}{Ki}
        \newcommand{\varPIDKd}{Kd}
        \newcommand{\varPIDKn}{Kn}
        \newcommand{\varPIDUSatLow}{U_{\mathrm{SatLow}}}
        \newcommand{\varPIDUSatHigh}{U_{\mathrm{SatHigh}}}
        \newcommand{\varPIDISatLow}{I_{\mathrm{SatLow}}}
        \newcommand{\varPIDISatHigh}{I_{\mathrm{SatHigh}}}
        \newcommand{\varPIDZ}{z}
        \newcommand{\varPIDupreSat}{u_{\mathrm{preSat}}}
        \newcommand{\varPIDu}{u}
        \newcommand{\varPIDudOne}{u_{\mathrm{d1}}}
        \newcommand{\varPIDudTwo}{u_{\mathrm{d2}}}
        \newcommand{\varPIDud}{u_{\mathrm{d}}}
        \newcommand{\varPIDup}{u_{\mathrm{p}}}
        \newcommand{\varPIDui}{u_{\mathrm{i}}}

        \newcommand{\formulaDNoFilter}{\varPIDKd \cdot \frac{\varPIDZ-1}{\varPIDTs \cdot \varPIDZ}}
        \newcommand{\formulaDWithFilter}{\varPIDKd \cdot \frac{\varPIDKn \cdot (\varPIDZ-1)}{\varPIDZ-1 + \varPIDKn \cdot \varPIDTs}}
        \newcommand{\formulaIntegrator}{\frac{\varPIDTs}{\varPIDZ-1}}

        %\drawGrid{16}{-15};

        % Input/Output nodes
        \connectionNode{input}{1, -5};
        \connect[label={-0.5,\labeloffset}{$e$}]{(0,-5) (input.center)}{-};
        % Output summation
        \sumNode{outputSum}{12,-5};
        \saturationBlockNode{outputSaturationNode}{$(outputSum.center) + (2,0)$}{$\varPIDUSatLow$}{$\varPIDUSatHigh$};
        \connectionNode{outSumNode}{$(outputSum.center) + (1,0)$};
        \connectionNode{outSatNode}{$(outputSaturationNode.center) + (1,0)$};
        \connect[label={-0.5,\labeloffset+0.1}{$\varPIDupreSat$}]{(outputSum.east) (outSumNode) (outputSaturationNode.west)}{->};
        \connect[label={-0.5,\labeloffset}{$\varPIDu$}]{(outputSaturationNode.east) (outSatNode) (16, -5)}{->};
        \sumNode{outputSatSubtract}{14,-7};
        \connect[sign={-0.1,\labeloffset}]{(outSumNode) (13,-7) (outputSatSubtract.west)}{->};
        \connect{(outSatNode) (15,-7) (outputSatSubtract.east)}{->};
        
        


        % Differenzierer Teil
        \node[font=\small, lightGrayColor] at (3.2,0) {Ableitungstyp};
        \connect{(1.5,-1.8) (1.5,0) (2,0)}{-, dashed, lightGrayColor};

        \switchTwo[east]{1,-2};
        \connect{(input.center) (1,-2)}{-};

        % Ableitung ohne filter
        \blockNode{diffNoFilter}{9,-1}{$\formulaDNoFilter$};
        \connect[label={-1.3,\labeloffset}{Ohne Filter}]{(2,-1.5) (2,-1) (diffNoFilter.west)}{->};

        % Ableitung mit filter
        \blockNode{diffWithFilter}{9,-3}{$\formulaDWithFilter$};
        \connect[label={-1,\labeloffset}{Mit Filter}]{(2,-2.5) (2,-3) (diffWithFilter.west)}{->};

        % Zusammenführung
        \sumNode{differentialSum}{11,-2};
        \connect[label={0.3,1}{$\varPIDudOne$}]{(diffNoFilter.east) (11,-1) (differentialSum.north)}{->};
        \connect[label={0.3,-1}{$\varPIDudTwo$}]{(diffWithFilter.east) (11,-3) (differentialSum.south)}{->};
        \connect[label={-0.3,0.5}{$\varPIDud$}]{(differentialSum.east) (12, -2) (outputSum.north)}{->};



        % kP Teil
        \gainNode{kpBlock}{9,-5}{$\varPIDKp$};
        \connect{(input.center) (kpBlock.west)}{->};
        \connect[label={-1,\labeloffset}{$\varPIDup$}]{(kpBlock.east) (outputSum.west)}{->};


        % kI Teil
        \switchThree[west]{8,-8};
        \node[font=\small, lightGrayColor] at (9.5,-9.5) {Anti-Windup Typ};
        \connect{(7.5, -7.85) (7.5,-9.5) (8,-9.5)}{-, dashed, lightGrayColor};
        \connectionNode{input1}{1,-7.5};
        \connect[label={-1.5, \labeloffset}{Ohne}]{(input1.center) (7,-7.5)}{-};

        \connectionNode{input2}{1,-8};
        \connect{(input2.center) (2,-8)}{-};
        \switchNC{2,-8};
        \connect[label={-1.5,\labeloffset}{Clamping}]{(3,-8) (7,-8)}{-};
        \node[draw, rectangle, minimum height=1cm, minimum width=1.2cm] at (2.5,-9.5) (clampingController) {
            \begin{tabular}
                {c}
                \small Clamping \\
                \small Controller
                \
            \end{tabular}
        };
        \connect{(clampingController.north) (2.5, -7.9) }{-, dashed, lightGrayColor};
        \connectionNode{input3}{2,-10.5};
        \connect{(input3.center) \vectorAdd{clampingController.south}{-0.5,0}}{->};
        \connectionNode{outSatDiv}{7,-12};
        \connect{(outSatDiv.center) (2.5,-12) (clampingController.south)}{->};

        \sumNode{backCalcSum}{4, -10.5};
        \connect[label={-1.5,\labeloffset}{Back Calculation}]{(backCalcSum.north) (4, -8.5) (7, -8.5)}{-};
        \gainNode[south]{kiBlock}{1, -6}{$\varPIDKi$};
        \connect{(input.center) (kiBlock.north)}{->};
        \connect{(kiBlock.south) (input1.center) (input2.center) (1, -10.5) (input3.center) (backCalcSum.west)}{->};

        \gainNode[west]{backCalcGain}{6, -10.5}{$\varPIDKb$};
        \connect{(backCalcGain.west) (backCalcSum.east)}{->};
        \connect{(outSatDiv.center) (7, -10.5) (backCalcGain.east)}{->};
        \connect{(outputSatSubtract.south) (14, -12) (outSatDiv.center)}{-};

        % Integrator
        \blockNode{integrator}{9,-8}{$\formulaIntegrator$};
        \connect{(8, -8) (integrator.west)}{->};
        \saturationBlockNode{iSaturationNode}{$(integrator.east) + (1,0)$}{$\varPIDISatLow$}{$\varPIDISatHigh$};
        \connect{(integrator.east) (iSaturationNode.west)}{->};
        \connect[label={0.75, 0.5+\labeloffset}{I-Begrenzung}]{(iSaturationNode.north) ($(iSaturationNode.center) + (0,1)$) ($(integrator.center) + (0,1)$) (integrator.north)}{->};
        \connectionNode{uiNode}{12, -8}
        \connect[label={-0.3,-0.5}{$\varPIDui$}]{(iSaturationNode.east) (uiNode.center) (outputSum.south)}{->};
        \connect{(uiNode.center) (12,-11.5) (3,-11.5) \vectorAdd{clampingController.south}{0.5,0}}{->};
        
    

        
        

\end{tikzpicture}
    \caption{Aufbau des Erweiterten PID-Reglers}
    \label{fig:ErweiterterPID}
\end{figure}

Aus allgemeiner Sicht auf den PID-Regler wird bei der Erweiterung für den Ausgang $u$ eine Sättigung eingeführt.
Eine Sättigung an dieser Stelle ist sinnvoll, da in realen Systemen die Stellgrösse meist physikalischen Grenzen unterliegt und 
die Sättigung somit das reale Verhalten besser abbildet.

\newpage
%\horizontalLine
\paragraph{P-Anteil}
Der Proportionalanteil verhält sich wie in einem klassischen PID-Regler.

\horizontalLine
\paragraph{I-Anteil}
Aufgrund der Sättigung am Ausgang des PID-Reglers kann es zu einem sogenannten \textit{\gls{Integral Windup}} kommen.
Der Integralanteil verfügt über 3 Anti-Windup Einstellungen:
\begin{itemize}
    \item Kein Anti-Windup
    \item Back-Calculation
    \item Clamping
\end{itemize}


\label{par:Clamping}
\label{par:Back-Calculation}
\begin{figure}[H]
{
    \vspace{-0.2cm}
    %\centering
    \begin{minipage}[t]{0.48\textwidth}
        \subparagraph{Back-Calculation}%\mbox{}\\[0.5em]
        Bei der Back-Calculation Methode wird die Differenz zwischen dem gesättigten Ausgang und dem ungesättigten Ausgang
        berechnet. Diese Differenz wird mit einem einstellbaren Faktor $Kb$ multipliziert und vom zu integrierenden Signal subtrahiert.
        Dies bewirkt, dass der Integrator Eingang sinkt, wenn der Ausgang gesättigt ist
        und somit der Integralanteil nicht beliebig weiter ansteigen soll.
        $Kb$ ist dabei ein zusätzlicher zu optimierender Parameter.
    \end{minipage}
    \hfill
    \vrule width 0.5pt % vertical line
    \hfill
    \begin{minipage}[t]{0.48\textwidth}
        \subparagraph{Clamping}%\mbox{}\\[0.5em]
        Bei der Clamping Methode wird der Integrator Eingang komplett auf null gesetzt, 
        wenn $\boxed{(\mathrm{sgn}(e \cdot K_\mathrm{i})=\mathrm{sgn}(u_\mathrm{i})) \wedge (u-u_{\mathrm{preSat}} \neq 0)}$,
        wobei \gls{sgn} der Signum-Operator ist.
        Diese Logik ist in~\fullref{fig:ErweiterterPID} als \textit{Clamping Controller} dargestellt.



        Neben dem Anti-Windup Mechanismus kann der Integrator auch noch mit einer Sättigung versehen werden, 
        welche über die Parameter $I_\mathrm{SatLow}$ und $I_\mathrm{SatHigh}$ eingestellt wird.
        Die dargestellte \textit{I-Begrenzung's} Rückführung symbolisiert, dass der interne Zustandswert durch die Sättigung überschrieben wird.
        Der Sättigungsblock bezeichnet also die Sättigung des internen Integrator Zustandswertes und nicht die Sättigung am Ausgang des Integrators.
        Wenn die Sättigung des Integrators aktiv ist, können die Parameter $I_\mathrm{SatLow}$ und $I_\mathrm{SatHigh}$ ebenfalls optimiert werden.
        Für die Optimierung werden jedoch nicht beide Parameter individuell optimiert. Es wird stattdessen nur ein Wert positiv 
        trainiert für die obere Sättigung $I_\mathrm{SatHigh}$. Der untere Sättigungswert $I_\mathrm{SatLow}$ wird dann automatisch als negativer Wert von $I_\mathrm{SatHigh}$ gesetzt.
        Dadurch wird die Sättigung symmetrisch um Null gehalten und die Anzahl der zu optimierenden Parameter reduziert.
    \end{minipage}
}
\end{figure}


\horizontalLine
\paragraph{D-Anteil}

\minipagedOrBelowEachOther
{
    Der in Simulink zur Verfügung stehende PID-Regler Block bietet für den diskreten D-Anteil zwei unterschiedliche Implementationen an.
}
{

}

\minipagedOrBelowEachOther
{
    Ohne Filterung wird der D-Anteil implementiert wie in~\ref{form:formulaPIDWithoutDFilter} dargestellt.
}
{
    \formulaWithDerivationShow{formulaPIDWithoutDFilter}
}


\minipagedOrBelowEachOther
{
    Mit Filterung wird der D-Anteil implementiert wie in~\ref{form:formulaPIDWithDFilter} dargestellt.
    Bei der Verwendung des D-Anteils mit Filterung kommt der zusätzliche Parameter $Kn$ hinzu, 
    welcher ebenfalls optimiert werden muss.
}
{
    \formulaWithDerivationShow{formulaPIDWithDFilter}
}



%\begin{itemize}
%    \item Ohne Filterung wird der D-Anteil implementiert als: \\
%      \formulaWithDerivationShow{formulaPIDWithoutDFilter}
%    \item Mit Filterung wird der D-Anteil implementiert als: \\
%      \formulaWithDerivationShow{formulaPIDWithDFilter}
%\end{itemize}

\paragraph{Implementierung im C++ Code}
Bei der Anti-Windup Methode \textit{Clamping} wird im Code nicht die gleiche Implementierung wie in Simulink verwendet.
Mehr Informationen dazu sind im Kapitel~\fullref{sec:Description_SimulinkPID_ClampingLogic_Problem} zu finden.
Nachfolgend wird lediglich die Update Funktion des PID-Reglers im C++ Code dargestellt.

\begin{cppcode}{Update Funktion des PID-Reglers im Code}{lst:PID_ReglerDefinition}
// Update Funktion des PID-Reglers im C++ Code. class PID
void PID::update(double deltaTime)
{
	double proportional = m_kp * m_inputValue;
	double derivative = 0;

    // D Anteil berechnen
	switch (getDerivativeType())
	{
		case DerivativeType::Unfiltered:
		{
			switch (getDifferentiationSolver())
			{
				case DifferentiationSolver::BackwardEuler:
				{
					derivative = TimeBasedSystem::getDifferentiated_backwardEuler(m_lastInputValue, m_inputValue, deltaTime) * m_kd;
					break;
				}
			}
			break;
		}
		case DerivativeType::Filtered:
		{
			derivative = m_kd * m_kn * (m_inputValue - m_lastInputValue) - m_lastDerivativeValue * (m_kn * deltaTime - 1);
			break;
		}
	}

	double toIntegrateSignal     = m_inputValue     * m_ki;
	double toIntegrateLastSignal = m_lastInputValue * m_ki;

    // Anti-Windup Methode anwenden
	switch (getAntiWindupMethod())
	{
		case AntiWindupMethod::None:
			break;
		case AntiWindupMethod::Clamping:
		{
			bool integralSignEqual = (m_integral > 0 && toIntegrateSignal > 0) || (m_integral < 0 && toIntegrateSignal < 0);
			if ((m_outputPositiveSaturated || m_outputNegativeSaturated) && integralSignEqual)
			{
				// Integration überspringen, wenn der Regler-Ausgang gesättigt ist und 
                // das Integrator-Eingang das gleiche Vorzeichen wie der Integrator-Ausgang hat
				goto afterIntegration;
			}
			break;
		}
		case AntiWindupMethod::BackCalculation:
		{
			double antiWindupSignal = (m_outputValue - m_outputValueBeforeSaturation) * m_backCalculationConstant;
			toIntegrateSignal += antiWindupSignal;
			toIntegrateLastSignal += antiWindupSignal;
			break;
		}
	}

    // Integration durchführen
	switch (getIntegrationSolver())
	{
		case IntegrationSolver::ForwardEuler:
		{
			m_integral = TimeBasedSystem::getIntegrated_forwardEuler(toIntegrateSignal, m_integral, deltaTime);
			break;
		}
		case IntegrationSolver::BackwardEuler:
		{
			m_integral = TimeBasedSystem::getIntegrated_backwardEuler(toIntegrateLastSignal, m_integral, deltaTime);
			break;
		}
		case IntegrationSolver::Bilinear:
		{
			m_integral = TimeBasedSystem::getIntegrated_Bilinear(toIntegrateLastSignal, toIntegrateSignal, m_integral, deltaTime);
			break;
		}
	}
	afterIntegration: // Goto Sprungziel für Clamping Anti-Windup Methode
	

	// Integrator Sättigung anwenden
	if (m_integral < -m_iSaturationLimit)
	{
		m_integral = -m_iSaturationLimit;
	}
	else if (m_integral > m_iSaturationLimit)
	{
		m_integral = m_iSaturationLimit;
	}

    // Regler-Ausgang vor Sättigung berechnen
	m_outputValueBeforeSaturation = proportional + m_integral + derivative;

	// Regler-Ausgangssättigung anwenden
	m_outputPositiveSaturated = false;
	m_outputNegativeSaturated = false;
	if (m_outputValueBeforeSaturation < m_outputSaturationLimitLower)
	{
		m_outputValue = m_outputSaturationLimitLower;
		m_outputNegativeSaturated = true;
	}
	else if (m_outputValueBeforeSaturation > m_outputSaturationLimitUpper)
	{
		m_outputValue = m_outputSaturationLimitUpper;
		m_outputPositiveSaturated = true;
	}
	else
	{
		m_outputValue = m_outputValueBeforeSaturation;
	}

	m_lastInputValue = m_inputValue;
	m_lastDerivativeValue = derivative;
}
\end{cppcode}




\newpage
\paragraph{Zusammenfassung}
\begin{figure}[H]
{
    \vspace{-0.2cm}
    %\centering
    \begin{minipage}[t]{0.3\textwidth}
        %\vspace{0.8cm}
        Der verwendete PID-Regler bietet die Möglichkeit unterschiedliche Einstellungen zu testen und zu vergleichen.
        In der Tabelle sind die möglichen Optimierungsparameter des PID-Reglers zusammengefasst.

    \end{minipage}\hfill
    \begin{minipage}[t]{0.67\textwidth}
    \vspace{-0.5cm}
    \centering
    % Tabelle der PID-Regler Parameter
    \begin{table}[H]
        %\centering
        \begin{tabular}{|c|c|c|}
            \hline
            \textbf{Parameter} & \textbf{Beschreibung} & \textbf{Parameter vorhanden, wenn:} \\
            \hline
            $Kp$ & Proportionalverstärkung & immer \\
            \hline
            $Ki$ & Integralverstärkung & immer \\
            \hline
            $Kd$ & Differentialverstärkung & immer \\
            \hline
            $Kn$ & D-Anteil Filterkonstante & Ableitungstyp = Mit Filter \\
            \hline
            $U_{\mathrm{SatHigh}}$ & Obere Ausgangssättigung & Ausgangssättigung aktiviert \\
            \hline
            $U_{\mathrm{SatLow}}$ & Untere Ausgangssättigung & Ausgangssättigung aktiviert \\
            \hline
            $Kb$ & Anti-Windup Verstärkung & Anti-Windup Typ = Back-Calculation \\
            \hline
            $I_{\mathrm{SatHigh}}/I_{\mathrm{SatLow}}$ & $\pm$ Integrator Sättigung  & aktiviert \\
            \hline
        \end{tabular}
        \caption{Übersicht der PID-Regler Parameter}
        \label{tab:PIDParameter}
    \end{table}
    \end{minipage}
}
\end{figure}
