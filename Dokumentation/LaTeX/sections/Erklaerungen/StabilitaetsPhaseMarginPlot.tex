\subsubsection{Phasen- \& Verstärkungsreserve Diagramm}
\label{sec:StabilitaetsPhaseMarginPlot}

\minipagedOrBelowEachOther
{
    Das Phasen- \& Verstärkungsreserve Diagramm \cite{matlabScriptStabilityRegionPlot} in 
    \ref{fig:stabilitaetsPhaseMarginPlot} zeigt die Verstärkungs- und Phasenreserven 
    des optimierten geschlossenen Systems in logarithmischer Darstellung.
    Der Rote Bereich stellt die instabilen Verstärkungs- und Phasen- Paare dar.
    Der Grüne Bereich stellt die stabilen Verstärkungs- und Phasen- Paare dar.
    Das Diagramm gibt an, wie viel Verstärkung und Phase dem open loop System $G$ hinzugefügt werden kann,
    bis das closed loop System instabil wird.\\


    Ermittelt wird das Diagramm, indem das zu testende System $G$ mit einem komplexen Faktor $K$ multipliziert wird
    und anschliessend geprüft wird, ob sich ein Pol des geschlossenen Systems in der rechten Halbebene befindet.
    \[
        G_{test}(s) = \frac{K \cdot G(s)}{1 + K \cdot G(s)}
    \]
    \[
        K = (r \cdot e^{j \cdot \phi}) \quad \text{mit} \quad r \in (0, \infty), \phi \in [-\pi, \pi)
    \]



    \begin{figure}[H]
        \centering
        \begin{tikzpicture}[auto, node distance=2cm,>=latex', scale=1]
    \newcommand{\labeloffset}{0.2}
    % Draw grid
    %\draw[step=1cm,gray,very thin] (0,0) grid (10,10);
    
    % Draw blocks
    \sumNode{sum}{1,0};
    \blockNode{kGain}{2.5,0}{$K$};
    \blockNode{plant}{$(kGain.east) + (1,0)$}{$G(s)$};

    % Draw connections
    \connect{(0,0)             (sum.west)          }{->};
    \connect{(sum.east)        (kGain.west)   }{->};
    \connect{(kGain.east) (plant.west)   }{->};
    \connectionNode{output}{$(plant.east) + (0.5,0)$};
    \connect{(plant.east) (output.center)   ($(output.center) + (0.5,0)$)}{->};
    
    \feedback[sign={-0.3,-0.1}              ]{output.center}{-1}{sum.south}{->};
\end{tikzpicture}
        \caption{Testsystem für Phasen- \& Verstärkungsreserve Diagramm}
        \label{fig:stabilitaetsMarginPlotSystem}
    \end{figure}
    Für jeden möglichen Wert von $K$ wird die Stabilität des geschlossenen Systems überprüft.
    Wenn es stabil ist, wird der Punkt $K$ in der komplexen Ebene mit grüner Farbe markiert, 
    andernfalls mit rot.
    Weil es natürlich nicht möglich ist alle möglichen Werte von $K$ zu testen,
    wird der Bereich in einem Raster getestet.
    Damit ein möglichst grosser Bereich abgedeckt und dargestellt werden kann, wird die komplexe Ebene 
    für beide Achsen logarithmisch skaliert.\\


    
    
}
{
    %\centering
    \begin{figure}[H]
        \centering
        \includegraphics[width=\linewidth]{images/MotorMitSchwungmasse/GeneticStabilityPhaseMarginsPlot.pdf}
        \caption{Phasen- \& Verstärkungsreserve Diagramm für System $G$}
        \label{fig:stabilitaetsPhaseMarginPlot}
    \end{figure}
}

