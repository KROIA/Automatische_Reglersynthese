\subsubsection{Regelkreis für Prozess: Motor-mit-Schwungmasse}
\label{sec:MotorMitSchwungmasseModellidentifikation}
\begin{figure}[H]
    \centering
    \begin{tikzpicture}[auto, node distance=2cm,>=latex', scale=1]
    \newcommand{\labeloffset}{0.2}
    % Draw grid
    %\draw[step=1cm,gray,very thin] (0,0) grid (10,10);
    
    % Draw blocks
    \sumNode{sum}{1,0};

    %\saturationBlockNode{saturation1}{-1, 0}{-10}{10};
    \saturationBlockNode{saturation3}{4.5, 2}{-10}{10};
    \blockNode{controller}{2.5,0}{$PID$};
    \saturationBlockNode{saturation2}{4.5,0}{-10}{10};
    \gainNode[east]{u1gain}{$(saturation2.east) + (1,0)$}{300};
    \gainNode[east]{u2gain}{$(saturation3.east) + (1,0)$}{300};
    \blockNode{plant}{$(u1gain.east) + (2,-0.25)$}{M=$|$w$|$=M};

    % Draw connections
    %\connect{(-2.5,0)             (saturation1.west)          }{->};
    \connect[label={-3.6,\labeloffset}{$\mathbf{d}$}              ]{(0,2)             (saturation3.west)          }{->};
    \connect[label={-0.5,\labeloffset}{$\mathbf{r}$}              ]{(0,0)             (sum.west)          }{->};
    \connect[label={-0.5,\labeloffset}{$\mathbf{e}$}   ]{(sum.east)        (controller.west)   }{->};
    \connect{(controller.east) (saturation2.west)   }{->};
    
    \connect{(saturation2.east) (u1gain.west)        }{->};
    \connect[label={-0.25, \labeloffset}{$\mathbf{u_{1}}$}   ]{(u1gain.east) ($(plant.west) + (0, 0.25)$)        }{->};
    \connectionNode{output}{$(plant.east) + (0.5,-0.25)$};
    \connect[label={0.2,0}{$\mathbf{y_{2}}$}               ]{\vectorAdd{plant.east}{0,-0.25} (output.center) \vectorAdd{output}{0.5,0}}{->};
    \connect[label={0.2,0}{$\mathbf{y_{1}}$}               ]{\vectorAdd{plant.east}{0,0.25} \vectorAdd{plant.east}{1,0.25}}{->};
    \connect{(saturation3.east) (u2gain.west)}{->};
    \connect[label={-0.25, -\labeloffset}{$\mathbf{u_{2}}$}]{(u2gain.east) ($(plant.west) + (-0.6, 2.25)$) ($(plant.west) + (-0.6, -0.25)$) ($(plant.west) + (0, -0.25)$)}{->};

    \feedback[sign={-0.3,-0.1}              ]{output.center}{-1}{sum.south}{->};
\end{tikzpicture}
    
  %  \begin{table}[h!]
    \caption{Regelkreis des Prozesses Motor mit Schwungmasse}
    %\label{tab:Spannungsfall_Legende}
    %\end{table}
    \label{fig:MotorMitSchwungmasseRegelkreisTikzDiagram}
\end{figure}

\minipagedOrBelowEachOther
{
    Der Prozess besteht aus zwei Motoren, welche auf einer federgekoppelten Welle sitzen.
    Auf den Achsen der Motoren befindet sich jeweils eine Schwungmasse, welche die Trägheit des Prozesses erhöht.
    Beide Motoren besitzen Encoder, welche die Winkelposition der Motorachsen messen.
    Der Prozess ist darauf ausgelegt, mit dem Motor1 $(u_1)$ einen bestimmten Winkel anzufahren.
    Der Motor am Eingang 2 $(u_2)$ dient zur Störung des Prozesses.
    Die Motoren sind unterschiedlich und verhalten sich deshalb nicht ganz gleich.
    Der Prozess beinhaltet somit zwei Eingänge und zwei Ausgänge.
    Für die Modellbildung werden folgende Schritte durchgeführt.
}
{
    \begin{itemize}
        \item Impulsmessungen an beiden Eingängen und Ausgängen
        \item Definition eines Modellgerüsts mit fehlenden Parametern
        \item Finden der Modellparameter mit Hilfe des genetischen Algorithmus
        \item Validierung des Modells
        \item Umwandlung des Modells in den Zustandsraum
        \item Verbesserung des Modells im Arbeitsbereich
        \item Normierung der Eingänge auf einen Eingangsbereich von $[-10, 10]$ im C++ Code
    \end{itemize}
}


% ========================================
%     Impulsmessungen
% ========================================
\horizontalLine
\paragraph{Impulsmessungen}
\minipagedOrBelowEachOther
{
    Eine Sprungantwort ist für dieses Prozess nicht geeignet für die Systemidentifikation, 
    da der Encoder-Wert bereits das zweite Integral der Stellgrösse darstellt und deshalb ein Sprung 
    am Eingang zu einem linearen Anstieg am Ausgang führen würde.
    Deshalb werden kurze Impulse am Eingang des Prozesses angelegt und die Antwort des Prozesses aufgezeichnet.
    \ref{fig:ImpulsEingangMotorMitSchwungmasse} zeigt die angelegten Impulse am den beiden Eingängen des Prozesses.
    Für die Identifizierung des Modells wurden diese Impulse mit unterschiedlichen Amplituden verwendet:
    \[
    [500, 1000, 1500, 2000, 2500, 3000]
    \]
}
{
    \vspace{-1cm}
    \begin{figure}[H]
        \center
        \includegraphics[width=\textwidth]{images/MotorMitSchwungmasse/MotorMitSchwungmasseSystemIdentifikationsmessungInputs.pdf}
        \caption{Eingansimpulse $2000$ des Motor-mit-Schwungmasse-Prozesses}
        \label{fig:ImpulsEingangMotorMitSchwungmasse}
    \end{figure}
}
\minipagedOrBelowEachOther
{
    \vspace{1cm}
    \ref{fig:ImpulsantwortenMotorMitSchwungmasse} zeigt die Antwort des Prozesses auf die angelegten Impulse.
    Die Encoder-Werte sind in Radianten dargestellt und verlaufen jeweils mit gegensätzlichem Vorzeichen.

}
{
    \begin{figure}[H]
        \center
        \includegraphics[width=\textwidth]{images/MotorMitSchwungmasse/MotorMitSchwungmasseSystemIdentifikationsmessungOutputs.pdf}
        \caption{Antworten des Motor-mit-Schwungmasse-Prozesses auf die Eingangsimpulse $2000$}
        \label{fig:ImpulsantwortenMotorMitSchwungmasse}
    \end{figure}
}


% ========================================
%     Modellgerüst
% ========================================
\newpage
\paragraph{Modellgerüst}
\minipagedOrBelowEachOther
{
    Der Modellentwurf basiert darauf, nur die Struktur des Prozesses festlegen zu müssen und die fehlenden Parameter
    mit Hilfe des genetischen Algorithmus zu finden. 
    Prinzipiell ist das der gleiche Vorgang wie beim Optimieren eines Reglers mit dem genetischen Algorithmus.

    Die im \ref{fig:MotorMitSchwungmasseBlockdiagramm} dargestellte Struktur entsteht aus folgenden Überlegungen:
    \begin{itemize}
        \item Die Eingangsgrössen werden mit den Vorfaktoren $k_1$ und $k_4$ skaliert, um die unterschiedlichen Motorstärken zu berücksichtigen.
        \item Beide Motoren verhalten sich wie PT1-Glieder. $\rightarrow$ Je ein Integrator mit den Vorfaktoren $k_2, k_5$ und Feedback Verbindungen.
        \item Die Umrechnung der Winkelgeschwindigkeit in die Winkelposition erfolgt über einen weiteren Integrator mit den Vorfaktoren $k_3, k_6$.
        \item Die Ausgänge sind die Winkelpositionen der beiden Motoren.
        \item Die Kopplung der beiden Motoren erfolgt über eine Schwungmasse, welche ein Drehmoment proportional zur Winkelabweichung erzeugt.
        Weil die Encoder-Werte ein entgegengesetztes Vorzeichen haben, 
        werden diese Beiden Encoder-Werte addiert. (Summation zwischen den Ausgängen)
        Die über $k_7$ skalierte Winkelabweichung wird auf beide Motoren als Kraft mit einander entgegengesetzten Vorzeichen zurückgeführt.
    \end{itemize}

    Die Grössen $Drehmoment$ und $Winkelgeschwindigkeit$ sind in diesem Modell nicht explizit dargestellt, 
    sondern werden nur implizit über die Integratoren modelliert. 
    Die Tatsächlichen physikalischen Grössen können beliebig skaliert sein im schlussendlichen Modell.
    Da diese Zustände jedoch am echten Prozess nicht gemessen werden, 
    ist auch in diesem Modell der Zugang zu diesen Zuständen nicht vorhanden. 
    Deshalb ist es egal wie diese Zustände skaliert sind.
}
{
    \begin{figure}[H]
        \center
        \begin{tikzpicture}[auto, node distance=2cm,>=latex', scale=0.7, transform shape]
    \newcommand{\labeloffset}{0.2}

    \newcommand{\topPosY}{2}
    \newcommand{\centerPosY}{0}
    \newcommand{\bottomPosY}{-2}

    % Draw grid
    %\draw[step=1cm,gray,very thin] (0,0) grid (10,10);



    \gainNode[east]{inputGain1}{1, \topPosY}{$k_{\mathbf{1}}$};
    \sumNode{sumA}{3,\topPosY};
    \gainNode[east]{inputGain2}{4, \topPosY}{$k_{\mathbf{2}}$};
    \blockNode{intA}{6, \topPosY}{$\frac{1}{s}$};
    \gainNode[east]{preIntB}{8, \topPosY}{$k_{\mathbf{3}}$};
    \blockNode{intB}{10, \topPosY}{$\frac{1}{s}$};
    \connectionNode{postIntA}{7, \topPosY};
    \connectionNode{outputA}{11, \topPosY};
    \connect[label={-1,0}{$u_{\mathbf{1}}$}]{(0, \topPosY) (inputGain1.west)}{->};
    \connect{(inputGain1.east) (sumA.west)}{->};
    \connect{(sumA.east) (inputGain2.west)}{->};
    \connect{(inputGain2.east) (intA.west)}{->};
    \connect{(intA.east) (postIntA.center) (preIntB.west)}{->};
    \connect{(preIntB.east) (intB.west)}{->};
    \connect[label={0.2,0}{$y_{\mathbf{1}}$}]{(intB.east) (outputA.center) ($(outputA.center) + (1, 0)$)}{->};
    \feedback[sign={-0.2,0.2}]{postIntA.center}{1}{sumA.north}{->};




    
    




    % bottom part
    \gainNode[east]{inputGain4}{1, \bottomPosY}{$k_{\mathbf{4}}$};
    \sumNode{sumB}{3,\bottomPosY};
    \gainNode[east]{inputGain5}{4, \bottomPosY}{$k_{\mathbf{5}}$};
    \blockNode{intA}{6, \bottomPosY}{$\frac{1}{s}$};
    \gainNode[east]{preIntB}{8, \bottomPosY}{$k_{\mathbf{6}}$};
    \blockNode{intB}{10, \bottomPosY}{$\frac{1}{s}$};
    \connectionNode{postIntA}{7, \bottomPosY};
    \connectionNode{outputB}{11, \bottomPosY};
    \connect[label={-1,0}{$u_{\mathbf{2}}$}]{(0, \bottomPosY) (inputGain4.west)}{->};
    \connect{(inputGain4.east) (sumB.west)}{->};
    \connect{(sumB.east) (inputGain5.west)}{->};
    \connect{(inputGain5.east) (intA.west)}{->};
    \connect{(intA.east) (postIntA.center) (preIntB.west)}{->};
    \connect{(preIntB.east) (intB.west)}{->};
    \connect[label={0.2,0}{$y_{\mathbf{2}}$}]{(intB.east) (outputB.center) ($(outputB.center) + (1, 0)$)}{->};
    \feedback[sign={-0.2,-0.2}]{postIntA.center}{-1}{sumB.south}{->};
    
    


    % center part
    \sumNode{angleDivSum}{11, \centerPosY};
    \connect{(outputA.center) (angleDivSum.north)}{->};
    \connect{(outputB.center) (angleDivSum.south)}{->};

    \gainNode[west]{backGain}{6.2, \centerPosY}{$k_{\mathbf{7}}$};
    \connect{(angleDivSum.west) (backGain.east)}{->};
    \connectionNode{backConnection}{3, \centerPosY};
    \connect[sign={-0.2,-0.2}]{(backGain.west)  (backConnection.center) (sumA.south)}{->};
    \connect{(backGain.west) (backConnection.center) (sumB.north)}{->};

    % Draw blocks
   % \sumNode{sum}{1,0};
%
   % \saturationBlockNode{saturation1}{-1, 0}{0}{10};
   % \blockNode{controller}{2.5,0}{$G_c(s)$};
   % \saturationBlockNode{saturation2}{$(controller.east) + (1.5,0)$}{0}{10};
   % \blockNode{plant}{$(saturation2.east) + (2,0)$}{$G(s)=\frac{K1}{T \cdot s + 1}$};
%
   % % Draw connections
   % \connect{(-2.5,0)             (saturation1.west)          }{->};
   % \connect[label={-0.5,\labeloffset}{$\mathbf{r}$}              ]{(saturation1.east)             (sum.west)          }{->};
   % \connect[label={-0.5,\labeloffset}{$\mathbf{e}$}   ]{(sum.east)        (controller.west)   }{->};
   % \connect{(controller.east) (saturation2.west)   }{->};
   % \connect[label={-0.4,\labeloffset}{$\mathbf{u}$}   ]{(saturation2.east) (plant.west)        }{->};
   % \connectionNode{output}{$(plant.east) + (0.5,0)$};
   % \connect[label={0.2,0}{$\mathbf{y}$}               ]{(plant.east) \vectorAdd{output}{0.5,0}}{->};
%
   % \feedback[sign={-0.3,-0.1}              ]{output.center}{-1}{sum.south}{->};
\end{tikzpicture}
        \caption{Blockdiagramm der Regelstrecke Motor-mit-Schwungmasse}
        \label{fig:MotorMitSchwungmasseBlockdiagramm}
    \end{figure}
}


% ========================================
%     Modellparameter finden
% ========================================
\horizontalLine
\paragraph{Modellparameter finden}
\minipagedOrBelowEachOther
{
    Um die fehlenden Modellparameter $k_1$ bis $k_7$ zu finden, wird der genetische Algorithmus verwendet.
    Als \gls{Fitnessfunktion} wird der mittlere quadratische Fehler (MSE) zwischen den gemessenen Ausgangsgrössen
    und den modellierten Ausgangsgrössen verwendet.
    Als Referenz wird das gemessene Prozessverhalten mit der Impulsamplitude von $2000$ verwendet,
    weil sich dieser Wert circa in der Mitte befindet und somit das nichtlineare Verhalten des Prozesses gut mittelt.
}
{
    Der genetische Algorithmus sucht die Parameterkombination, welche den MSE minimiert.
    Die Initialwerte der Parameter wurden von Hand abgeschätzt und als Startpunkt für den Algorithmus verwendet um 
    die Chance zu erhöhen, dass der Algorithmus konvergiert.
}

% ========================================
%     Modellvalidierung
% ========================================
\newpage
\paragraph{Modellvalidierung}
\minipagedOrBelowEachOther
{
    In \ref{fig:MotorMitSchwungmasseModelEvalOutputs} sind die Impulsantworten des realen Prozesses
    mit den Impulsantworten des modellierten und nach den Modellparametern optimierten Prozesses, verglichen.
    Die gestrichelten Linien stellen die Referenzantworten aus den echten Messungen dar.
    Die durchgezogenen Linien stellen das Antwortverhalten des modellierten Prozesses dar.
    
    
}
{
    \vspace{-1cm}
    \begin{figure}[H]
        \center
        \includegraphics[width=\textwidth]{images/MotorMitSchwungmasse/MotorMitSchwungmasseModelEvalOutputs.pdf}
        \caption{Vergleich der Impulsantworten des realen Prozesses mit dem modellierten Prozess}
        \label{fig:MotorMitSchwungmasseModelEvalOutputs}
    \end{figure}
}

\minipagedOrBelowEachOther
{
    \vspace{1cm}
    Die~\ref{fig:MotorMitSchwungmasseModelEvalOutputsAll} zeigt den Vergleich des Modelles mit allen gemessenen Impulsantworten.
    Es ist ersichtlich, dass sich die DC-Verstärkung des realen Prozesses stark nichtlinear verhält.
    Es ist ausserdem ersichtlich, dass sich der reale Prozess, bei kleinen Eingangsgrössen, nicht gleich verhält,
    je nach dem in welche Richtung der Eingang angesteuert wird.
    Die Impulse zum Zeitpunkt $t=1s$ sind positiv und die Impulse zum Zeitpunkt $t=5s$ sind negativ und haben die gleiche Fläche.
    Das bedeutet, dass der Prozess eigentlich wieder an die Ursprungsposition zurückkehren müsste.
    Dies ist jedoch bei kleinen Eingangssignalen nicht der Fall.
    \\
    Das Modell ist nicht so gut wie erhofft und um ein besseres Modell zu erhalten müsste die Modellstruktur angepasst werden
    und weitere Modellparameter hinzugefügt werden. 
    Für die Verwendung in der Reglersynthese ist dieses Modell jedoch ausreichend.
}
{
    \begin{figure}[H]
        \center
        \includegraphics[width=\textwidth]{images/MotorMitSchwungmasse/MotorMitSchwungmasseModelEvalOutputsAllY1.pdf}
        \includegraphics[width=\textwidth]{images/MotorMitSchwungmasse/MotorMitSchwungmasseModelEvalOutputsAllY2.pdf}
        \caption{Stark nichtlineare DC-Verstärkung des realen Prozesses im Vergleich zum modellierten Prozess}
        \label{fig:MotorMitSchwungmasseModelEvalOutputsAll}
    \end{figure}
}


% ========================================
%     Zustandsraummodell
% ========================================
\horizontalLine
\paragraph{Zustandsraummodell}
\minipagedOrBelowEachOther
{
    Das fertige Simulink Modell kann mit dem Befehl \textit{linmod} \cite{matlabLinmod} in ein Zustandsraummodell umgewandelt werden.
    Das Zustandsraummodell kann anschliessend im Simulink und im MATLAB für die Reglersynthese verwendet werden.  
}
{

}


% ========================================
%     Verfeinerung des Modells im Arbeitspunkt
% ========================================
\horizontalLine
\paragraph{Verbesserung des Modells im Arbeitsbereich}
\minipagedOrBelowEachOther
{
    Das Modell verhält sich sehr unterschiedlich im Vergleich zum realen Prozess. 
    Das hat sich im Verlauf der Tests ergeben, deshalb wurde das Modell verfeinert um im verwendeten Arbeitsbereich 
    besser zu funktionieren.
    
    Dafür wurde der reale Prozess im closed loop Betrieb mit einem PID-Regler betrieben.
    Der PID-Regler wurde mit dem alten Modell entworfen und anschliessend auf den realen Prozess angewendet.
    Das Verhalten am realen Prozess mit dem PID-Regler wurde aufgezeichnet. 
}
{
    Anschliessend wurde das Modell wieder mit Hilfe des genetischen Algorithmus verfeinert.
    Dabei wurde natürlich auch das geschlossene System mit dem gleichen PID-Regler verwendet.
    Das Ergebnis war ein Modell, welches sich im verwendeten Arbeitsbereich deutlich besser verhält.
}



% ========================================
%     Normierung der Eingänge
% ========================================
\horizontalLine
\paragraph{Normierung der Eingänge}
\minipagedOrBelowEachOther
{
    Die Eingänge des Modells werden auf einen Bereich von $[-10, 10]$ normiert.
    Der Hauptgrund dafür ist, die dadurch verbesserte Darstellung in den Diagrammen.
}
{
    Die Normierung wird durch die vorgeschalteten Verstärkungen von $300$ erreicht.
    Dadurch wird der Eingangsbereich von $[-10, 10]$ auf $[-3000, 3000]$ skaliert,
    was dem Arbeitsbereich des realen Prozesses entspricht.
}


% ========================================
%     Implementierung im C++ Code
% ========================================
\newpage
\paragraph{Implementierung im C++ Code}
\begin{cppcode}{Update Funktion des Motors Mit Schwungmasse im Code}{lst:MotorMitSchwungmasseDefinitionCPP}
// Update Funktion des Modells im C++ Code. class DCMotorWithMassSystem
void update(double deltaTime) override
{
    /*
        m_parameters = { 
            0.0637851, 
            -0.0232653, 
            189.219, 
            -0.207233, 
            18.8808, 
            -0.809018, 
            -0.228862 };
    */
	double input1Scaled = m_inputs[0] * 300 * m_parameters[0];
	double input2Scaled = m_inputs[1] * 300 * m_parameters[1];

	double angleDifferenceTorque = (m_disk1.angle + m_disk2.angle) * m_parameters[2];

	double torque1 = (input1Scaled - m_disk1.angularVelocity - angleDifferenceTorque) * m_parameters[3];
	double torque2 = (input2Scaled - m_disk2.angularVelocity + angleDifferenceTorque) * m_parameters[4];

	double lastAngularVelocity1 = m_disk1.angularVelocity * m_parameters[5];
	double lastAngularVelocity2 = m_disk2.angularVelocity * m_parameters[6];
	m_disk1.angularVelocity = TimeBasedSystem::getIntegrated_Bilinear(m_disk1.lastPreIntegrationTorque, torque1, m_disk1.angularVelocity, deltaTime);
	m_disk2.angularVelocity = TimeBasedSystem::getIntegrated_Bilinear(m_disk2.lastPreIntegrationTorque, torque2, m_disk2.angularVelocity, deltaTime);


	m_disk1.angle = TimeBasedSystem::getIntegrated_Bilinear(lastAngularVelocity1, m_disk1.angularVelocity * m_parameters[5], m_disk1.angle, deltaTime);
	m_disk2.angle = TimeBasedSystem::getIntegrated_Bilinear(lastAngularVelocity2, m_disk2.angularVelocity * m_parameters[6], m_disk2.angle, deltaTime);
	m_outputs[0] = m_disk1.angle;
	m_outputs[1] = m_disk2.angle;


	m_disk1.lastPreIntegrationTorque = torque1;
	m_disk2.lastPreIntegrationTorque = torque2;
}
\end{cppcode}