% Pakete nur laden, wenn sie nicht bereits geladen sind
\makeatletter
\@ifpackageloaded{inputenc}{}{%
  \usepackage[utf8]{inputenc}%
}
\@ifpackageloaded{babel}{}{%
  \usepackage[ngerman]{babel}%
}
\@ifpackageloaded{amsmath}{}{%
  \usepackage{amsmath}%
}
\@ifpackageloaded{hyperref}{}{%
  \usepackage{hyperref}%
}

% Zähler für Formeln mit Herleitungen
\newcounter{formulaDerivationcounter}

% Liste zum Speichern der Herleitungen
\newcommand{\formulaDerivationlist}{}

% Makro zum DEKLARIEREN einer Formel mit Herleitung
% \formulaWithDerivationDeclaration{label}{formel}{herleitung}
\newcommand{\formulaWithDerivationDeclaration}[3]{%
  % Prüfen, ob diese Formel bereits deklariert wurde
  \@ifundefined{formula@#1}{%
    % Formel speichern
    \expandafter\gdef\csname formula@#1\endcsname{#2}%
    % Herleitung zur Liste hinzufügen und als definiert markieren
    \expandafter\gdef\csname derivation@#1\endcsname{1}%
    \g@addto@macro\formulaDerivationlist{%
      \subsubsection{Herleitung von \ref{form:#1}}%
      \label{deriv:#1}%
      #3%
      \vspace{1em}%
    }%
  }{%
    \PackageWarning{formulaDerivation}{Formel '#1' wurde bereits deklariert und wird ignoriert}%
  }%
}

% Makro zum ANZEIGEN einer deklarierten Formel
% \formulaWithDerivationShow{label}
\newcommand{\formulaWithDerivationShow}[1]{%
  \@ifundefined{formula@#1}{%
    \PackageError{formulaDerivation}{Formel '#1' wurde nicht deklariert! Verwenden Sie zuerst \string\formulaWithDerivationDeclaration\space um die Formel zu definieren.}{%
      %
    }%
  }{%
    % Prüfen, ob die Formel bereits angezeigt wurde (und somit eine Nummer hat)
    \@ifundefined{formulanumber@#1}{%
      % Erste Anzeige: normale Nummerierung
      \begin{equation}
        \label{form:#1}%
        \csname formula@#1\endcsname
      \end{equation}%
      % Nummer speichern für spätere Verwendungen
      \expandafter\xdef\csname formulanumber@#1\endcsname{\theequation}%
    }{%
      % Weitere Anzeigen: gleiche Nummer mit \tag
      \begin{equation}
        \tag{\csname formulanumber@#1\endcsname}%
        \csname formula@#1\endcsname
      \end{equation}%
    }%
    % Referenz zur Herleitung hinzufügen
    \begin{center}
      {\footnotesize Herleitung siehe \hyperref[deriv:#1]{Anhang \ref{deriv:#1}}}
    \end{center}
  }%
}

% Alternatives Makro: Formel inline anzeigen (ohne equation-Umgebung)
\newcommand{\formulaWithDerivationShowInline}[1]{%
  \@ifundefined{formula@#1}{%
    \PackageError{formulaDerivation}{Formel '#1' wurde nicht deklariert}{}%
  }{%
    $\csname formula@#1\endcsname$%
  }%
}

% Das ursprüngliche Makro bleibt für Abwärtskompatibilität erhalten
\newcommand{\formulaWithDerivation}[3]{%
  \formulaWithDerivationDeclaration{#1}{#2}{#3}%
  \formulaWithDerivationShow{#1}%
}

\makeatother

% Makro zum Anzeigen aller Herleitungen im Anhang
\newcommand{\showFormulaDerivations}{%
  \appendix
  \section{Herleitungen der Formeln}\label{sec:herleitungen}
  \formulaDerivationlist
}