\documentclass[a4paper,12pt]{article}


% Pakete nur laden, wenn sie nicht bereits geladen sind
\makeatletter
\@ifpackageloaded{inputenc}{}{%
  \usepackage[utf8]{inputenc}%
}
\@ifpackageloaded{babel}{}{%
  \usepackage[ngerman]{babel}%
}
\@ifpackageloaded{amsmath}{}{%
  \usepackage{amsmath}%
}
\@ifpackageloaded{hyperref}{}{%
  \usepackage{hyperref}%
}

% Zähler für Formeln mit Herleitungen
\newcounter{formulaDerivationcounter}

% Liste zum Speichern der Herleitungen
\newcommand{\formulaDerivationlist}{}

% Makro zum DEKLARIEREN einer Formel mit Herleitung
% \formulaWithDerivationDeclaration{label}{formel}{herleitung}
\newcommand{\formulaWithDerivationDeclaration}[3]{%
  % Prüfen, ob diese Formel bereits deklariert wurde
  \@ifundefined{formula@#1}{%
    % Formel speichern
    \expandafter\gdef\csname formula@#1\endcsname{#2}%
    % Herleitung zur Liste hinzufügen und als definiert markieren
    \expandafter\gdef\csname derivation@#1\endcsname{1}%
    \g@addto@macro\formulaDerivationlist{%
      \subsubsection{Herleitung von \ref{form:#1}}%
      \label{deriv:#1}%
      #3%
      \vspace{1em}%
    }%
  }{%
    \PackageWarning{formulaDerivation}{Formel '#1' wurde bereits deklariert und wird ignoriert}%
  }%
}

% Makro zum ANZEIGEN einer deklarierten Formel
% \formulaWithDerivationShow{label}
\newcommand{\formulaWithDerivationShow}[1]{%
  \@ifundefined{formula@#1}{%
    \PackageError{formulaDerivation}{Formel '#1' wurde nicht deklariert! Verwenden Sie zuerst \string\formulaWithDerivationDeclaration\space um die Formel zu definieren.}{%
      %
    }%
  }{%
    % Prüfen, ob die Formel bereits angezeigt wurde (und somit eine Nummer hat)
    \@ifundefined{formulanumber@#1}{%
      % Erste Anzeige: normale Nummerierung
      \begin{equation}
        \label{form:#1}%
        \csname formula@#1\endcsname
      \end{equation}%
      % Nummer speichern für spätere Verwendungen
      \expandafter\xdef\csname formulanumber@#1\endcsname{\theequation}%
    }{%
      % Weitere Anzeigen: gleiche Nummer mit \tag
      \begin{equation}
        \tag{\csname formulanumber@#1\endcsname}%
        \csname formula@#1\endcsname
      \end{equation}%
    }%
    % Referenz zur Herleitung hinzufügen
    \begin{center}
      {\footnotesize Herleitung siehe \hyperref[deriv:#1]{Anhang \ref{deriv:#1}}}
    \end{center}
  }%
}

% Alternatives Makro: Formel inline anzeigen (ohne equation-Umgebung)
\newcommand{\formulaWithDerivationShowInline}[1]{%
  \@ifundefined{formula@#1}{%
    \PackageError{formulaDerivation}{Formel '#1' wurde nicht deklariert}{}%
  }{%
    $\csname formula@#1\endcsname$%
  }%
}

% Das ursprüngliche Makro bleibt für Abwärtskompatibilität erhalten
\newcommand{\formulaWithDerivation}[3]{%
  \formulaWithDerivationDeclaration{#1}{#2}{#3}%
  \formulaWithDerivationShow{#1}%
}

\makeatother

% Makro zum Anzeigen aller Herleitungen im Anhang
\newcommand{\showFormulaDerivations}{%
  \appendix
  \section{Herleitungen der Formeln}\label{sec:herleitungen}
  \formulaDerivationlist
}



% Dokumentbeginn
\begin{document}

\title{Beispieldokument mit Formeln und Herleitungen}
\author{Autor}
\date{\today}
\maketitle

\section{Einleitung}
Dies ist ein Beispiel für die Verwendung des Makros.

\section{Mathematische Formeln}

\[
  E = mc^2
\]

\subsection{Quadratische Formel}
Die Lösungen einer quadratischen Gleichung $ax^2 + bx + c = 0$ sind gegeben durch:

\formulaWithDerivationDeclaration{quadratic}{
  x_{1,2} = \frac{-b \pm \sqrt{b^2 - 4ac}}{2a}
}{
    
  Wir beginnen mit der quadratischen Gleichung:
  \begin{equation*}
    ax^2 + bx + c = 0
  \end{equation*}
  
  Dividieren durch $a$:
  \begin{equation*}
    x^2 + \frac{b}{a}x + \frac{c}{a} = 0
  \end{equation*}
  
  Quadratische Ergänzung:
  \begin{equation*}
    x^2 + \frac{b}{a}x + \left(\frac{b}{2a}\right)^2 = \left(\frac{b}{2a}\right)^2 - \frac{c}{a}
  \end{equation*}
  
  \begin{equation*}
    \left(x + \frac{b}{2a}\right)^2 = \frac{b^2 - 4ac}{4a^2}
  \end{equation*}
  
  Wurzel ziehen und nach $x$ auflösen:
  \begin{equation*}
    x = -\frac{b}{2a} \pm \frac{\sqrt{b^2-4ac}}{2a} = \frac{-b \pm \sqrt{b^2-4ac}}{2a}
  \end{equation*}
  \newpage
}
% Formel ANZEIGEN (beliebig oft!)
\formulaWithDerivationShow{quadratic}


\subsection{Binomische Formel}
Die erste binomische Formel lautet:

\formulaWithDerivationDeclaration{binomial}{
  (a+b)^2 = a^2 + 2ab + b^2
}{
  Ausmultiplizieren von $(a+b)^2$:
  \begin{align*}
    (a+b)^2 &= (a+b)(a+b) \\
    &= a \cdot a + a \cdot b + b \cdot a + b \cdot b \\
    &= a^2 + ab + ab + b^2 \\
    &= a^2 + 2ab + b^2
  \end{align*}
  \newpage
}
\formulaWithDerivationShow{binomial}
\formulaWithDerivationShow{binomial}
\formulaWithDerivationShow{binomial}

\subsection{Eulersche Formel}
Eine der schönsten Formeln der Mathematik:

\formulaWithDerivationDeclaration{euler}{
  e^{i\pi} + 1 = 0
}{
  Die Eulersche Formel lautet allgemein:
  \begin{equation*}
    e^{ix} = \cos(x) + i\sin(x)
  \end{equation*}
  
  Dies folgt aus den Taylor-Reihen:
  \begin{align*}
    e^{ix} &= \sum_{n=0}^{\infty} \frac{(ix)^n}{n!} \\
    &= 1 + ix - \frac{x^2}{2!} - i\frac{x^3}{3!} + \frac{x^4}{4!} + \ldots
  \end{align*}
  
  Für $x = \pi$:
  \begin{equation*}
    e^{i\pi} = \cos(\pi) + i\sin(\pi) = -1 + 0i = -1
  \end{equation*}
  
  Daher:
  \begin{equation*}
    e^{i\pi} + 1 = 0
  \end{equation*}
  \newpage
}
\formulaWithDerivationShow{euler}

% Anhang mit allen Herleitungen
\showFormulaDerivations

\end{document}