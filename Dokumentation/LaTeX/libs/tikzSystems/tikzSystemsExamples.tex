% LaTeX document for examples of tikzSystems package
\documentclass[tikz,border=10pt]{article}
\usepackage{listings}

% set page width
\usepackage[a4paper, total={170mm,257mm}, left=20mm, top=20mm]{geometry}

\lstset{
  basicstyle=\ttfamily\small,
  backgroundcolor=\color{gray!10},
  frame=single,
  rulecolor=\color{gray!50},
  breaklines=true,
  columns=fullflexible,
  showstringspaces=false,
  keywordstyle=\color{blue},
  commentstyle=\color{green!50!black},
  morekeywords={usepackage,documentclass,newcommand,pgfkeys},
}

% ========================================================================================
% Autor:   Alex Krieg
% Datum:   06.11.2025
% Zweck:   Makros für einfacheres Zeichnen von Regelungssystemen mit TikZ
% ========================================================================================
\usepackage{tikz}
\usepackage{xkeyval}
\usepackage{etoolbox}
\usetikzlibrary{shapes,arrows,positioning,calc}
\usetikzlibrary{shapes,arrows,positioning}


%\tikzstyle{block} = [draw, rectangle, minimum height=1cm, minimum width=1cm]
\tikzstyle{input} = [coordinate]
\tikzstyle{output} = [coordinate]
\tikzstyle{sum} = [draw, circle, node distance=1cm, minimum size=0.5cm]

\tikzset{
  block/.style={
    draw,
    thick,
    rectangle,
    minimum height=1cm, 
    minimum width=1cm,
    font=\large
  }
}

\tikzset{
  gain block/.style={
    draw,
    thick,
    shape=isosceles triangle,
    isosceles triangle apex angle=60,
    minimum height=1cm,
    minimum width=1cm,
    inner sep=0pt,
    outer sep=0pt,
    shape border rotate=0,
    anchor=west,
    font=\large
  }
}


\newcommand{\tikzSystemStyleSignScale}{0.8} % Scale for minus sign in feedback and connect commands
\newcommand{\tikzSystemStyleSumNodeScale}{0.8} % Offset for minus sign in feedback and connect commands

\newcommand{\drawGrid}[2]
{
    \definecolor{lightGrayColor}{rgb}{0.8,0.8,0.8};
    \draw[step=1cm,lightGrayColor,very thin] (0,0) grid (#1,#2);
}


% Param 1: position
% Param 2: label text
% Usage: \drawLabel{position}{label text}
\newcommand{\drawLabel}[2]{%
    \node[font=\large] at (#1) {#2};
}

% Param 1: from coordinate
% Param 2: to coordinate
% Param 3: color
% Usage: \definecolor{color}{rgb}{0.8,0.8,0.8};
%        \drawArrow{from}{to}{color}
\newcommand{\drawArrow}[3]{%
    \draw[->, color=#3] (#1) -- (#2);
}

% Custom command to add two vectors
% Param 1: point1
% Param 2: point2
% Usage: \vectorAdd{point1}{point2}
\newcommand{\vectorAdd}[2]{
    ($(#1) + (#2)$)
}




% Custom command to draw a block with label and formula
% Param 1: node name
% Param 2: position
% Param 3: label text
% Usage: \blockNode{node_name}{position}{label}
\newcommand{\blockNode}[3]{
    \node [block] at (#2) (#1) {#3};
}

% Custom command to draw a summation node
% Param 1: node name
% Param 2: position
% Usage: \sumNode{node_name}{position}
\newcommand{\sumNode}[2]{
    \node [sum, scale=\tikzSystemStyleSumNodeScale] at (#2) (#1) {$+$};
}

% Custom command to draw a filled connection node at a coordinate
% Param 1: node name
% Param 2: position
% Usage: \connectionNode{node_name}{position}
\newcommand{\connectionNode}[2]{
    \node[circle, fill, inner sep=1.5pt] (#1) at (#2) {};
}



%% ============================================
%% COMMAND: \feedback
%% ============================================
% Custom command to draw feedback connection with optional minus sign
% Optional parameters:
%   label={<xOffset,yOffset>}{text}  - Adds a label at the position (xOffset,yOffset) relative to the to_coordinate with the text "text"
%   sign={<xOffset,yOffset>}         - Adds a minus sign at the position (xOffset,yOffset) relative to the to_coordinate
% Usage: \feedback{from_coordinate}{yOffsetOfLoop}{to_coordinate}{arrow style}
% example: \feedback{output}{yOffsetOfLoop}{sum}{->};
%          \feedback{7,5}{-2}{3,5}{->};
\makeatletter
% Define keys for feedback command
\define@key{feedbackparams}{label}{\feedback@parselabel#1\@nil}
\def\feedback@parselabel#1#2\@nil{%
  \def\feedback@labelcoords{#1}%
  \def\feedback@labeltext{#2}%
}
\define@key{feedbackparams}{sign}{\def\feedback@sign{#1}}

% Helper to check if label was set for feedback
\newcommand{\feedback@iflabelset}[2]{%
  \@ifundefined{feedback@labelcoords}{#2}{#1}%
}

% Helper to check if parameter was set for feedback
\newcommand{\feedback@ifparamset}[3]{%
  \@ifundefined{feedback@#1}{#3}{#2}%
}

% Reset parameters for feedback
\newcommand{\feedback@resetparams}{%
  \let\feedback@labelcoords\undefined
  \let\feedback@labeltext\undefined
  \let\feedback@sign\undefined
}
\newcommand{\feedback}[5][]{%
    \feedback@resetparams% Clear previous values
    \setkeys{feedbackparams}{#1}% Parse the optional parameters
    \begingroup
        % draw the path and create a coordinate at the path's end
        \draw[#5] (#2) -| ($ (#2) + (0,#3) $) -| (#4);

        % Check if label was set
        \feedback@iflabelset{%
        \node[font=\small] at ($(#4)+(\feedback@labelcoords)$) {\feedback@labeltext};
        }{}
        
        % Check if sign was set
        \feedback@ifparamset{sign}{%
        \node[scale=\tikzSystemStyleSignScale] at ($(#4)+(\feedback@sign)$) {$-$};
        }{}
    \endgroup
}
\makeatother




%% ============================================
%% COMMAND: \connect
%% ============================================
% Parameters:
%  Optional parameter:
%   label={<xOffset,yOffset>}{text}  - Adds a label at the position (xOffset,yOffset) relative to the end of the path with the text "text"
%   sign={<xOffset,yOffset>}         - Adds a minus sign at the position (xOffset,yOffset) relative to the end of the path
%  Mandatory parameters:
%   {<coordinate list>}               - A list of coordinates or node names to connect, e.g. (A) (B) (C) ...
%   {<arrow style>}                   - The arrow style, e.g. ->, <->, etc.
%
% Example usage:
%   \connect{(0,0) (3,2) (5,1) (7,3)}{->};
%   \connect[label={0.5,0.5}{\textbf{Path}}]{(0,0) (3,2) (5,1) (7,3)}{->};
%   \connect[sign={0.5,0.5}]{(0,0) (3,2) (5,1) (7,3)}{->};  
\makeatletter
% Define keys for connect command
\define@key{connectparams}{label}{\connect@parselabel#1\@nil}
\def\connect@parselabel#1#2\@nil{%
  \def\connect@labelcoords{#1}%
  \def\connect@labeltext{#2}%
}
\define@key{connectparams}{sign}{\def\connect@sign{#1}}

% Helper to check if label was set for connect
\newcommand{\connect@iflabelset}[2]{%
  \@ifundefined{connect@labelcoords}{#2}{#1}%
}

% Helper to check if parameter was set for connect
\newcommand{\connect@ifparamset}[3]{%
  \@ifundefined{connect@#1}{#3}{#2}%
}

% Reset parameters for connect
\newcommand{\connect@resetparams}{%
  \let\connect@labelcoords\undefined
  \let\connect@labeltext\undefined
  \let\connect@sign\undefined
}

% The connect command
\newcommand{\connect}[3][]{%
  \connect@resetparams% Clear previous values
  \setkeys{connectparams}{#1}% Parse the optional parameters
  \begingroup
    % draw the path and create a coordinate at the path's end
    \draw[#3] plot coordinates {#2} coordinate (connect@last);
    
    % Check if label was set
    \connect@iflabelset{%
      \node[font=\small] at ($(connect@last)+(\connect@labelcoords)$) {\connect@labeltext};
    }{}
    
    % Check if sign was set
    \connect@ifparamset{sign}{%
      \node[scale=\tikzSystemStyleSignScale] at ($(connect@last)+(\connect@sign)$) {$-$};
    }{}
  \endgroup
}

\makeatother


























% Custom command to draw a gain block
% Param 1: node name
% Param 2: position
% Param 3: gain text
% Param 4 (optional): orientation (east, south, west, north) - default: east
% Usage: \gainNode[orientation]{node_name}{x,y}{K}
\newcommand{\gainNode}[4][east]{
    \ifstrequal{#1}{east}{%
        \node[gain block, shape border rotate=0, anchor=center] at (#3) (#2) {#4};
    }{}
    \ifstrequal{#1}{south}{%
        \node[gain block, shape border rotate=-90, anchor=center] at (#3) (#2) {#4};
    }{}
    \ifstrequal{#1}{west}{%
        \node[gain block, shape border rotate=180, anchor=center] at (#3) (#2) {#4};
    }{}
    \ifstrequal{#1}{north}{%
        \node[gain block, shape border rotate=90, anchor=center] at (#3) (#2) {#4};
    }{}
  
}


%\makeatletter
%% Define keys for gainNode command
%\define@key{gainNodeParams}{label}{\gainNode@parselabel#1\@nil}
%\def\gainNode@parselabel#1#2\@nil{%
%  \def\gainNode@labelcoords{#1}%
%  \def\gainNode@labeltext{#2}%
%}
%\define@key{gainNodeParams}{showCoordinates}{\def\gainNode@showCoordinates{#1}}
%\define@key{gainNodeParams}{scale}{\def\gainNode@scale{#1}}
%
%% Helper to check if label was set for gainNode
%\newcommand{\gainNode@iflabelset}[2]{%
%  \@ifundefined{gainNode@labelcoords}{#2}{#1}%
%}
%
%% Helper to check if parameter was set for gainNode
%\newcommand{\gainNode@ifparamset}[3]{%
%  \@ifundefined{gainNode@#1}{#3}{#2}%
%}
%
%% Reset parameters for gainNode
%\newcommand{\gainNode@resetparams}{%
%  \let\gainNode@labelcoords\undefined
%  \let\gainNode@labeltext\undefined
%  \let\gainNode@showCoordinates\undefined
%  \let\gainNode@scale\undefined
%}
%
%% Debugging command to mark coordinates and show their values
%% Usage: \debugCoordinate{coordinate_name}
%\newcommand{\gainNode}[3][]{%
%    \gainNode@resetparams% Clear previous values
%    \setkeys{gainNodeParams}{#1}% Parse the optional parameters (now #1 is the optional arg)
%
%    % if scale is not set, set it to 1
%    \gainNode@ifparamset{scale}{%
%      % scale is set, do nothing
%    }{%
%      \def\gainNode@scale{1}%
%    }%
%
%    \draw[fill=white] \vectorAdd{(1,0)}{} -- (0,0.5) -- (0,-0.5) -- cycle;
%    % Gain-Text in der Mitte
%    \node at (0.3,0) {$K$};
%}
%
%\makeatother













\makeatletter

% Define keys for debugCoordinate command
\define@key{debugCoordinateParams}{label}{\debugCoordinate@parselabel#1\@nil}
\def\debugCoordinate@parselabel#1#2\@nil{%
  \def\debugCoordinate@labelcoords{#1}%
  \def\debugCoordinate@labeltext{#2}%
}
\define@key{debugCoordinateParams}{showCoordinates}{\def\debugCoordinate@showCoordinates{#1}}
\define@key{debugCoordinateParams}{scale}{\def\debugCoordinate@scale{#1}}

% Helper to check if label was set for debugCoordinate
\newcommand{\debugCoordinate@iflabelset}[2]{%
  \@ifundefined{debugCoordinate@labelcoords}{#2}{#1}%
}

% Helper to check if parameter was set for debugCoordinate
\newcommand{\debugCoordinate@ifparamset}[3]{%
  \@ifundefined{debugCoordinate@#1}{#3}{#2}%
}

% Reset parameters for debugCoordinate
\newcommand{\debugCoordinate@resetparams}{%
  \let\debugCoordinate@labelcoords\undefined
  \let\debugCoordinate@labeltext\undefined
  \let\debugCoordinate@showCoordinates\undefined
  \let\debugCoordinate@scale\undefined
}

% Debugging command to mark coordinates and show their values
% Usage: \debugCoordinate{coordinate_name}
\newcommand{\debugCoordinate}[2][]{%
    \debugCoordinate@resetparams% Clear previous values
    \setkeys{debugCoordinateParams}{#1}% Parse the optional parameters (now #1 is the optional arg)

    % if scale is not set, set it to 1
    \debugCoordinate@ifparamset{scale}{%
      % scale is set, do nothing
    }{%
      \def\debugCoordinate@scale{1}%
    }%

    % #2 is now the coordinate name
    \begingroup
      % Draw cross at coordinate
      \draw[red, scale=\debugCoordinate@scale] ($(#2) + (-0.1, -0.1)$) -- ($(#2) + (0.1, 0.1)$);
      \draw[red, scale=\debugCoordinate@scale] ($(#2) + (-0.1, 0.1)$) -- ($(#2) + (0.1, -0.1)$);

      % Label coordinate name
      %\node[red, above right, scale=0.5] at ($(#2)+(0.1,0.0)$) {\detokenize{#2}};
      
      % Check if custom label was set
      \debugCoordinate@iflabelset{%
        \node[red, scale=0.5] at ($(#2)+(\debugCoordinate@labelcoords)$) {\debugCoordinate@labeltext};
      }{}

      % Check if sign was set
    \debugCoordinate@ifparamset{showCoordinates}{%
      \node[red, scale=0.5] at ($(#2)+(\debugCoordinate@showCoordinates)$) {\detokenize{#2}};
    }{}
    \endgroup
}

\makeatother


% Custom command to draw a normally closed switch
% Param 1: (optional) orientation (east, south, west, north) - default: east
% Param 2: position
% Usage: \switchNC[orientation]{x,y}
\newcommand{\switchNC}[2][east]{
    \begin{scope}[shift={(#2)}, scale=0.25]    
      \ifstrequal{#1}{east}{%
          \draw (0,0) -- (1,0);
          \draw (1,0) -- (3.2,0.8);  % slanted line showing closed position
          \draw (3,0) -- (4,0);
          \draw (3,0) -- (3,1);
      }{}
      \ifstrequal{#1}{south}{%
          \draw (0,-4) -- (0,-3);
          \draw (0,-1) -- (0.8,-3.2);  % slanted line showing closed position
          \draw (0,-1) -- (0,0);
          \draw (0,-3) -- (1,-3);
      }{}
      \ifstrequal{#1}{west}{%
          \draw (0,0) -- (1,0);
          \draw (3,0) -- (0.8,0.8);  % slanted line showing closed position
          \draw (3,0) -- (4,0);
          \draw (1,0) -- (1,1);
      }{}
      \ifstrequal{#1}{north}{%
          \draw (0,0) -- (0,1);
          \draw (0,1) -- (0.8,3.2);  % slanted line showing closed position
          \draw (0,3) -- (0,4);
          \draw (0,3) -- (1,3);
      }{}
    \end{scope}
}

\newcommand{\switchNO}[2][east]{
    \begin{scope}[shift={(#2)}, scale=0.25]    
      \ifstrequal{#1}{east}{%
          \draw (0,0) -- (1,0);
          \draw (1,0) -- (3.2,0.8);  
          \draw (3,0) -- (4,0);
      }{}
      \ifstrequal{#1}{south}{%
          \draw (0,-4) -- (0,-3);
          \draw (0,-1) -- (0.8,-3.2);  
          \draw (0,-1) -- (0,0);
      }{}
      \ifstrequal{#1}{west}{%
          \draw (0,0) -- (1,0);
          \draw (3,0) -- (0.8,0.8);  
          \draw (3,0) -- (4,0);
      }{}
      \ifstrequal{#1}{north}{%
          \draw (0,0) -- (0,1);
          \draw (0,1) -- (0.8,3.2); 
          \draw (0,3) -- (0,4);
      }{}
    \end{scope}
}

\newcommand{\switchTwo}[2][east]{
    \begin{scope}[shift={(#2)}, scale=0.25]    
      \ifstrequal{#1}{east}{%
          \draw (4,2)   -- (3,2);      % In1 
         % \draw (4,0)   -- (3,0);      % In2
          \draw (4,-2)  -- (3,-2);      % In3
          \draw (3,2)   -- (3,1);
          \draw (3,-2)   -- (3,-1);
          \draw (1,0)   -- (3.2,1.4);  % slanted line showing closed position
          \draw (1,0)   -- (0,0);
      }{}
      \ifstrequal{#1}{south}{%
          \draw (2,-4)  -- (2,-3);      % In1 
       %   \draw (0,-4)   -- (0,-3);      % In2
          \draw (-2,-4)  -- (-2,-3);      % In3
          \draw (2,-3)   -- (1,-3);
          \draw (-2,-3)   -- (-1,-3);
          \draw (0,-1)   -- (1.4,-3.2);  % slanted line showing closed position
          \draw (0,-1)   -- (0,0);
      }{}
      \ifstrequal{#1}{west}{%
          \draw (-4,2)   -- (-3,2);      % In1 
          %\draw (-4,0)   -- (-3,0);      % In2
          \draw (-4,-2)  -- (-3,-2);      % In3
          \draw (-3,2)   -- (-3,1);
          \draw (-3,-2)   -- (-3,-1);
          \draw (-1,0)   -- (-3.2,1.4);  % slanted line showing closed position
          \draw (-1,0)   -- (0,0);          
      }{}
      \ifstrequal{#1}{north}{%
          \draw (2,4)  -- (2,3);      % In1 
        %  \draw (0,4)   -- (0,3);      % In2
          \draw (-2,4)  -- (-2,3);      % In3
          \draw (2,3)   -- (1,3);
          \draw (-2,3)   -- (-1,3);
          \draw (0,1)   -- (1.4,3.2);  % slanted line showing closed position
          \draw (0,1)   -- (0,0);
      }{}
    \end{scope}
}

\newcommand{\switchThree}[2][east]{
    \begin{scope}[shift={(#2)}, scale=0.25]    
      \ifstrequal{#1}{east}{%
          \draw (4,2)   -- (3,2);      % In1 
          \draw (4,0)   -- (3,0);      % In2
          \draw (4,-2)  -- (3,-2);      % In3
          \draw (3,2)   -- (3,1);
          \draw (3,-2)   -- (3,-1);
          \draw (1,0)   -- (3.2,1.4);  % slanted line showing closed position
          \draw (1,0)   -- (0,0);
      }{}
      \ifstrequal{#1}{south}{%
          \draw (2,-4)  -- (2,-3);      % In1 
          \draw (0,-4)   -- (0,-3);      % In2
          \draw (-2,-4)  -- (-2,-3);      % In3
          \draw (2,-3)   -- (1,-3);
          \draw (-2,-3)   -- (-1,-3);
          \draw (0,-1)   -- (1.4,-3.2);  % slanted line showing closed position
          \draw (0,-1)   -- (0,0);
      }{}
      \ifstrequal{#1}{west}{%
          \draw (-4,2)   -- (-3,2);      % In1 
          \draw (-4,0)   -- (-3,0);      % In2
          \draw (-4,-2)  -- (-3,-2);      % In3
          \draw (-3,2)   -- (-3,1);
          \draw (-3,-2)   -- (-3,-1);
          \draw (-1,0)   -- (-3.2,1.4);  % slanted line showing closed position
          \draw (-1,0)   -- (0,0);          
      }{}
      \ifstrequal{#1}{north}{%
          \draw (2,4)  -- (2,3);      % In1 
          \draw (0,4)   -- (0,3);      % In2
          \draw (-2,4)  -- (-2,3);      % In3
          \draw (2,3)   -- (1,3);
          \draw (-2,3)   -- (-1,3);
          \draw (0,1)   -- (1.4,3.2);  % slanted line showing closed position
          \draw (0,1)   -- (0,0);
      }{}
    \end{scope}
}


% Custom command to draw a saturation block
% Param 1: node name
% Param 2: position
% Param 3: lower limit text
% Param 4: upper limit text
% Usage: \saturationBlockNode{node_name}{x,y}{lower_limit}{upper_limit}
\newcommand{\saturationBlockNode}[4]{
    \begin{scope}[shift={(#2)}, scale=0.25]  
        % draw light gray
        \definecolor{lightGrayColor}{rgb}{0.8,0.8,0.8};
        \draw[lightGrayColor,very thin, dashed] (-2,-1) -- (2,-1);
        \draw[lightGrayColor,very thin, dashed] (-2,1)  -- (2, 1);

        \draw plot coordinates {(-2,-1) (-1,-1) (1,1) (2,1)};
       %\draw (-2,-1)  -- (-1,-1);
       %\draw (-1,-1)  -- (1,1);
       %\draw (1,1)  -- (2,1);

        \node[font=\small] at (0,-1.9) {#3};
        \node[font=\small] at (0, 1.9) {#4};
    \end{scope}
    
    \node[draw, rectangle, minimum height=1.5cm, minimum width=1.3cm, thick] at (#2) (#1) {};
}



\newcommand{\drawOrigin}
{
    \definecolor{lightGrayColor}{rgb}{0.8,0.8,0.8};
    \debugCoordinate[label={-0.4,0.1}{Origin}, showCoordinates={-0.4,-0.1}]{0,0};
    \drawArrow{0,0}{0,1}{lightGrayColor};
    \drawArrow{0,0}{1,0}{lightGrayColor};
}
\newcommand{\drawExampleBackground}{
    \drawOrigin;
    \drawGrid{16}{4};
}

\begin{document}


%% ============================================
%% COMMAND: \blockNode
%% ============================================
\section{blockNode}
\begin{tikzpicture}[auto, node distance=2cm,>=latex', scale=1]
    \drawExampleBackground;
    
    % Param1 is node name
    % Param2 is position
    % Param3 is label
    \blockNode{controller}{3,2}{$G_c(s)$};    

    % Debugging stuff...
    \definecolor{dbgArrowsColor}{rgb}{0.8,0.8,0.0};
    \drawArrow{0,0}{controller.center}{dbgArrowsColor};


    % Mark blockNode coordinates
    \debugCoordinate[label={0.0,-0.2}{.center}]{controller.center};
    \debugCoordinate[label={0.0,0.2}{.north}]{controller.north};
    \debugCoordinate[label={0.4,0.0}{.east}]{controller.east};
    \debugCoordinate[label={0.0,-0.2}{.south}]{controller.south};
    \debugCoordinate[label={-0.4,0.0}{.west}]{controller.west};

    \debugCoordinate[label={-0.6,-0.2}{.south west}]{controller.south west};
    \debugCoordinate[label={0.6,-0.2}{.south east}]{controller.south east};
    \debugCoordinate[label={0.6,0.2}{.north east}]{controller.north east};
    \debugCoordinate[label={-0.6,0.2}{.north west}]{controller.north west};
\end{tikzpicture}
\begin{lstlisting}[language={[LaTeX]TeX}]
% Param1 is node name
% Param2 is position
% Param3 is label
\blockNode{controller}{3,2}{$G_c(s)$};
\end{lstlisting}

%% ============================================
%% COMMAND: \sumNode
%% ============================================
\section{sumNode}
\begin{tikzpicture}[auto, node distance=2cm,>=latex', scale=1]
    \drawExampleBackground;
    
    % Param1 is node name
    % Param2 is position
    % Param3 is label
    \sumNode{sum}{3,2};    

    % Debugging stuff...
    \definecolor{dbgArrowsColor}{rgb}{0.8,0.8,0.0};
    \debugCoordinate[label={-0.4,0.1}{Origin}, showCoordinates={-0.4,-0.1}]{0,0};
    \drawArrow{0,0}{sum.center}{dbgArrowsColor};


    % Mark blockNode coordinates
    \debugCoordinate[label={0.0,-0.3}{.center}, scale=0.5]{sum.center};
    \debugCoordinate[label={0.0,0.2}{.north}, scale=0.5]{sum.north};
    \debugCoordinate[label={0.4,0.0}{.east}, scale=0.5]{sum.east};
    \debugCoordinate[label={0.0,-0.3}{.south}, scale=0.5]{sum.south};
    \debugCoordinate[label={-0.4,0.0}{.west}, scale=0.5]{sum.west};

    \debugCoordinate[label={-0.6,-0.2}{.south west}, scale=0.5]{sum.south west};
    \debugCoordinate[label={0.6,-0.2}{.south east}, scale=0.5]{sum.south east};
    \debugCoordinate[label={0.6,0.2}{.north east}, scale=0.5]{sum.north east};
    \debugCoordinate[label={-0.6,0.2}{.north west}, scale=0.5]{sum.north west};
\end{tikzpicture}
\begin{lstlisting}[language={[LaTeX]TeX}]
% Param1 is node name
% Param2 is position
% Param3 is label
\sumNode{sum}{3,2};  
\end{lstlisting}


%% ============================================
%% COMMAND: \connect
%% ============================================
\section{connect}
\subsection{simple path connection}
\begin{tikzpicture}[auto, node distance=2cm,>=latex', scale=1]
    \drawExampleBackground;
    
    % Param1 is a list of points to connect
    % Param2 is arrow style
    \connect{(0,0) (3,2) (5,1) (7,3)}{->};   
\end{tikzpicture}
\begin{lstlisting}[language={[LaTeX]TeX}]
% Param1 is a list of points to connect
% Param2 is arrow style
\connect{(0,0) (3,2) (5,1) (7,3)}{->}; 
\end{lstlisting}

\subsection{path with label}
\begin{tikzpicture}[auto, node distance=2cm,>=latex', scale=1]
    \drawExampleBackground;
    
    % Optional parameter "label" specifies label position and text
    % label={<xOffset,yOffset>}{text}
    % xOffset and yOffset are relative to the last point in the path
    % The text origin is in the center of the text box
    \connect[label={0.5,0.5}{\textbf{Path}}]{(0,0) (3,2) (5,1) (7,3)}{->};   
\end{tikzpicture}
\begin{lstlisting}[language={[LaTeX]TeX}]
% Optional parameter "label" specifies label position and text
% label={<xOffset,yOffset>}{text}
% xOffset and yOffset are relative to the last point in the path
% The text origin is in the center of the text box
\connect[label={0.5,0.5}{\textbf{Path}}]{(0,0) (3,2) (5,1) (7,3)}{->}; 
\end{lstlisting}

\subsection{path with minus sign}
\begin{tikzpicture}[auto, node distance=2cm,>=latex', scale=1]
    \drawExampleBackground;
    
    % Optional parameter "label" specifies label position and text
    % sign={<xOffset,yOffset>}
    % xOffset and yOffset are relative to the last point in the path
    % The text origin is in the center of the text box
    \connect[sign={0.5,0.5}]{(0,0) (3,2) (5,1) (7,3)}{->};   
\end{tikzpicture}
\begin{lstlisting}[language={[LaTeX]TeX}]
% Optional parameter "label" specifies label position and text
% sign={<xOffset,yOffset>}
% xOffset and yOffset are relative to the last point in the path
% The text origin is in the center of the text box
\connect[sign={0.5,0.5}]{(0,0) (3,2) (5,1) (7,3)}{->};
\end{lstlisting}

%% ============================================
%% COMMAND: \feedback
%% ============================================
\newpage
\section{feedback}
\subsection{simple}
\begin{tikzpicture}[auto, node distance=2cm,>=latex', scale=1]
    \drawExampleBackground;

    % Param1 is from_coordinate
    % Param2 is yOffsetOfLoop
    % Param3 is to_coordinate
    % Param4 is arrow style
    \feedback{5,3}{-1}{2,3}{->};

    % Debugging stuff...
    \definecolor{dbgArrowsColor}{rgb}{1,0.5,0};
    \drawArrow{6,3}{6,2}{dbgArrowsColor};
    \drawLabel{8.5,2.5}{Param2: yOffsetOfLoop = -1};

\end{tikzpicture}
\begin{lstlisting}[language={[LaTeX]TeX}]
% Param1 is from_coordinate
% Param2 is yOffsetOfLoop
% Param3 is to_coordinate
% Param4 is arrow style
\feedback{5,3}{-1}{2,3}{->};
\end{lstlisting}

\subsection{with label and minus sign}
\begin{tikzpicture}[auto, node distance=2cm,>=latex', scale=1]
    \drawExampleBackground;

    % Param1 is from_coordinate
    % Param2 is yOffsetOfLoop
    % Param3 is to_coordinate
    % Param4 is arrow style
    \feedback[sign={-0.25,-0.25}, label={1.5,-1.2}{\textbf{Feedback}}]{5,3}{-1}{2,3}{->};

    % Debugging stuff...
    \definecolor{dbgArrowsColor}{rgb}{1,0.5,0};
    \drawArrow{6,3}{6,2}{dbgArrowsColor};
    \drawLabel{8.5,2.5}{Param2: yOffsetOfLoop = -1};
\end{tikzpicture}
\begin{lstlisting}[language={[LaTeX]TeX}]
% Param1 is from_coordinate
% Param2 is yOffsetOfLoop
% Param3 is to_coordinate
% Param4 is arrow style
\feedback[sign={-0.25,-0.25}, label={1.5,-1.2}{\textbf{Feedback}}]{5,3}{-1}{2,3}{->};
\end{lstlisting}


%% ============================================
%% COMMAND: \connectionNode
%% ============================================
\section{connectionNode}
\begin{tikzpicture}[auto, node distance=2cm,>=latex', scale=1]
    \drawExampleBackground;

    % Param1 is node name
    % Param2 is position
    \connectionNode{nodeName}{3,2};
\end{tikzpicture}
\begin{lstlisting}[language={[LaTeX]TeX}]
% Param1 is node name
% Param2 is position
\connectionNode{nodeName}{3,2};
\end{lstlisting}


%% ============================================
%% Example system
%% ============================================
\section{Example system}
\subsection{Simple feedback system}
\begin{tikzpicture}[auto, node distance=2cm,>=latex', scale=1]
    \drawExampleBackground;

    % Draw blocks
    \sumNode{sum}{1,3};
    \blockNode{controller}{2.5,3}{$G_c(s)$};
    \blockNode{plant}{5,3}{$G(s)$};

    % Draw connections
    \connect[label={-0.5,0.25}{r}]{(0,3) (sum.west)}{->};
    \connect[label={-0.5,0.25}{e}]{(sum.east) (controller.west)}{->};
    \connect[label={-0.5,0.25}{u}]{(controller.east) (plant.west)}{->};
    \connectionNode{output}{6,3};
    \connect[label={-0.5,0.25}{y}]{(plant.east) \vectorAdd{output}{0.5,0}}{->};

    \feedback[sign={-0.25,-0.25}]{output}{-1}{sum}{->};


    
\end{tikzpicture}
\begin{lstlisting}[language={[LaTeX]TeX}]
% Draw blocks
\sumNode{sum}{1,3};
\blockNode{controller}{2.5,3}{$G_c(s)$};
\blockNode{plant}{5,3}{$G(s)$};

% Draw connections
\connect[label={-0.5,0.25}{r}]{(0,3) (sum.west)}{->};
\connect[label={-0.5,0.25}{e}]{(sum.east) (controller.west)}{->};
\connect[label={-0.5,0.25}{u}]{(controller.east) (plant.west)}{->};
\connectionNode{output}{6,3};
\connect[label={-0.5,0.25}{y}]{(plant.east) \vectorAdd{output}{0.5,0}}{->};
\feedback[sign={-0.25,-0.25}]{output}{-1}{sum}{->};
\end{lstlisting}

\subsection{MIMO system}
\begin{tikzpicture}[auto, node distance=2cm,>=latex', scale=1]
    \drawExampleBackground;

    % Draw blocks
    \sumNode{sum}{1,3};
    \blockNode{controller}{2.5,3}{$G_c(s)$};
    \blockNode{plant}{5,3}{$G(s)$};

    %% Draw connections
    %Input signals
    \connect[label={-1.2,0}{r1}   ]{(0,3)   (sum.west)                      }{->};
    \connect[label={-1.2,-0.5}{r2}]{(0,2.5) (0.5,2.5)    (0.5,3)  (sum.west)}{->};

    %Feedback signals
    \connect[label={-0.5,0.25}{e}]{(sum.east) (controller.west)}{->};
    \connect[label={-0.5,0.25}{u}]{(controller.east) (plant.west)}{->};
    \connectionNode{output}{6,3};
    
    %Output signals
    \connect[label={0.2,0}{y1}]{(plant.east) \vectorAdd{output}{1,0}                                                          }{->};
    \connect[label={0.2,0}{y2}]{(plant.east) \vectorAdd{output}{0.5,0} \vectorAdd{output}{0.5,-0.5} \vectorAdd{output}{1,-0.5}}{->};

    \feedback[sign={-0.25,-0.25}]{output}{-1}{sum}{->};


    
\end{tikzpicture}
\begin{lstlisting}[language={[LaTeX]TeX}]
% Draw blocks
\sumNode{sum}{1,3};
\blockNode{controller}{2.5,3}{$G_c(s)$};
\blockNode{plant}{5,3}{$G(s)$};
%% Draw connections
%Input signals
\connect[label={-1.2,0}{r1}   ]{(0,3)   (sum.west)                      }{->};
\connect[label={-1.2,-0.5}{r2}]{(0,2.5) (0.5,2.5)    (0.5,3)  (sum.west)}{->};

%Feedback signals
\connect[label={-0.5,0.25}{e}]{(sum.east) (controller.west)}{->};
\connect[label={-0.5,0.25}{u}]{(controller.east) (plant.west)}{->};
\connectionNode{output}{6,3};

%Output signals
\connect[label={0.2,0}{y1}]{(plant.east) \vectorAdd{output}{1,0}                                                          }{->};
\connect[label={0.2,0}{y2}]{(plant.east) \vectorAdd{output}{0.5,0} \vectorAdd{output}{0.5,-0.5} \vectorAdd{output}{1,-0.5}}{->};
\feedback[sign={-0.25,-0.25}]{output}{-1}{sum}{->};
\end{lstlisting}







\begin{tikzpicture}[auto, node distance=2cm,>=latex', scale=1]
    \drawExampleBackground;

     % Blöcke erstellen
  %\gainNode{G1}{0,0}{$K_1$}
  %\gainNode[south]{G2}{3,0}{$K_2$}
  %\gainNode[west]{G3}{6,0}{$K_3$}
  %
  %% Jetzt mit Punkt-Notation
  %\draw[->] (G1.west) -- ++(-1,0);
  %\draw[->] (G1.east) -- (G2.west);
  %\draw[->] (G3.east) -- ++(-1,0);


    %\draw (0,-2) -- (0.5,-2);
    %\draw (0.5,-2) -- (1.5,-2.5);  % slanted line to show switch
    %\draw (1.5,-2) -- (2,-2);
    %\draw[fill=black] (0.5,-2) circle (0.05);
    %\draw[fill=black] (1.5,-2) circle (0.05);

    \switchNC[east]{0,0}

    \switchNC[south]{1,0}

    \switchNC[west]{2,0}

    \switchNC[north]{3,0}

    \switchNO[east]{0,2}

    \switchNO[south]{1,2}

    \switchNO[west]{2,02}

    \switchNO[north]{3,2}

    \switchTwo[north]{5,0}


  
    %\gainNode[east]{G1}{3,0}{$K_1$}
    \saturationBlockNode{satBlock}{7,0}{-10,10}

    % Debugging stuff...
\end{tikzpicture}




\end{document}