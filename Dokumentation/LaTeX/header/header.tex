\usepackage[T1]{fontenc}	% ä,ü...
\usepackage[utf8]{inputenc} % utf-8 Unterstützung
\usepackage[ngerman]{babel} % Silbentrennung und Rechtschreibung Deutsch
\usepackage[{left=1.5cm,right=1.5cm,top=0cm,bottom=0cm}]{geometry} % Seitenränder Titelblatt
\usepackage{pdflscape}
\usepackage[fontsize=8pt]{fontsize}
\usepackage{needspace}
\usepackage{sectsty}
\sectionfont{\raggedright}



%%%%%%%%%%%%%%%%%%%%%%%
%% Packages
%%%%%%%%%%%%%%%%%%%%%%%
\usepackage[page,header]{appendix}
%\usepackage{acronym} 		   % Für Abkürzungsverzeichnis
%\usepackage{glossaries}
%\usepackage[hidelinks]{hyperref}
\usepackage{hyperref}
\usepackage{etoolbox}
\hypersetup{
    colorlinks=true,
    linkcolor=blue,
    citecolor=blue,
    urlcolor=blue!50!black,  % darker, less aggressive blue
}
\usepackage[nonumberlist, nopostdot]{glossaries}
\usepackage{pdfcomment}
\usepackage{cooltooltips}
\usepackage{xifthen}
\renewcommand{\glossarysection}[2][]{} % Remove "Glossar" title when using \printglossaries

%\let\oldgls\gls
%\renewcommand{\gls}[1]{%
%  \ifcsname glo@#1@long\endcsname
%    % For acronyms - use original \gls
%    \pdftooltip{\oldgls{#1}}{\glsentrylong{#1}}%
%  \else
%    % For symbols - manually create hyperlink
%    \oldgls{#1}%
%  \fi
%}


% Redefine \gls to include tooltips
%\let\oldgls\gls
%\renewcommand{\gls}[1]{%
%  \pdftooltip{\oldgls{#1}}{\glsentrylong{#1}}%
%}

% \setglossarystyle{long4col}
\usepackage{glossary-longragged}
\setglossarystyle{longragged3col}
\usepackage{csquotes}
\usepackage{paracol}
\usepackage{pgfplots}
\pgfplotsset{compat=1.18}
\usetikzlibrary{intersections}
\usepgfplotslibrary{fillbetween}


\usepackage{adjustbox} 		   % adjustbox, minipage..
\usepackage{amsmath}   		   % Allgemeine Matheumgebungen
\usepackage{amssymb}  		   % Fonts: msam,msbm, eufm & Mathesymbole, Mengen (lädt automatisch amsfonts)
\usepackage{amsfonts}
\usepackage{array} 			   % Extending the array and tabular environment -> m,b,p,..
%\usepackage{caption}  		   % Verändern der Schriftart von Bildunterschriften
\usepackage{changepage}
\usepackage{epstopdf}
\usepackage{expl3}
\usepackage{float}
\usepackage{framed, color}
\usepackage{fancyhdr}			%Für Kopf & Fusszeile
\usepackage{graphbox}
\usepackage{graphicx}
\usepackage{titletoc}			% Editieren des TOCs muss vor hyperref geladen werden
\usepackage{tocloft}			% Editieren des TOCs muss vor hyperref geladen werden
\usepackage{hyphenat}			%Wortumbruch
\hyphenation{he-lio-trope opos-sum}

\usepackage{minted}          % Erlaubt es Programmcode in der gewünschten Sprache zu hinterlegen (C++, Matlab,..)
\usepackage{longtable}
\usepackage{lastpage}			%Lastpage ref for Footer
\usepackage{marginnote} 		% Für Seitenkommentare \marginnote
\usepackage{mathtools}          % Vor mathabx laden!
\usepackage{mathabx} 			% Package mit vielen weiteren Mathe Symbolen
\usepackage{mparhack}  			% Improved marginpar placement
\usepackage{multicol} 			% In­ter­mix sin­gle and mul­ti­ple columns
\usepackage{multirow} 			% Create tabular cells spanning multiple rows
\usepackage{paralist}
\usepackage{pdfpages} 
\usepackage{pxfonts} 			% Mathsymbols
\usepackage[section]{placeins}  % Float-Barrier for Section
\usepackage{rotating} 			% Rotation tools, including rotated fullpage floats
\usepackage[onehalfspacing]{setspace}
\usepackage{scrhack}        	% Fixes koma-script incompatibilities
\usepackage{subcaption}
\usepackage{tabularx}
\usepackage{textcomp} 			% Wird für Copyright-Symbol,Währungen, Musikalische-Symbole benötigt
\usepackage[colorinlistoftodos,prependcaption,textsize=tiny]{todonotes}
\usepackage[most]{tcolorbox}
\usepackage{trfsigns}
\usepackage{varwidth}
\usepackage{wrapfig}
\usepackage{fontawesome5} % for note icon


\usepackage{circuitikz}
\usepackage{tikz}
\usetikzlibrary{shapes.geometric, arrows, fit, positioning,arrows.meta,decorations.pathreplacing,calc}
\usepackage{animate}


\usepackage[table]{xcolor} % für Farben in Tabellen
\usepackage{pgf}           % für Farbinterpolation
\usepackage{colortbl}      % für farbige Tabellenzellen
\usepackage{booktabs}      % schönere Tabellenlinien
\usepackage{pgffor}         % für Schleifen
\usepackage{etoolbox}       % für Listen-Operationen
\usepackage{pifont}
\usepackage{graphicx}
\tcbuselibrary{vignette}

%%%%%%%%%%%%%%%%%%%%%%%
% Caption Setup
%%%%%%%%%%%%%%%%%%%%%%%
\captionsetup[figure]{labelfont={it,bf},textfont={it}}
\captionsetup[table]{labelfont={it,bf},textfont={it},singlelinecheck=off,justification=centering}
\captionsetup[lstlisting]{labelfont={it,bf},textfont={it}}
\captionsetup[subfigure]{labelfont=bf,textfont=normalfont,singlelinecheck=off,justification=centering}

\newenvironment{nscenter}
{\parskip=-5pt\par\nopagebreak\centering}
{\parskip=-15pt\par\noindent\ignorespacesafterend}

%Rewritte Referenze Style and TOF,TOT
\renewcommand{\thefigure}{Abb. \arabic{figure}}
\renewcommand{\thetable}{Tab. \arabic{table}}
\renewcommand{\theequation}{Formel \arabic{equation}}
\newcommand{\myref}[1]{Abschnitt~\ref{#1}}

\addto{\captionsngerman}{%
    %	\renewcommand*{\contentsname}{Inhalt}
    %	\renewcommand*{\listfigurename}{Abbildungen}
    %	\renewcommand*{\listtablename}{Tabellen}
    \renewcommand*{\figurename}{} %Delet Figure name 
    \renewcommand*{\tablename}{}
}
\setlength{\cftfignumwidth}{2cm}	% Width of equation number in List of Figure
\setlength{\cfttabnumwidth}{2cm}	% Width of equation number in List of Table

%List of Equations
\newcommand{\listequationsname}{Gleichungen}
\newlistof{myequations}{equ}{\listequationsname}
\newcommand{\myequations}[1]{% 
    \addcontentsline{equ}{myequations}{\protect\numberline{\theequation}#1}\par}
\setlength{\cftmyequationsnumwidth}{2cm}% Width of equation number in List of Equations



%%%%%%%%%%%%%%%%%%%%%%%%
%% Header and Footer %%
%%%%%%%%%%%%%%%%%%%%%%%%
%\pagestyle{fancy} %eigener Seitenstil
\renewcommand{\sectionmark}[1]{\markright{#1}} %entfernt nummer vor section
\renewcommand{\subsectionmark}[1]{}

\fancypagestyle{plain}{
    \fancyhf{} % Alle Kopf- und Fußzeilenfelder löschen
    \fancyhead[C]{} %zentrierte Kopfzeile
    \fancyhead[L]{\LogoIET}
    \fancyhead[R]{\LogoOST}
    \renewcommand{\headrulewidth}{0.4pt} %obere Trennlinie
    
    \fancyfoot[L]{Seite \thepage}
    \fancyfoot[R]{\finalday}
    \fancyfoot[C]{\Author}
    \renewcommand{\footrulewidth}{0.4pt} %untere Trennlinie
}

\fancypagestyle{docstyle}{
    \fancyhf{} %alle Kopf- und Fußzeilenfelder bereinigen
    \fancyhead[C]{\textsl{\rightmark}} %zentrierte Kopfzeile
    \fancyhead[L]{\LogoIET}
    \fancyhead[R]{\LogoOST}
    \renewcommand{\headrulewidth}{0.4pt} %obere Trennlinie
    
    \fancyfoot[L]{Seite \thepage}
    \fancyfoot[R]{\finalday}
    \fancyfoot[C]{\Author}
    \renewcommand{\footrulewidth}{0.4pt} %untere Trennlinie
}

\fancypagestyle{empty}{
    \fancyhf{} %alle Kopf- und Fußzeilenfelder bereinigen
    \fancyhead[C]{} %zentrierte Kopfzeile
    \fancyhead[L]{}
    \fancyhead[R]{}
    \renewcommand{\headrulewidth}{0pt} %obere Trennlinie
    
    \fancyfoot[L]{}
    \fancyfoot[R]{}
    \fancyfoot[C]{}
    \renewcommand{\footrulewidth}{0pt} %untere Trennlinie
}
\pagestyle{docstyle} %Pagesytle


%%%%%%%%%%%%%%%%%%%%%
%% Load HSR Color  %%
%%%%%%%%%%%%%%%%%%%%%
\usepackage{xcolor}
\usepackage{header/HSRColors}

%%%%%%%%%%
% Colors %
%%%%%%%%%%
\definecolor{black}{rgb}{0,0,0}
\definecolor{red}{rgb}{1,0,0}
\definecolor{white}{rgb}{1,1,1}
\definecolor{grey}{rgb}{0.8,0.8,0.8}
\definecolor{green}{rgb}{0,.8,0.05}
\definecolor{brown}{rgb}{0.603,0,0}
\definecolor{mymauve}{rgb}{0.58,0,0.82}
\definecolor{mygreen}{RGB}{28,172,0}
\definecolor{mygray}{rgb}{0.5,0.5,0.5}
\definecolor{mymauve}{rgb}{0.58,0,0.82}
\definecolor{mylilas}{RGB}{170,55,241}

\definecolor{gray80}{gray}{0.8}
\definecolor{gray60}{gray}{0.6}
\definecolor{gray40}{gray}{0.4}
\definecolor{gray20}{gray}{0.2}

%%%%%%%%%%%%%%%%%%%%%
%% Title %%
%%%%%%%%%%%%%%%%%%%%%



%Title Spacing-----------------------------------------
\usepackage{titlesec}
\titleformat{\section}[hang]
  {\normalfont\Large\bfseries}
  {\thesection}
  {1em}
  {}
%\titlespacing{name=\section}{-\marginparwidth}{0pt}{0.2em}
%\titlespacing{name=\subsection}{-\marginparwidth+10pt}{0pt}{0.2em}
%\titlespacing{name=\subsubsection}{-\marginparwidth+14pt}{0.2em}{0.2em}
%\titlespacing{name=\paragraph}{-\marginparwidth+18pt}{0.2em}{0.2em}

\titlespacing{name=\section}{1pt}{0pt}{0.2em}
\titlespacing{name=\subsection}{1pt}{0pt}{0.2em}
\titlespacing{name=\subsubsection}{1pt}{0.2em}{0.2em}
\titlespacing{name=\paragraph}{1pt}{0.2em}{0.2em}

%%%%%%%%%%%%%%%%%%
%% Bibliography %%
%%%%%%%%%%%%%%%%%%
%\usepackage[fixlanguage]{babelbib}
%\selectbiblanguage{german}


\usepackage[backend=bibtex, style=numeric, sorting=ydnt, urldate=iso, seconds=true]{biblatex}
%\usepackage[backend=bibtex,style=ieee, defernumbers=true]{biblatex} %Do not use biber with TexWorks
\addbibresource{Literatur}


% Abbildungen im Quellenverzeichnis nach Alphabet ordnen, Rest nummerieren
%\DeclareFieldFormat{labelnumber}{\ifkeyword{abb}{\mknumalph{#1}}{#1}}


%%%%%%%%%%%%%%%%%%%%%%%%%%%%%%%%%%%
%% Itemize and Enumerate spacing %%
%%%%%%%%%%%%%%%%%%%%%%%%%%%%%%%%%%%
% \topsep: space between first item and preceding paragraph
% \partopsep: extra space added to \topsep when environment starts a new paragraph
% \itemsep: space between successive items. 
\usepackage{enumitem} % Controls Layout of itemize, enumerate, description
\setlist[itemize]{topsep=0pt,itemsep=-1ex,partopsep=1ex,parsep=1ex,after=\vskip0.1\baselineskip}
\setlist[enumerate]{topsep=0pt,itemsep=-1ex,partopsep=1ex,parsep=1ex,after=\vskip0.1\baselineskip}

%%%%%%%%%%%
%% Index %%
%%%%%%%%%%%
\usepackage{imakeidx}
\makeindex[intoc,columnseprule]
\indexsetup{firstpagestyle=plain}    % Show header/footer on index page

%-------------------------------------------------
% Marginalien/Seitenränder
%-------------------------------------------------
%\marginpar{Eine Randnotiz}
\newcommand{\marg}[1]{\marginpar{\raggedright \textbf{#1} }}	

%%%%%%%%%%%%%%%%%%%%%%%
%% Aligned footnotes %%
%%%%%%%%%%%%%%%%%%%%%%%
\usepackage[hang]{footmisc}
\setlength{\footnotemargin}{1em}

%%%%%%%%%%%%%
%% Tabular %%
%%%%%%%%%%%%%
\newcolumntype{L}[1]{>{\raggedright\arraybackslash}p{#1}} % Tabelleninhalt linksausgerichtet
\newcolumntype{R}[1]{>{\raggedleft\arraybackslash}p{#1}} % Tabelleninhalt rechtsausgerichtet
\newcolumntype{C}[1]{>{\centering\arraybackslash}p{#1}} %  Tabelleninhalt zentriert


%%%%%%%%%%%%%%%%%%%%%%
%% Generelle Makros %%
%%%%%%%%%%%%%%%%%%%%%%

% If \Print=true, then make all links black for nicer print
\providecommand*{\True}{true}
\ifx \Print \True
\hypersetup{hidelinks, colorlinks, linkcolor = black, citecolor = black, filecolor = black, urlcolor = black}
\fi

\parindent0pt % Zeileneinzug verhindern

%Matlab font
\newcommand{\matlab}[1]{\footnotesize{(Matlab: \texttt{#1})}\normalsize{}}

% Makro für Tabellenbilder gleich unterhalb der Linie
\newcommand\tabbild[2][]{%
	\raisebox{0pt}[\dimexpr\totalheight+\dp\strutbox\relax][\dp\strutbox]{%
		\includegraphics[#1]{#2}%
	}%
}

% Makro für Vorteile und Nachteil mit Plus und Minus
\newcommand\pro{\item[$+$]}
\newcommand\con{\item[$-$]}


%Float-Barrier for Subsection
\makeatletter
\AtBeginDocument{%
	\expandafter\renewcommand\expandafter\subsection\expandafter{%
		\expandafter\@fb@secFB\subsection
	}%
}
\makeatother


%%%%%%%%%%%%%%%%%%%%%%%%%%%%
% Mathematical Operators %
%%%%%%%%%%%%%%%%%%%%%%%%%%%%
\DeclareMathOperator{\sinc}{sinc}
\DeclareMathOperator{\sgn}{sgn}
\DeclareMathOperator{\Real}{Re}
\DeclareMathOperator{\Imag}{Im}
\DeclareMathOperator{\euler}{e}
\DeclareMathOperator{\cov}{cov}
\DeclareMathOperator{\PolyGrad}{PolyGrad}
\DeclareMathOperator{\gradient}{grad}
\DeclareMathOperator{\rotation}{rot}
\DeclareMathOperator{\divergenz}{div}
\DeclareMathOperator{\imag}{j}

%Grösse Integral anpassen
\def\Int{\mbox{\Large$\displaystyle\int$\normalsize}}
\def\Int{\mbox{\Large$\displaystyle\iint$\normalsize}}
\def\OInt{\mbox{\Large$\displaystyle\oint$\normalsize}}

%Makro für 'd' von Integral- und Differentialgleichungen 
\newcommand*{\diff}{\mathop{}\!\mathrm{d}}

%%%%%%%%%%%%%%%%%%%%%%%%%%%
% Fouriertransform %
%%%%%%%%%%%%%%%%%%%%%%%%%%%

\unitlength1cm
\newcommand{\FT}
{
	\begin{picture}(1,0.5)
	\put(0.2,0.1){\circle{0.14}}\put(0.27,0.1){\line(1,0){0.5}}\put(0.77,0.1){\circle*{0.14}}
	\end{picture}
}


\newcommand{\IFT}
{
	\begin{picture}(1,0.5)
	\put(0.2,0.1){\circle*{0.14}}\put(0.27,0.1){\line(1,0){0.45}}\put(0.77,0.1){\circle{0.14}}
	\end{picture}
}

%%%%%%%%%%%%%%%%%%%%%%%%%%%
% Costum Functions %
%%%%%%%%%%%%%%%%%%%%%%%%%%%

\pretocmd{\equation}{\noindent}{}{}



% Command for chapter references that were defined using "\label{sec:labelName}"
% Usage: ~\fullref{sec:labelName}
\newcommand{\fullref}[1]{\textbf{\ref{#1} \nameref{#1}}}


%%%%%%%%%%%%%%%%%%%%%%%%%%%
% Matlab Code Highlighting %
%%%%%%%%%%%%%%%%%%%%%%%%%%%
\usepackage{listings}
\usepackage{xcolor}

\definecolor{matlabgreen}{rgb}{0,0.6,0}
\definecolor{matlabblue}{rgb}{0,0,0.9}
\definecolor{matlabpurple}{rgb}{0.5,0,0.5}
\definecolor{mygreen}{RGB}{28,172,0} % color values Red, Green, Blue
\definecolor{mylilas}{RGB}{170,55,241}

\lstdefinelanguage{Matlab}{
  morekeywords={break,case,catch,continue,else,elseif,end,for,function,
    global,if,otherwise,persistent,return,switch,try,while},
  morecomment=[l]% comment style
  %{...}, % block comments
  morestring=[m]', % strings
}

%\lstset{
%  language=Matlab,
%  basicstyle=\ttfamily\small,
%  keywordstyle=\color{matlabblue}\bfseries,
%  stringstyle=\color{matlabpurple},
%  commentstyle=\color{matlabgreen},
%  numbers=left,
%  numberstyle=\tiny\color{gray},
%  stepnumber=1,
%  numbersep=5pt,
%  showstringspaces=false,
%  breaklines=true,
%  frame=single,
%  captionpos=b
%}
%\lstset{language=Matlab,%
%    %basicstyle=\color{red},
%    breaklines=true,%
%    morekeywords={matlab2tikz},
%    keywordstyle=\color{blue},%
%    morekeywords=[2]{1}, keywordstyle=[2]{\color{black}},
%    identifierstyle=\color{black},%
%    stringstyle=\color{mylilas},
%    commentstyle=\color{mygreen},%
%    showstringspaces=false,%without this there will be a symbol in the places where there is a space
%    numbers=left,%
%    numberstyle={\tiny \color{black}},% size of the numbers
%    numbersep=9pt, % this defines how far the numbers are from the text
%    emph=[1]{for,end,break},emphstyle=[1]\color{red}, %some words to emphasise
%    %emph=[2]{word1,word2}, emphstyle=[2]{style},    
%}

% Global style for all code blocks

% Define colors for syntax highlighting
\definecolor{codegreen}{rgb}{0,0.6,0}
\definecolor{codegray}{rgb}{0.5,0.5,0.5}
\definecolor{codepurple}{rgb}{0.58,0,0.82}
\definecolor{backcolour}{rgb}{0.95,0.95,0.92}

% Enable UTF-8 support in listings
\lstset{
    inputencoding=utf8,
    extendedchars=true,
    literate=
        {°}{{\textdegree}}1
        {á}{{\'a}}1 {é}{{\'e}}1 {í}{{\'i}}1 {ó}{{\'o}}1 {ú}{{\'u}}1
        {Á}{{\'A}}1 {É}{{\'E}}1 {Í}{{\'I}}1 {Ó}{{\'O}}1 {Ú}{{\'U}}1
        {à}{{\`a}}1 {è}{{\`e}}1 {ì}{{\`i}}1 {ò}{{\`o}}1 {ù}{{\`u}}1
        {À}{{\`A}}1 {È}{{\`E}}1 {Ì}{{\`I}}1 {Ò}{{\`O}}1 {Ù}{{\`U}}1
        {ä}{{\"a}}1 {ë}{{\"e}}1 {ï}{{\"i}}1 {ö}{{\"o}}1 {ü}{{\"u}}1
        {Ä}{{\"A}}1 {Ë}{{\"E}}1 {Ï}{{\"I}}1 {Ö}{{\"O}}1 {Ü}{{\"U}}1
        {â}{{\^a}}1 {ê}{{\^e}}1 {î}{{\^i}}1 {ô}{{\^o}}1 {û}{{\^u}}1
        {Â}{{\^A}}1 {Ê}{{\^E}}1 {Î}{{\^I}}1 {Ô}{{\^O}}1 {Û}{{\^U}}1
        {ß}{{\ss}}1
}

% Define MATLAB language with proper keywords
\lstdefinelanguage{Matlab}{
    morekeywords={break,case,catch,continue,else,elseif,end,for,function,
        global,if,otherwise,persistent,return,switch,try,while,classdef,
        properties,methods,events},
    morecomment=[l]\%,
    morestring=[m]',
    morestring=[m]"
}



% C++ listing style
%\lstdefinestyle{cppstyle}{
%    backgroundcolor=\color{backcolour},
%    commentstyle=\color{codegreen},
%    keywordstyle=\color{blue},
%    numberstyle=\tiny\color{black},
%    stringstyle=\color{codepurple},
%    basicstyle=\ttfamily\footnotesize,
%    breakatwhitespace=false,
%    breaklines=true,
%    captionpos=b,
%    keepspaces=true,
%    numbers=left,
%    numbersep=10pt,
%    showspaces=false,
%    showstringspaces=false,
%    showtabs=false,
%    tabsize=4,
%    language=C++,
%    frame=l,
%    framesep=3pt,
%    framexleftmargin=17pt,
%    rulecolor=\color{codegray}
%}
% C++ listing style
\lstdefinestyle{cppstyle}{
    backgroundcolor=\color{backcolour},
    commentstyle=\color{codegreen},
    keywordstyle=\color{blue},
    numberstyle=\tiny\color{codegray},
    stringstyle=\color{codepurple},
    basicstyle=\ttfamily\footnotesize,
    breakatwhitespace=false,
    breaklines=true,
    captionpos=b,
    keepspaces=true,
    numbers=left,
    numbersep=5pt,
    showspaces=false,
    showstringspaces=false,
    showtabs=false,
    tabsize=4,
    language=C++
}

% MATLAB listing style
%\lstdefinestyle{matlabstyle}{
%    backgroundcolor=\color{backcolour},
%    commentstyle=\color{codegreen},
%    keywordstyle=\color{blue},
%    numberstyle=\tiny\color{black},
%    stringstyle=\color{codepurple},
%    basicstyle=\ttfamily\footnotesize,
%    breakatwhitespace=false,
%    breaklines=true,
%    captionpos=b,
%    keepspaces=true,
%    numbers=left,
%    numbersep=10pt,
%    showspaces=false,
%    showstringspaces=false,
%    showtabs=false,
%    tabsize=4,
%    language=Matlab,
%    frame=l,
%    framesep=3pt,
%    framexleftmargin=17pt,
%    rulecolor=\color{codegray}
%}
\lstdefinestyle{matlabstyle}{
    backgroundcolor=\color{backcolour},
    commentstyle=\color{codegreen},
    keywordstyle=\color{blue},
    numberstyle=\tiny\color{codegray},
    stringstyle=\color{codepurple},
    basicstyle=\ttfamily\footnotesize,
    breakatwhitespace=false,
    breaklines=true,
    captionpos=b,
    keepspaces=true,
    numbers=left,
    numbersep=5pt,
    showspaces=false,
    showstringspaces=false,
    showtabs=false,
    tabsize=4,
    language=Matlab
}

% Environment for C++ code section
% Usage: \begin{cppcode}{title}{label} 
%  ... 
% \end{cppcode}
\VerbatimEnvironment
\lstnewenvironment{cppcode}[2]{%
    \lstset{style=cppstyle, caption={#1}, label={#2}}
}{}

% Environment for MATLAB code section
% Usage: \begin{matlabcode}{title}{label} 
%  ... 
% \end{matlabcode}
\VerbatimEnvironment
\lstnewenvironment{matlabcode}[2]{%
    \lstset{style=matlabstyle, caption={#1}, label={#2}}
}{}

\VerbatimEnvironment
\lstnewenvironment{matlabcode2}[2]{%
    \lstset{style=matlabstyle, caption={#1}, label={#2}}%
}{}


% Command for including C++ file
% Usage: \cppfile{title}{label}{filepath}
\newcommand{\cppfile}[3]{%
    \lstinputlisting[style=cppstyle, caption={#1}, label={#2}]{#3}
}

% Command for including MATLAB file
% Usage: \matlabfile{title}{label}{filepath}
\newcommand{\matlabfile}[3]{%
    \lstinputlisting[style=matlabstyle, caption={#1}, label={#2}]{#3}
}


\renewcommand{\lstlistingname}{Code} % Change "Listing" to "Code"


%\newtcolorbox{matlabcode2}[2][]{
%    listing only,
%    listing options={style=matlabstyle},
%    title={#1}, label={#2},
%    colback=white, colframe=black!50,
%    left=2mm, right=2mm, top=1mm, bottom=1mm,
%    breakable
%}

%%%%%%%%%%%%%%%%%%%%%%%%%%%
% Horizontal Line Macro  %
% Draw a horizontal line  for separating content %
%%%%%%%%%%%%%%%%%%%%%%%%%%%
\newcommand{\horizontalLine}{
  \noindent\hrulefill
}



\definecolor{brightRed}{RGB}{255,200,200}
\definecolor{brightGreen}{RGB}{200,255,200}
\newcommand{\cmark}{
  \edef\colorname{brightGreen}
  \expandafter\cellcolor\expandafter{\colorname}
  \textcolor{green}{\ding{51}}
  }  % grünes Häkchen
\newcommand{\xmark}{
  \edef\colorname{brightRed}
  \expandafter\cellcolor\expandafter{\colorname}
  \textcolor{red}{\ding{55}}
  }    % rotes Kreuz

% Dynamische Zellenfärbung basierend auf Bewertung (1=rot, 10=grün)
\newcommand{\colcell}[1]{%
  \pgfmathsetmacro{\pct}{round((#1-1)/9*100)}% Prozentwert berechnen
  \edef\colorname{green!\pct!red}% Farbe zusammensetzen
  \expandafter\cellcolor\expandafter{\colorname}\textbf{#1}%
}




% Small helper to draw a chromosome as a row of boxes with bits
% Small helper to draw a chromosome as a row of boxes with bits
\tikzset{
  gene/.style={
    draw,
    minimum width=8mm,
    minimum height=8mm,
    inner sep=0pt,
    font=\ttfamily\small,
    align=center
  }
}

\newcommand{\mathFunction}[2]{%
  \mathbf{\mathrm{#1}}\bigl(#2\bigr)
}

\newcommand{\mathFunctionRaw}[2]{%
  #1\bigl(#2\bigr)
}

\newcommand{\mathFunctionDot}[2]{%
  \dot{\mathbf{\mathrm{#1}}}\bigl(#2\bigr)
}


\newcommand{\mathFunctionD}[2]{%
  \mathbf{\mathrm{#1}}\bigl[#2\bigr]
}

\newcommand{\mathFunctionDRaw}[2]{%
  #1\bigl[#2\bigr]
}

\newcommand{\mathFunctionDDot}[2]{%
  \dot{\mathbf{\mathrm{#1}}}\bigl[#2\bigr]
}


% Box über ganze Breite
\newtcolorbox{notebox}{
    colback=gray!15,
    colframe=gray!50,
    boxrule=0.3pt,
    arc=2mm,
    width=\textwidth,  % <--- volle Seitenbreite
    enhanced,
}

\newcommand{\infoBlock}[1]{%
\begin{notebox}
\textcolor{blue}{\faInfoCircle}\hspace{0.5em}\textbf{Info:}\\[0.3em] % <--- neue Zeile
#1
\end{notebox}
}


\newcommand{\minipagedOrBelowEachOther}[3][t]{%
\begin{figure}[H]
{
    \vspace{-0.2cm}
    %\centering
    \begin{minipage}[#1]{0.48\textwidth}
    #2
    \end{minipage}\hfill
    \begin{minipage}[#1]{0.48\textwidth}
    #3
    \end{minipage}
}
\end{figure}
}


\newcommand{\minipagedWithHFillOrBelowEachOther}[3][t]{%
\begin{figure}[H]
{
    \vspace{-0.2cm}
    %\centering
    \begin{minipage}[#1]{0.48\textwidth}
    #2
    \end{minipage}
    \hfill
    \vrule width 0.5pt % vertical line
    \hfill
    \begin{minipage}[#1]{0.48\textwidth}
    #3
    \end{minipage}
}
\end{figure}
}






