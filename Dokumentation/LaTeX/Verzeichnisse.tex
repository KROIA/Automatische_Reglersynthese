
% Erstelle 2 Verzeichnisse "Abkürzungen" & "Beschreibungen"
\newglossary[gal]{acronyms}{gax}{gad}{Abkürzungen}
\newglossary[gsl]{symbols}{gsx}{gsd}{Beschreibungen}

\makeglossaries

%\newpage % Damit de nöchst Itrag: sgn(x) uf de nöchste Site chunnt und nöd zerhackt wird uf zwei Site... 

% Symbole, Wörter mit genauer Beschreibung
% Usage: \gls{Pol-Nullstellekürzung}
\newglossaryentry{Pol-Nullstelle-Kürzung}{
    type=symbols,
    name={Pol-Nullstelle-Kürzung},
    description={Methode bei welcher versucht wird ein Pol durch eine Nullstelle aufzuheben. 
    In der Theorie verschwindet dadurch der Pol an dieser Position und kann durch einen Pol an einer anderen Stelle
    ersetzt werden. In der Praxis lässt sich diese Methode nur Näherungsweise umsetzen, da die exakte Position des Pols nie genau getroffen werden kann
    und sich der Pol auch verschieben kann mit der Zeit.}
}

\newglossaryentry{PolePlacement}{
    type=symbols,
    name={Polplatzierung},
    description={Methode bei welcher die vorhandenen Pole der geschlossenen Regelstrecke, mit Hilfe der 
    Zustandsrückführung, gezielt in der S-Ebene platziert werden können um das gewünschte 
    Systemverhalten zu erreichen.
    }
}

\newglossaryentry{Beobachter}{
    type=symbols,
    name={Beobachter},
    description={Eine möglichst realistische Kopie des \gls{System}s, welche mit Hilfe der gemessenen Ein- und Ausgangssignale
    den internen Zustand des Systems schätzt. Dies ist notwendig, 
    wenn nicht alle Zustandsgrössen des Systems gemessen werden können.}
}

\newglossaryentry{System}{
    type=symbols,
    name={System},
    text={System}, % Schreibweise im Fliesstext
    plural={Systeme},
    description={
        Ein System bezeichnet in der Regelungstechnik ein technisches Gebilde,
        das Eingangsgrössen in Ausgangsgrössen umwandelt.
        In den meisten Fällen wird ein System als ein \gls{LTI} System modelliert.
    }
}

\newglossaryentry{nichtlineares System}{
    type=symbols,
    name={Nichtlineares System},
    text={nichtlineares System}, % Schreibweise im Fliesstext
    plural={nichtlineare Systeme},
    description={Ein System, bei welchem die Beziehung zwischen Ein- und Ausgang nicht linear im mathematischen Sinne ist.
    Nichtlinearitäten entstehen beispielsweise durch: Sättigungen, Hysteresen, Trigonometrische Funktionen, Totzeiten, etc.
    Da jedes reale System nichtlineares Verhalten aufweist, wird in der Praxis das \gls{Modell} des Systems durch eine 
    linearisierte Näherung beschrieben. Der Arbeitspunkt der Linearisierung ist dabei Anwendungsspezifisch zu wählen.}
}

\newglossaryentry{zeitvariantes System}{
    type=symbols,
    name={Zeitvariantes System},
    text={zeitvariantes System}, % Schreibweise im Fliesstext
    plural={zeitvariante Systeme},
    description={
        Ein System, bei welchem die Beziehung zwischen Ein- und Ausgang nicht konstant ist, sondern sich über die Zeit verändert.
        Alterung des realen \gls{System}s führt zur Veränderung des Systems. Z.B. mehr Reibung durch altes Schmiermittel oder Verschleiss von Teilen.
        In der Regel werden solche Systeme durch zeitkonstante \gls{Modell}e angenähert.
    }
}

\newglossaryentry{minimalphasiges System}{
    type=symbols,
    name={Minimalphasiges System},
    text={minimalphasiges System}, % Schreibweise im Fliesstext
    plural={minimalphasige Systeme},
    description={Ein System ist minimalphasig, wenn alle Nullstellen des \gls{System}s in der linken Halbebene der S-Ebene liegen.
    Ein \textbf{nicht} minimalphasiges System kann daran erkannt werden, dass die Sprungantwort zu beginn des Sprunges in die entgegengesetzte Richtung des Sprunges verläuft.}
    \newpage % Damit de nöchst Itrag: sgn(x) uf de nöchste Site chunnt und nöd zerhackt wird uf zwei Site... 
}

\newglossaryentry{Hyperparameter}{
    type=symbols,
    name={Hyperparameter},
    description={Parameter, die nicht direkt im Optimierungsprozess angepasst werden, sondern vorab festgelegt werden.
    Diese Parameter beeinflussen das Verhalten des Optimierungsprozesses selbst und können Einfluss auf die Qualität der Lösung haben.}
}


\newglossaryentry{Integral Windup}{
    type=symbols,
    name={Integral Windup},
    description={Ein Phänomen bei PID-Reglern, bei dem der I-Anteil aufgrund von Sättigungen am 
    Ausgang des Reglers übermässig ansteigt, was zu einer Verzögerung oder Überschwingung im Regelverhalten führt.
    Der Integrator im Regler weiss nicht, dass der Ausgang gesättigt ist und somit der Aktor nicht mehr steuern kann.
    Dadurch kann es sein, dass der Fehler nie auf Null sinkt wodurch der Integrator immer weiter ansteigt.
    Das Problem dabei ist, wenn der Fehler plötzlich in die andere Richtung kippt, weil z.B. die Referenz verändert wird,
    wird der Regler am Ausgang immer noch den gesättigten Wert liefern bis der Integrator wieder entladen ist.
    Dies kann zu grossen Überschwingern und langen Einschwingzeiten führen.
    Anti-Windup Mechanismen verhindern dieses Verhalten durch verschiedene Lösungsansätze.}
}

\newglossaryentry{Solver}{
    type=symbols,
    name={Solver},
    description={
        Ein Werkzeug oder ein Algorithmus, das mathematische Probleme löst, 
        entweder indem es eine exakte Lösung findet oder eine annähernde Lösung durch iterative Berechnungen liefert
    }
}

\newglossaryentry{Reglersynthese}{
    type=symbols,
    name={Reglersynthese},
    description={
        Der Prozess der Entwicklung und Gestaltung eines Reglers, 
        der die gewünschten Leistungsanforderungen für ein gegebenes System erfüllt.
    }
}

\newglossaryentry{Modellwissen}{
    type=symbols,
    name={Modellwissen},
    description={
        Das Verständnis und die Kenntnisse über das Verhalten und die Eigenschaften eines \gls{System}s,
        die zur Erstellung eines genauen \gls{Modell}s des Systems verwendet werden können.
        Modellwissen umfasst Informationen über die physikalischen Gesetze, 
        die das System steuern, sowie über die Parameter und Dynamiken des Systems.
    }
}

\newglossaryentry{Gesamtsystem}{
    type=symbols,
    name={Gesamtsystem},
    description={
        Das Gesamtsystem bezeichnet in der Regelungstechnik die Kombination aus dem zu regelnden \gls{Prozess} 
        und dem Regler, der den Prozess steuert.
        Das Gesamtsystem umfasst alle Komponenten, die an der Regelung beteiligt sind,
        einschließlich Sensoren, Aktoren und Rückkopplungsschleifen.
    }
}

\newglossaryentry{Minimierungsproblem}{
    type=symbols,
    name={Minimierungsproblem},
    description={
        \begin{minipage}[t]{\linewidth}
        Ein mathematisches Problem, bei dem das Ziel darin besteht, 
        die Parameter einer Funktion zu finden, welche zu dem kleinsten möglichen Resultat der Funktion führen.
        \begin{equation}
            \min_{x \in \mathcal{X}} f(x)
        \end{equation}
        Wobei:
        \begin{itemize}
            \item $x \in \mathbb{R}^n$ sind die Parameter (Entscheidungsvariablen)
            \item $\mathcal{X} \subseteq \mathbb{R}^n$ ist die zulässige Menge (Beschränkungsmenge)
            \item $f: \mathcal{X} \rightarrow \mathbb{R}$ ist die Zielfunktion
            \item $x^* \in \mathcal{X}$ ist die optimale Lösung
        \end{itemize}
        So dass:
        \begin{equation}
            f(x^*) \leq f(x), \quad \forall x \in \mathcal{X}
        \end{equation}
        \end{minipage} 
    }
}

\newglossaryentry{Maximierungsproblem}{
    type=symbols,
    name={Maximierungsproblem},
    description={
        \begin{minipage}[t]{\linewidth}
            Ein mathematisches Problem, bei dem das Ziel darin besteht, 
            die Parameter einer Funktion zu finden, welche zu dem grösstmöglichen Resultat der Funktion führen.
            \begin{equation}
                \max_{x \in \mathcal{X}} f(x)
            \end{equation}
            Wobei:
            \begin{itemize}
                \item $x \in \mathbb{R}^n$ sind die Parameter (Entscheidungsvariablen)
                \item $\mathcal{X} \subseteq \mathbb{R}^n$ ist die zulässige Menge (Beschränkungsmenge)
                \item $f: \mathcal{X} \rightarrow \mathbb{R}$ ist die Zielfunktion
                \item $x^* \in \mathcal{X}$ ist die optimale Lösung
            \end{itemize}
            So dass:
            \begin{equation}
                f(x^*) \geq f(x), \quad \forall x \in \mathcal{X}
            \end{equation}
        \end{minipage}
    }
}

\newglossaryentry{sgn}{
    type=symbols,
    name={sgn(x)},
    description=
    {
        \needspace{3cm}
        \begin{minipage}[t]{\linewidth}
        Der Signum-Operator gibt das Vorzeichen einer Zahl zurück.
        \begin{equation}
            \mathrm{sgn}(x) = 
            \begin{cases} 
            1 & \text{wenn } x >= 0 \\ 
            -1 & \text{wenn } x < 0 
            \end{cases}
        \end{equation}
        \end{minipage}
    }
}

\newglossaryentry{Konvergenz}{
    type=symbols,
    name={Konvergenz},
    description=
    {
        Der Prozess, bei dem eine Folge von Werten oder eine iterative Methode sich einem festen Punkt oder einer Lösung annähert.
        Im Kontext dieser Arbeit bezieht sich Konvergenz auf den Prozess,
        bei dem ein Optimierungsalgorithmus schrittweise bessere Lösungen findet,
        bis er eine optimale oder zufriedenstellende Lösung erreicht.
    }
}

\newglossaryentry{Fitnessfunktion}{
    type=symbols,
    name={Fitnessfunktion},
    description=
    {
        Eine Funktion, die verwendet wird, um die Qualität oder Eignung einer Lösung in einem Optimierungsproblem zu bewerten.
        In der Regel wird die Fitnessfunktion so gestaltet, dass sie höhere Werte für bessere Lösungen liefert.
        Im Kontext dieser Arbeit wird die Fitnessfunktion jedoch als Minimierungsproblem formuliert,
        sodass niedrigere Werte für bessere Lösungen stehen. 
        Die Fitnessfunktion wird als Sammelbegriff für die Fehler-Minimierungs-Funktion und die Maximierungs-Fitness-Funktion verwendet.
    }
}


\newglossaryentry{Prozess}{
    type=symbols,
    name={Prozess},
    description=
    {
        Ein technisches System oder eine Anlage, die physikalische, chemische oder biologische Vorgänge durchführt.
        In der Regel wird ein Prozess durch Eingangsgrössen beeinflusst und liefert Ausgangsgrössen als Resultat.
        In der Regelungstechnik wird ein Prozess oft als das zu regelnde System betrachtet,
        das durch einen Regler gesteuert wird, um ein gewünschtes Verhalten zu erreichen.
    }
}

\newglossaryentry{LTI}{
    type=symbols,
    name={LTI},
    description=
    {
        Abkürzung für "Linear Time-Invariant" (Linear Zeitinvariant).
        Ein LTI-System ist ein System, das sowohl linear als auch zeitinvariant ist.
        Linearität bedeutet, dass das System die Prinzipien der Superposition erfüllt,
        während Zeitinvarianz bedeutet, dass die Systemparameter sich nicht mit der Zeit ändern.
        LTI-Systeme sind in der Regelungstechnik und Signalverarbeitung weit verbreitet,
        da sie mathematisch gut handhabbar sind und viele nützliche Eigenschaften besitzen.\\
        (Dies ist nur eine Übersichtsbeschreibung und entspricht nicht der genauen Definition von LTI-Systemen)
    }
}


\newglossaryentry{Modell}{
    type=symbols,
    name={Modell},
    description=
    {
        Eine mathematische Darstellung eines realen \gls{System}s oder Prozesses,
        die dessen Verhalten und Eigenschaften beschreibt.
        Modelle sind immer vereinfachte Darstellungen der Realität und können unterschiedliche Komplexitätsgrade aufweisen.
        Modelle werden verwendet, um das Verhalten von Systemen zu analysieren, vorherzusagen und zu steuern.
        In der Regelungstechnik können Modelle in verschiedenen Formen vorliegen,
        wie z.B. Differentialgleichungen, Übertragungsfunktionen oder Zustandsraumdarstellungen.
    }
}


\newglossaryentry{Fitness}{
    type=symbols,
    name={Fitness},
    description=
    {
        Die Fitness ist ein Mass für die Qualität oder Eignung einer Lösung in einem Optimierungsproblem.
        In der Regel wird die Fitnessfunktion so gestaltet, dass höhere Werte bessere Lösungen darstellen.
        Im Kontext dieser Arbeit wird die Fitnessfunktion jedoch als Minimierungsproblem formuliert,
        sodass niedrigere Werte für bessere Lösungen stehen.
        Synonyme dieses Begriffs sind: 
        \begin{itemize}
            \item Fehler
            \item Score 
        \end{itemize}
    }
}

\newglossaryentry{Fehler}{
    type=symbols,
    name={Fehler},
    description=
    {
        Der Fehler ist ein Mass für die Qualität oder Eignung einer Lösung in einem Optimierungsproblem.
        In der Regel wird die Fehlerfunktion so gestaltet, dass sie niedrige Werte bessere Lösungen darstellen.
        Synonyme dieses Begriffs sind: 
        \begin{itemize}
            \item Fitness
            \item Score 
        \end{itemize}
    }
}


\newglossaryentry{Population}{
    type=symbols,
    name={Population},
    description=
    {
        In der Evolutionären Optimierung bezeichnet die Population 
        eine Menge von Individuen (Lösungen), die gleichzeitig optimiert werden.
        Jedes Individuum repräsentiert eine mögliche Lösung des Optimierungsproblems.
        Die Population entwickelt sich über Generationen hinweg durch Auswahl, Kreuzung und Mutation.
    }
}

\newglossaryentry{Individuum}{
    type=symbols,
    name={Individuum},
    description=
    {
        In der Evolutionären Optimierung bezeichnet ein Individuum 
        eine einzelne Lösung innerhalb einer Population.
        Ein Individuum wird durch einen Satz von Parametern (Genotyp) definiert,
        die das Verhalten oder die Eigenschaften der Lösung bestimmen.
        Die Qualität eines Individuums wird durch die Fitnessfunktion bewertet.
    }
}

% Abkürzungen
\newacronym[type=acronyms]{OST}{OST}{Ostschweizer Fachhochschule}
\newacronym[type=acronyms]{SISO}{SISO}{Single Input Single Output}
\newacronym[type=acronyms]{MIMO}{MIMO}{Multiple Input Multiple Output}
%\newacronym[type=acronyms]{Reglersynthese}{Reglersynthese}{Entwicklung eines Reglers}
\newacronym[type=acronyms]{GA}{GA}{Genetischer Algorithmus}
\newacronym[type=acronyms]{DE}{DE}{Differential Evolution}
\newacronym[type=acronyms]{NEAT}{NEAT}{NeuroEvolution of Augmenting Topologies}
\newacronym[type=acronyms]{NN}{NN}{Neuronales Netzwerk}